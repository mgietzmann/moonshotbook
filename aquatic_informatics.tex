\documentclass[11pt,a5paper]{book}
\usepackage[utf8]{inputenc}
\usepackage{amsmath}
\usepackage{amsfonts}
\usepackage{amssymb}
\usepackage{graphicx}
\usepackage[super]{nth}

\title{Aquatic Informatics}
\author{Marcel Gietzmann-Sanders}
\date{}
\setcounter{tocdepth}{1}
\begin{document}
\maketitle
\tableofcontents
\newpage
\chapter{Capture Fisheries}
Wild caught fish - also referred to as capture fisheries - represent a phenomenal opportunity for the likes of Network Earth. As ecosystem services go it is probably the most direct in its nature. Pollination, water filtration, biodiversity - these all have indirect effects on human well being. Powerful effects for sure, but indirect ones. Compare this to capture fisheries where the thing you measure is also the thing you take. Knowing how many fish to expect in the coming years equates quite directly to mouths fed. Therefore, given all the uncertainty in starting something new, this value proposition is an extremely compelling one for an informatics industry - the connection to well being is direct and clear.
\newline

However capture fisheries are also a bit odd in that they are the only form of food that is both commercial \textit{and} wild. Long ago we stopped eating wild cereals and our terrestrial animal protein comes primarily from domesticated animals. Aquaculture has been on the rise over the past few decades - isn't farming our destiny when it comes to food thus putting capture fisheries into the box of quaint holdover from times passed? To understand this we must start by examining how much of our seafood comes from aquaculture today and what the expectations are for the future.

\section{Modeling the Future}

If you're wanting statistics on current consumption and production of foods the FAO (Food and Agriculture Organization) is where you go. Both the FAO \cite{faofish} and the World Bank \cite{wbfish} have modeled expected seafood production and consumption out to 2030. The FAO expects an 18\% increase from 2018 to 2030 which corresponds to a yearly increase in consumption of 1.4\% whereas the World Bank expects a 21\% growth rate between 2011 and 2030 which is a 1\% annual growth rate. Another group - the Blue Food Assessment (BFA) - used the FAO's data as well and predicted a growth of 80\% between 2015 and 2050 \cite{bfafish} - a 1.7\% annual growth rate. 
\newline

Why such a wide range of predictions? To greatly simplify the models used there are effectively four components present (in one way or another) in each of these models. 
\begin{enumerate}
\item Expected supply growth
\item Expected population and income growth
\item Demand elasticities
\item Supply elasticities
\end{enumerate}

The first represents the potential for growth in the supply - things like improvements in technology or policy. The second is driven primarily by changes in developing countries. The third and fourth represent how demand and supply respectively respond to and drive prices of the commodities in question. The former is how much you're willing to buy as prices or your income change, the latter is how much people are willing to produce if prices or costs are adjusted to specific levels. 
\newline

What's important here though is that the first and fourth represent supply growth, the second represents underlying demand growth, and the third represents people's preferences. 
\newline

So with this in mind let's return to our three studies. The BFA study used a "fixed cost" assumption which more or less means that no matter how much or what kind of demand there is, the price of commodities stays the same. This then explains why it is so much higher than the others - it is a supply unconstrained projection - i.e. it is more or less telling us what would happen if tech could keep up with demand and preferences stayed the same. 
\newline

The World Bank's study also included an example where they allowed aquaculture growth to become some 50\% stronger than the trend at the time of the study. They found this added an additional 8.1\% growth to aquaculture or 5\% growth overall. This would correspond to a 1.2\%  annual growth rate - 20\% higher than their original estimate. 
\newline

They also ran a scenario to judge what a change in preferences (demand elasticity in the model) would do. Specifically they allowed consumption of "high value" seafoods (like salmon and shrimp) to go three times higher in China than in the baseline model. This resulted in a whooping 15\% growth overall getting us to a 1.6\% annual growth rate. 
\newline 

To back up the sense that preferences can change so much between the years of 1998 and 2018 fish consumption \textit{per capita} increased by approximately 25\% and a study of growth during that time showed that preference change alone accounted for 60\% of the overall change in fish demand \cite{bfafish}. 
\newline

So what's the point? The point is how much fish will be consumed in the future is highly dependent upon preference and technological growth as well as the present trends of population growth and things like aquaculture expansion. But given all of these estimates and sensitivity analysis it also seems fair to say that between 2023 and 2050 we can expect somewhere between a 30\% and 70\% growth in seafood consumption. 
\newline

However looking at what's happening on land with food security would suggest that the picture is much more likely to lean toward the 70\% side - or rather, will need to.

\section{Back on Land}

Between 2010 and 2050 the FAO projects that an overall 56\% increase in food production will be required to feed the world \cite{nymore}. Assuming we were making steady progress to that goal it would mean a 35\% increase between 2023 and 2050. This seems far closer to the lower end of our seafood consumption spectrum but there's an issue lying in the background. 
\newline 

According to the FAO 90\% of the increases in production \textit{have} to come from agriculture intensification because there simply isn't much arable land left that isn't already being utilized and much of it can't even be used for economically useful crops \cite{faosecurity}. It's in this intensification that the issues crop up.
\newline

The first problem is that the present ways we've intensified agriculture are actually reducing the amount of arable land we have. Between the 1960's and the 1990's a \textit{third} of our planet's arable land was lost to erosion and pollution \cite{lostarable}. Further land is being lost to desertification thanks to climate change of which current commercial agriculture is a big contributor. So the current forms of intensification are taking us steps forward while also taking us steps back. All in all then, we can't afford to just get more intense, we have to become \textit{far} more efficient. 
\newline

One of the biggest and most notable inefficiencies in agriculture today (not to mention one of the most environmentally damaging) is the production of red meat like beef and pork. Indeed today 41\% of cereals are fed to animals and it takes 100 times more land to produce a gram of vegetable protein than a gram of beef protein. But in one study it was shown that while switching the world vegan would reduce our land use from 4.1 billion ha to 1 billion ha, just cutting out beef, lamb, and their derivatives would get us to 1.1 billion ha alone \cite{plantbased}. This is especially useful because as incomes increase interest in animal proteins increase as well \cite{bfafish} - so to have a practical future that uses land more efficiently the shift needs to be from beef, lamb, and pork to poultry and fish.
\newline

To add on top of this there are loads of campaigns that point out the health benefits of eating fish and poultry as compared to fish. The grand total of all of these things? \textit{Fish preferences are going to change.} Even today studies show that beef consumption is on the downturn whereas poultry and fish are on the up and up \cite{wbfish}. 

\section{Reasonable Growth}

At this point you should be fairly convinced how important fisheries are going to be to future food security on this planet. And it should be fairly clear that with the down trend of beef the preferences are going to change thereby pushing us closer to the higher end of our seafood production spectrum. So back to our original question - is aquaculture going to squash capture fisheries? As of 2018 aquaculture represented 46\% of overall seafood consumption and was growing at a rate of 3.2\% globally \cite{faofish}. This would mean that our upper estimate of 2\% growth per year in overall fisheries (or 88\% growth from 2018 to 2050) would require a 290\% growth in aquaculture (because the overfished nature of capture fisheries prevents them for being responsible for the growth). A 290\% growth over 32 years corresponds to a 3.3\% growth rate which lines right up with the 2018 estimate, except for one thing - the rate of growth itself is slowing down. Between 2001 and 2018 the average growth rate was 5.3\% yet 4\% by 2017 and then (as we've mentioned) 3.2\% by 2018. So in order to keep pace we'd need to halt the slowdown in aquaculture growth.
\newline

However this is all assuming capture fisheries stick around. If they don't we'd need a 410\% growth in aquaculture (remember today it's majority capture fisheries) which corresponds to a 4.5\% growth rate or 40\% higher than the growth rate today maintained for over 30 years. 
\newline

Suppose though, for sake of argument, that I'm completely wrong and the low end of the spectrum is true - the 1.1\% growth rate. For aquaculture to soak up all this growth \textit{and} decimate capture fisheries would require a growth rate of 3.5\% annually which brings us right back to our doubts about the ability for aquaculture to keep up with our high end of the spectrum case. In other words, even in this case it seems prudent to keep capture fisheries largely intact.
\newline

This all suggests that rather than being something we'd want to overrun with "more efficient" farming, capture fisheries will remain a very valuable and growth saving asset well into the future. Indeed even in the maximal growth scenario capture fisheries still would constitute a third of total seafood - a proportion not to be messed with. 
\newline

In conclusion then it is philosophically a bit odd that we'll be continuing to capture from the wild well into the future when on land we've seemingly domesticated our every food source - however given our growing populations and income, the need to move away from red meat toward poultry and seafood, and the current rates of aquaculture growth capture fisheries seem to be here to stay. It will remain a necessary component in our strategy to resolve world food security in the future. 

\bibliographystyle{plain}
\bibliography{reference}
\end{document}