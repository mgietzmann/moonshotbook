\documentclass[11pt,a5paper]{book}
\usepackage[utf8]{inputenc}
\usepackage{amsmath}
\usepackage{amsfonts}
\usepackage{amssymb}
\usepackage{graphicx}
\usepackage[super]{nth}

\title{How to Shoot for the Moon and Land in the Stars}
\author{Marcel Gietzmann-Sanders}
\date{}
\setcounter{tocdepth}{1}
\begin{document}
\maketitle
\tableofcontents
\newpage
\chapter{Why I Care - Network Earth}
\section{The Biosphere}
Take a breath. 
\newline

Washing into your lungs is a rich mixture of gasses containing the all important and life giving element - oxygen. While you may traditionally think of oxygen as coming from the plants around you, it turns out that the mixture swirling about in your lungs comes from a much wider and more exotic set of sources. The majority of the oxygen you are breathing comes from our oceans - one in five breaths from a cyanobacterium named Prochlorococcus \cite{kmorsink}. On the complete other end of spectrum are rain-forests like the Amazon - a place that produces something close to one in ten of your breaths \cite{ymalhi}. The result of this is that the brew of gasses swirling around in your lungs has come from creatures smaller than the single strands of hair on your head to those as tall as multistory buildings. However, the story doesn't end there.
\newline

If you're near some kind of urban center, you probably deal with some level of pollution. The daily goings of urban living produce loads of pollutants such as nitrogen and sulfur dioxide. However, plants have a habit of pulling such pollutants out of the air and storing them safely away in their tissues. In fact a 2006 estimate found that in the United States alone urban forests were responsible for more than 3 billion dollars of air purification \cite{dnowak}. So while most of your oxygen was sourced from our oceans and rainforests, it's purification was local and done by the trees and shrubs in your back yard. 
\newline

Take a bite. 
\newline 

If you happen to be eating cassava from South East Asia then you can thank a tiny parasitic wasp - \textit{Anagyrus lopezi} - that is reared in the millions to control populations of a devastating pest - the mealybug \cite{wpark}. An enormous boon for both the financial and food security in the region, the economic value of this millimeter long wasp is more than \$14 billion a year. 
\newline

If instead you're dining on a lovely, flaky slab of wild salmon, you are effectively eating the entire ocean. As predators that sit relatively high in the food chain, they depend on "bait fish" like herring as their primary source of food. Herring feed on things like krill, who in turn eat the very plankton providing us with the majority of our oxygen. If any one of these levels in the food chain starts to experience issues so too does your flaky slab of salmon. 
\newline

Finally, if there are plants on your plate there's a reasonable chance you can thank the leagues of beetles, flies, butterflies, and wasps managing pollination in our world. These services are supplied to an estimated 75\% of crop species, and are worth something upwards of \$200 billion every year \cite{avanbergen}! Without the insects pollinating the plants that produce our food the variety in our diet would shrink dramatically.
\newline

Go on vacation. 
\newline

Diving at a coral reef? The colors, the vivacity, the sheer brilliance of all the fish darting this way and that around you? It's thanks to the sharks. Sharks specifically target sick or unhealthy fish, thereby improving the overall health of fish populations and possibly preventing outbreaks of disease \cite{reefcause}. Furthermore, they help prevent the kinds of population explosions that lead to mass die offs. Just another way they preserve the fish we all love seeing. 
\newline

Hiking alongside giant sequoias or passing through the incredibly diverse south african fynbos? Both of these are only possible thanks to the drama of fire and lightning. Sequoias depend on fires to open their tightly sealed cones and the resultant sprouts depend on the brush clearing power of the fires to give them any chance of seeing the sun \cite{california}. Then, of the more than 9,000 plant species found in South Africa alone, many, like the proteas, also depend on fire to release their seeds and prepare the ground for the next generation of amazing \cite{shoek}. So why do these fires occur? Lightning is the spark, but the reason the fires spread is thanks to the vast diversity of plants that leave behind dried out remains. When it comes to sequoias and proteas, it truly takes a village. 
\newline

Enjoying all the ocean has to offer in Florida? Thank the mangroves. Not only do mangroves provide nurseries for an incredible diversity of creatures including the fish that crowd coral reefs and the manatees and birds that people flock to see (pun intended), but they are also veritable buttresses against coastal damage. One estimate suggested that during Huricane Irma in 2017 they prevented over a billion dollars of flood damage! That's a lot of insurance. 
\newline

Take a step back. 
\newline

These examples are just a tiny taste of how \textit{every single aspect of our lives} is deeply, intrinsically dependent on the living world around us - the biosphere. The Millenium Ecosystem Assessment identified 25 different categories of ecosystem services - and these are as broad as things like food provisioning and erosion control \cite{mas}! From millimeter wasps to three hundred foot tall trees, from ecosystems half a world away to the shrubs in your back yard, our world provides our food, water, air, health, happiness, and so much more. It is our life support system, our spaceship as we hurtle through the vast, cold emptiness that is space and for the first time in our history we have the power to overwhelm it. 

\section{The Anthropocene}
Let's start with some perspective. We as human beings have a very hard time thinking about pretty much any period of time longer than the ones we've already lived. Tell me something is ten years old - no problem I can grasp that. Tell me something is one thousand years old? While I can make it out mathematically I really don't have any way to grasp just how much longer that is. 100,000 years? Forget about it. So let's use a frame of reference that's a little easier to understand. Let's pretend all of our universe's history fits in a single year \cite{csagan}. 

From this point of view (and if we accept standard scientific time lines) the big bang starts us off at midnight January 1. The next four months are spent in a dizzying dance of expanding hot gas, newly forming stars, and, finally, galaxy formation with the milky way only showing up around May \cite{eellis}. But don't assume that because of this our time line is about to speed up because we still have to wait until September before our solar system even forms! We've gotten roughly three quarters through our year and our planet has only \textit{just} shown up. Surprisingly, given how long it's taken to get to this point, it only takes 20 days for the first things we'd deem living to show up, but then things slow down again as we have to wait until December (around 70 more days) before the first multi-cellular organism joins the scene. Now things really start to take off. 

15 days later we've got plants colonizing the land masses of our planet. 3 days after that the first reptiles join them. Only a mere two days later and we've entered the age of the dinosaurs. Think about that, in less than a third of the time it took to figure out how to make multi-cellular organisms we've gone from nothing macroscopic on land to dinosaurs! A day later, our ancestors - the first mammals - join the production and by December 30th (4 days later) the first primates appear. 

But hold on one second. Did I just say December 30th? Does that mean we've got 1 day left in our year and humans haven't even shown up?! That's right. But it gets even wilder. Homo sapiens, us, don't show up until 12 \textit{minutes} to midnight. Just chew on that for a second. All of human history, finding fire, inventing tools, starting the first cities, all the empires, all the wars, absolutely \textit{everything} that our species has ever done occurred in the last 12 minutes of our universe's year. That's the amount of time it takes to heat up a microwave dinner!

But it gets even wilder because we, as a species, didn't figure out how to farm until 28 \textit{seconds} before midnight. So really, everything you've learned about in your human history courses gets packed into about as much time as you spend watching a \textit{single} tiktok. We are but a blip in the history of our universe. But, oh, what a blip we are. 
\newline

From nature's perspective, humans have generally been, with few exceptions, a footnote. There's a reason why the oceans were terrifying to early humans, or why people prayed to gods to stave off floods or famine. If nature decided to have its way, there was little we could do about it. But post the industrial revolution and especially since 1950, that relationship has \textit{utterly} changed \cite{eellis}.

Let's start with some simple numbers. Around 71\% of the land on our planet is considered habitable \cite{hritchie}. Of that habitable land we have put around 46\% of it into agricultural production. That means we use a solid \textit{third} of the planet's surface (not covered by ocean) to just feed ourselves. Except that we eat more than what's on the land. When it comes to fishing the picture is even wilder. Of global fish stocks we overfish 34\% percent and maximally fish 60\% \cite{mroser}. That means there are only 6\% of fish stocks that we could fish more without starting to deplete their numbers!

The amount of water we use is no less incredible than the amount of area we farm. Each year approximately 40,000 $km^3$ of freshwater flows into the ocean \cite{eellis}. Of this only around 13,000 $km^3$ is accessible to humans with our current technology. Yet nearly half of that flows through human engineered systems! So not only are we using half of the accessible land, and pretty much all of the accessible fish, we're using half the available fresh water too. 

In the 2004 the International Geosphere-Biosphere Programme published a report that illustrated what is now called the Great Acceleration \cite{wsteffen}. Across 24 distinct measures from total GDP to atmospheric carbon to the number of rivers dammed they found an explosive spike in activity starting in the 1950's. We've already talked about a few of them (farm, fish, and water use) but let's sprinkle a few more in for color. 

Since around the beginning of the 1900's we've lost 30\% of our forests. We've now dammed close to 25 thousand rivers. We consume about 300 million tons of fertilizer per year. Our global population has gone from 2 to 7 billion. And we have more than 3 times as many floods as we used to per decade. But here's the one that kind of summarizes it all for me - extinction rates have gone through the roof.  
\newline

Extinctions are nothing new and in general pretty normal. As time marches on species come and go and that's... well... just evolution doing its thing. So what matters is not \textit{whether} extinctions are happening but \textit{how many} are happening. 

In the past we've had some pretty spectacular extinction events. Around 444 million years ago the Ordovician-Silurian extinction took out around 70\% of everything living at the time \cite{adubey}. 372 million years ago the Late Devonian extinction took out some 70\% of all marine species. The Permian-Triassic extinction, which occurred roughly 252 million years ago, took out over 70\% of everything alive again. Then in the Late Triassic (208 million years ago) we had yet another mass extinction. Finally we have the most famous of them all, the Cretaceous-Paleogene extinction where 66 million years ago over 60\% of all species died as the result of a massive asteroid smashing into our planet and creating the equivalent of nuclear fallout. Bad for the dinosaurs but great for us as mammals often are attributed their place center stage thanks to that big old space rock. 

Since we were visited by Armageddon-a-la-space extinction rates have pretty much held at their normal, background levels. Until, that is, the Great Acceleration. Since the 1950's extinction rates have skyrocketed to more than 1000 times their background rate \cite{wwf}! At the current rate scientists estimate that 50\% of all species could be gone by 2100. We have entered the sixth extinction. 

Here is where our current impact on our biosphere really comes into perspective. Mass extinctions in the past have occurred as the result of insanely powerful events. Country sized asteroids smashing into the planet. Massive anomalies in global temperatures (sound familiar?). Such incredible dips in the amount of oxygen that more than 70\% of all living species just... well... died of oxygen deprivation. Causing mass extinctions requires planet shaping power. \textit{We now possess that kind of power}. And as a result we've entered an entirely new geological epoch, an epoch where humankind is the domineering, world shaping, power - the Anthropocene \cite{eellis}. And in the perspective of our "universe in a year" we managed this power grab in less time than it takes to watch a tiktok.
\newline

Contrary to how you may be feeling, none of this is meant to scare you (although fear and awe are perfectly appropriate reactions). Instead it's meant to impress upon you the gravity of our situation. In the first section we got a glimpse of just how important the biosphere is to us and now you've got a taste of just how powerful we, as a species, have become - that asteroid that took out the dinosaurs is taking notes from us now. Thankfully this power means that in so much as we can absolutely wreck our world, we can also take care of it too. The problem is not that our world is doomed, but rather that we can no longer just take it for granted. Given our planet shaping capabilities we have no choice but to become active, conscious stewards of our planet. So the question is simple, how do we do that?

\section{A Revolution}
Suppose for a moment you have a garden. While at first it may seem like all you need to do is water your plants and let them flourish, anyone who's tried gardening before knows there's a lot more to it than that. For one thing watering is a tricky business. Water too much and your plants can get water logged and die. Water too little and your plants will wither up and die. Plant your plants too close together and they won't be able to grow properly. Plant them too far apart and you'll be wasting loads of garden space. Then there are the pests. Caterpillars, beetles, mice, deer, you name it it's probably trying to eat your plants and every single one of these needs a different strategy for dealing with. Then there's disease. Plants, like us, can get sick and it's especially likely if they're not getting the right balance of nutrients and care. Diagnosing these sicknesses is difficult enough that there are apps to help you with the task and then there are a whole slew of different methodologies to deal with the various ailments. And we haven't even gotten to the fact that choosing plants can be its own nightmare. Do you have enough sun? Too much sun? Direct or indirect? What's the soil quality, what's your growing zone? On and on it goes. Point is, stewardship of your garden requires a lot of specific knowledge about the plants you a growing, their life histories, the life histories of everything they interact with, the place you live in, and so on. 

So now replace your garden with the entire world. Obviously the amount of knowledge you need is going to skyrocket but something else changes as well. In your garden you've got lots of room to learn by trial and error - if you manage to kill a plant (like I tend to do) it's no big deal, just try again next year or in a different plot in your garden. We, however, have only one world. While we can certainly try little things here and there, at the end of the day, we've got to be pretty sure we know what we're doing before we launch large scale efforts to shape our world one way or another. And that requires models sophisticated enough to let us trial things in simulation. So not only do we need a lot of knowledge about how our biosphere works, we also need to make sure it's the kind of knowledge that allow us to build simulations. Turns out, this kind of science is pretty new. 
\newline

Some history. The scientific revolution started in the 1500's and was set off by Copernicus' now famous idea that perhaps we aren't at the center of the universe \cite{sbrush}. What followed was a massive explosion in all things astronomy and over the course of a few hundred years everything we thought we knew about our place in the universe changed. In 1620 Robert Hooke discovered cells for the first time launching the whole field of microbiology \cite{wsd}. In 1735 Carl Linnaeus put together the system for classifying plants and animals that we use to this day and in 1745 Ewald Jürgen Georg von Kleist invented the first capacitor. Fast forwarding to the 1900's and we've got Max Planck's theory of black body radiation in 1900 followed quickly by Einstein's special relatively and his explanation of the photoelectric effect in 1915 - all theories that were seminal in the creation of modern physics. In 1953 Wilkins, Franklin, Watson, and Crick made the now famous discovery that DNA has a helical structure - thereby totally changing our understanding of the stuff - and in 1996 we cloned a sheep for the first time. So, given all of this progress, you might be surprised to learn that it was only in 1999 that Hans Schellnhuber published a "groundbreaking" paper on earth system modeling that spoke of a "second Copernican revolution" in our ability to model the biosphere \cite{hschellnhuber}. Our first thoughts on this issue came around the same time as our ability to clone things and well after we came up with the idea of black holes in space. 

To understand this we need to look at what was going on alongside each of these revolutions in science. The scientific revolution, with all of its progress in astronomy, came at the same time as the first telescopes were being invented. Hooke made his discoveries about the cell thanks to the newly invented compound microscope. Linnaeus only came up with his system of taxonomy after extensively traveling the world - something only possible thanks to all the nautical inventions that also created the "age of discovery". The first electrical inventions were all thanks to developments in what materials were becoming available as the industrial revolution unfolded \cite{tkuhn}. The discoveries of modern physics were all thanks to continuous improvements in our ability to poke at smaller and smaller things using tools like the x-ray diffraction techniques that allowed Wilkins, Franklin, Watson, and Crick to infer the structure of DNA. The pattern from all of these examples should be pretty clear - to understand our world we need the tools to probe it and each scientific revolution has been attended by updated capabilities in our ability to gather data \cite{tkuhn}. So it should come as no surprise then that one of the major tools used by earth system modelers - satellites - are a relatively recent development. As an example, the Landsat program, one of the largest programs to provide satellite imagery of Earth, only started in 1972 \cite{wls}.

But satellites are only part of the picture and the need for further instrumentation and tooling is clear. Whether it's land management \cite{jpongratz}, pollination services \cite{ibartomeus}, or general biodiversity management \cite{hkuhl} (just to name a few) scientists in recent years continue to point to a lack of comprehensive data as one of the primary challenges they face world over. Science is driven by data, and the data is sorely lacking. 
\newline

So if we want to spark another scientific revolution, one that will give us the tools we need to be good stewards of our planet, we need to start providing the instrumentation required. So question is - how do we do it?

\section{An Aside}
Before we dive into how to develop instrumentation I want to stress something. The last section argues that we \textit{need} models of our biosphere and all the attendant knowledge in order to be good stewards of our planet. I firmly believe this is true. However (and this is important) I \textit{do not} believe that such knowledge should block actions that are already supported by the science that exists \textit{today}. I \textit{do not} believe we should wait to decarbonize or wait to stop wasting as much as we do or wait to stop polluting. These are all good, clear actions that should be taken \textit{now} because the science clearly backs them. In general there will always be gaps in our knowledge that need filling, so if we ever want to be good stewards who actually take action we have to learn to listen to the knowledge we have. 

Alright, back to the scheduled programming. 

\section{An Industry}
We've got a problem. While pointing out the immediate need for biosphere instrumentation is cool and all, it's also laughably vague. How on earth are we supposed to execute on a mission statement that's that high level? Fact is, we need a strategy. 

Strategies start with understanding our goals and, especially in our case, clarifying goals is really important. The history of conservation and "biosphere activism" has... well... a checkered past. Early biologists exploring the pacific islands sometimes caused extinctions in "the name of science" and part of the reason we're having to reintroduce wolves all over the place is that early conservationists helped get rid of them in the name of creating a more bounteous United States \cite{mnijhuis}. At the other extreme, in my own experience, I've found that frustration with human negligence often leads to a desire to take humans out of the picture entirely, a philosophy that culminated in the Half-Earth Hypothesis - a proposal that we should set aside half of the planet as a human free zone (and they're not talking about totally uninhabitable places) \cite{ewilson}. 

But let's look at intentions. The folks causing extinctions in the pacific islands? They just wanted to study birds. Unfortunately, at the time, studying birds was synonymous with shooting them. So while studying birds is a great idea, shooting them all as a means to an end is a little more than short sighted. 

How about the wolves? Well quite a lot of the money in conservation coffers comes from hunters. So making our forests more bounteous isn't a half bad idea. The problems are twofold. First, it turns out wolves are needed to keep deer populations healthy (ecology!). Second, extirpation is a really extreme take on "reducing" wolf populations. So once again, reasonable intentions with a short sighted strategy.

Finally I get the frustration with people seemingly just bulldozing everything that isn't "producing human good" or treating the natural world with less than respect. But the problem here is one of education - rather than understanding that rainforests are useful natural resources, most economies see them as unrealized cattle farms instead. The problem isn't that humans are included, it's that those same humans aren't including the natural world in their calculus. 

Regardless of the various strategies taken, one thing is clear from all of these examples. The goal is to nurture the value that our biosphere provides us. Unfortunately this is still too vague for our purposes so let's dig deeper.
\newline

The Millennium Ecosystem Assessment \cite{mes} suggests a rather lovely framework for thinking about all of this. They suggest (and the suggestion is backed by data) that human well being exists on a continuous spectrum with poverty that is made up of five key components - the necessary material for a good life, health, good social relations, security, and freedom and choice. So while you may have a lot of material wealth (1) if your health (2) is extremely poor because of, say, pollution, you are still impoverished to some degree. In other words there is no such thing as poor or wealthy. Instead there is a five dimensional spectrum that we have to keep our eyes on at all times. 

As a quick example of why this is important, almost every assessment of ecosystems that I have read is stated in dollars. Even the first section of this essay had a dollar bent! But how do you put dollar values on social relations, feelings of security, or autonomy? You obviously can't! Having these five dimensions of human well-being helps us break out of the habit of only seeing value as economic. And this level of specificity and holism is exactly what we need in our goals.
\newline

Okay so that gives us a better sense of how to make sure we're measuring value completely, but how on earth are we supposed to connect this back to the literal jungle's worth of complexity in our biosphere? Trying to work out all the ways in which our biosphere provides for, say, our health brings the phrase - boiling the ocean - to mind. Clearly we need to divide the problem up into more manageable parts. Once again, the Millennium Ecosystem Assessment is here to help \cite{mas}. 

As I mentioned before they have already done the work of divvying up ecosystem services into nice neat bins. At the top level we have three overarching categories - provisioning services, regulating services, and cultural services - which are each then divided up into specific sub categories such as food and fiber, water regulation, pollination, sense of place, disease regulation, and so forth. These, obviously, are much more manageable pieces of the overall biosphere pie.
\newline

Alright, let's bring this all back. How does this help clarify our path to building biospheric instrumentation? Well first we know now that there are 25 specific models that we are trying to build - one for each ecosystem service category. Second we know that in order to be complete, those models need to be capable of helping us predict the effects on each of the 5 dimensions of human well-being. Finally we know that those models need to be comprehensive. So we can clarify our path forward as such - we need to help scientists obtain the data required to build comprehensive models that connect each of the 25 ecosystem service categories outlined by the Millennium Ecosystem Assessment to the 5 dimensions of human well-being. Which leaves a much simpler question - what data is required?
\newline

Here, we come to the kicker. Scientists have not exactly been twiddling their thumbs. Huge strides have been taken in learning how to think about and model this kind of stuff. Remember, the issue scientists have is not in having \textit{no} data, it's in not having \textit{enough} of it. Pull down the journal Ecological Informatics and you will find \textit{hundreds} of examples of instrumentation techniques that scientists have already developed and proven useful. They've already done the de-risking work and know what data they need - they just need help deploying the instrumentation at scale! And doing this, would be a real value add. 

Think about it this way. Most of us could, if we really wanted to, wire our own homes, do our own plumbing, and so forth. But doing so requires buying all sorts of tools, learning all sorts of techniques, going through all sorts of safety training, and so on. That's a lot of overhead for one house. A professional electrician also has to go through all of this overhead but then they apply their tools and training to loads of homes. This is the economy of scale - and it makes a big difference. We absolutely need this same economy of scale if we're going to attempt biosphere-wide instrumentation. And that means we need professional instrumentors. Put a slightly different way, to achieve the economy of scale, biosphere instrumentation can't just be limited to a research activity - it needs to be its own industry, driven by its own professional workforce. To trigger the scientific revolution we want, we need to build an entire industry.
\newline

\section{Conclusion}

If we're going to be sophisticated stewards of our planet, we need a scientific revolution in biosphere modeling. Scientific revolutions in the past were driven by revolutions in instrumentation. Today is no exception. Our strategy? Using the 5 dimensions of human well being and the 25 ecosystem service categories outlined by the Millennium Ecosystem Assessment as a guide, find the researchers already building the instrumentation we need and create professional industries dedicated to deploying and maintaining that instrumentation at scale. Let's create a biosphere instrumentation industry - let's create Network Earth.

\chapter{Why This Book?}
When I was a child, my family lived across the street from a wonderful little pond. Every time it rained our yard, the road, and, of course, the pond itself would become chock full of amphibians. I'd run out there in the rain and catch several of them to keep for a few a days in the bank of terrariums that my parents had purchased for my brother and I.
\newline

My brother and I just liked playing with them. We'd line em up and see which could jump the farthest, we'd build places for them to live with Legos, and before we let them go I'd take pictures of the patterns on their skin so I could check if we'd met before. Then, after their short stay was over, we'd bring 'em back to where we found them and let them go on their merry way.
\newline

I like to think that the stay with us was pleasant for them as much as for me, but I suspect that I got the lion's share of joy out of those interactions. Indeed catching critters was such a formative part of my life. I became obsessed with learning everything I could about the living world and had stacks upon stacks of National Geographic issues, natural history books, and field guides. It seemed obvious to everyone around me that I'd end up becoming a biologist of some kind. However fate had other things in mind.
\newline

Through a whole of series of events and existential crises that seem as bizarre to me now as they did when I was a hormonal teenager, at the age of 16 I found myself in a dormitory at Worcester Polytechnic Institute (WPI) to study physics of all things. But physics soon turned into math and by the end I'd given up on that too and was teaching myself software engineering and machine learning. Upon graduating with a math degree (I switched to computers too late) I took up a role as a "Member of the Technical Staff" at a telecommunications company that I'd interned with the prior summer.
\newline

Over the next couple years I ended up being relatively successful as an engineer and even ended up co-managing the team I was on for a while (something I was wholly unprepared for). However throughout that period I just felt like there was something missing in my life and so,after another existential crisis or two, I ended up wondering if it was time to return to my roots - wildlife biology and conservation. 
\newline

Thankfully I did the rational thing and decided to run a small experiment instead of just quitting my job wholesale - over the course of an academic year I took a marine biology course and a class in conservation. In the process of taking these classes I came to the realization that the kinds of technology I had been building over the past few years were almost completely absent from both biology and conservation. It was clear that those in the field could stand to take advantage of what I knew and so (at the advice of my professor) I started looking around for ways to volunteer my skills.
\newline

Over the next couple of years I volunteered with a bunch of different groups. At one point I helped a couple biologists automate a machine learning pipeline. At the end of that project it was clear that what we'd just built could be applied to so many other things, yet that project (for reasons I still wonder about) just sort of disappeared. I ended up working with a marine rehabilitation group and after a couple of months doing a lot of manual, analog stuff that definitely could get automated started floating the idea of modernizing some things. I was very quickly shut down and left a few months later. At another point I started contributing to an open source project that was dear to me but quickly found that they were so overwhelmed they could only really hand off  more trivial work to outsiders. Indeed in perusing their forums I quickly discovered that due to a lack of funding they had to shut down most of the more interesting features people using the tools wanted. All of this left a rather sour taste in my mouth. I couldn't get away from this feeling that I had so much to offer but no one seemed to have the time nor resources (and in some cases vision) to take advantage of it. Everyone was happy to take my money, or put my time to menial tasks, but no one really had a place for my \textit{actual} skills.
\newline

So, after a time, I decided to take things into my own hands. If others weren't going to put my passion and brain to use I would - and after months of reading, talking to people, writing, prototyping, and thinking I came up with the idea for Network Earth. But as I began working on that project something clicked in my brain. 
\newline

The whole idea behind network earth is to create industries around the kinds of technology that can help us instrument our biosphere. There are 25 different ecosystem services to monitor and loads of technologies to manage in each. I suddenly realized that Network Earth didn't just solve my own problem, but it could provide a direction for anyone else in my position - and goodness had I met a lot of those people. When working at the rehabilitation center I'd been surrounded by \textit{actual medical staff} who were being tasked with nothing more than doing laundry and mopping floors. In both of the classes I took nearly everyone there had loads of energy and ideas and each of them was looking for a way to help out. At work itself I was surrounded by colleagues anxious about the state of our world but unsure of how to become involved. So much skill, talent, passion, time, money, and energy with nowhere to go. 
\newline

And so I had an epiphany that I probably should have had ages ago (but as they say, better late than never) - the real problem to solve here is not the biosphere stewardship crisis but how to harness all the energy and skill of those who \textit{already want to do something about it}! If I could build the organization that I had spent the last few years looking for I could harness that energy and end up being a much more effective force of change than if I built one specific technology to solve one specific part of the problem. 
\newline

This, however, presented a problem. While I had some small amount of management experience I was totally unprepared to \textit{build an organization from scratch}! I clearly needed to learn organization development and quickly. Thankfully the stars happened to align in my favor.
\newline

Learning anything comes in two parts - theory and practice. Without the theory there's no way for you to "stand on the shoulders of giants" and learn from those who've come before you. But without practice it's all in one ear and out the other. Theory I knew I could do, I'd just simply read everything I could about organizational development and management until I felt confident there wasn't much left that would surprise me. But practice was going to be the harder thing to sort out. Happily, at around the same time as this epiphany I ended up getting something of a promotion at work and found myself co-leading a team of about a dozen people. And that team was tasked with building out a component of our broader organization that, up until that point, never really existed in any clear form. I found myself with a perfect mirror to the problem I wanted to solve - I had a group of brilliant people who's energy needed to be harnessed toward an organization that did not yet exist. Ample opportunity for practice had just landed in my lap!
\newline

Which brings us to why I started writing this book.
\newline

This book serves two purposes. First, I'm a firm believer that you don't really understand something until you can explain it to someone else. Therefore I am using the process of writing this book to firm up my own understanding of organization development. However I am also a firm believer in self organizing systems. To my mind the best systems are ones that can build themselves and so, if I do end up being successful in building organizations, part of that success is going to come through teaching others, within the organization, how to build organizations themselves. When that time comes I hope this book will be a central resource.
\newline

So, this book is nothing more than the compendium of the things I've learned about organizational development. And while a lot of it is going to feel philosophical or abstract in nature you can rest assured that I myself lean upon these principles each and every day I spend leading teams, building organizations, and dreaming of a better world. 

\chapter{How This Book is Organized}
To me building an organization can be summarized as such - you start with a dream and know you've succeeded when you've made that dream actionable and you're executing toward it. As a result, building an organization is all about resolving the specifics of that dream. Like a sculptor who starts with a block of material, chisels out the general shapes, and then begins working on the details and textures, organizational developers start with broad questions and then work our way down to questions about finer and finer details. 
\newline

This book then is an attempt to outline, as clearly and methodically as possible what those questions are, when to ask them, how to go about answering them, and what answers look like. However there are two dramatically different ways of presenting these questions that each serves a different, but very important, purpose.
\newline

The first is to build your intuition around why these questions exist, for what purpose, and why they take the form they do. This treatment is given in the first part of this book - \textbf{Intuition}. The second part of the book is far more practical. Theory is great and all but it can be hard to translate it directly into action to take. Instead, when it comes to actually putting things into practice, a manual is a far more useful tool. So the second part of this book - \textbf{Action} - attempts to play that role. As such it's not really meant to be read in linear sequence like a normal book, but instead to act as a practical reference for obtaining specific protocols to follow as you go about resolving your organization. However, without the theory it'll be terribly difficult to use the second section - hard to use a car manual to repair a car if you have no clue how a car works. 
\newline

So, when you're using this book, consider taking the following approach. Read through the \textbf{Intuition} entirely to ensure that you've got your bases covered and then, like you would for mending a car, use \textbf{Action} opportunistically, taking what you need, when you need it, and nothing more. 
\newline

Happy reading!


\part{Intuition}
\chapter{Creating Catalysts}
\section{Skill}
It's a reasonably well established fact that the average human being can only hold around 7 things in short term memory at once. Put another way, if I start sounding off digits to you in quick succession you'll only reliably be able to remember 7 of them at a time. However there are a rare handful of people out there who blow this number straight out of the water. There have been people who've been able to recite upward of 80 digits consistently and others who've been able to demonstrate this trick for 100's of digits at a time! Given the rather measly limitations on the average human's brain you may be wondering what makes them so special. The answer? Nothing at all.
\newline

For example take Steve Faloon - a student the psychologist Anders Ericsson recruited to study how far people can push their short term memory. When Steve first started he, like the rest of us, could only reliably recite 7 or 8 digits from memory. Yet by the end of this experiment he was reliably reciting 82 digits at a time! How had he made this jump? By using mnemonics. Being a runner, Steve realized that he could group series of numbers into running times. So if he saw the digits 1, 0, 3, 5 he'd remember the running time 10 minutes and 35 seconds. This alone allowed him to focus on remembering 7 running times instead of 7 digits which translates into around 28 digits overall. A huge leap forward! By continuing to create structures like these that made better use of his normal short term memory, he was effectively able to extend his short term memory by over 10 times. 
\newline

Another feat of intellect that seems near miraculous to the rest of us is the ability of master chess players to play blindfolded or without a board. Indeed many master chess players have taken this to the extreme with some playing a couple dozen games simultaneously all while unable to see or touch any of the boards! Wondering what sets them apart, a set of psychologists performed a rather interesting experiment. They took a national-level, mid-level, and novice chess player and showed them a board with a whole load of chess pieces arranged in a way that might arise from a game of chess. Then with the board removed they asked them to recall the arrangement of the pieces. Unsurprisingly, the better the person was at chess the better they did at remembering the pieces and where they were. But then they showed those same people a chess board with \textit{randomly} arranged pieces, took the board away, and asked them to recall again. Surprisingly in this case there was no longer any difference in performance! The national-level chess player no longer had any advantage. 
\newline

In interviews with a variety of experienced chess players, it has become clear that what they are seeing in their head when they play chess with no board are "chunks" of interacting pieces. They'll see setups that they are familiar with from prior games, lines of attack or pressure that have been important in the past, etc. And so when you randomly arrange the pieces on the board, none of these chunks can be found and so the ability to remember it with any level of real skill goes away - once again the exceptionalism is coming from specific mental representations that they've built up. 
\newline 

Ericsson and others studying exceptionalism have found this pattern over and over again pretty much everywhere they've looked. From swimmers being taught ways of perceiving their body and performance as they swim to musicians learning music theory to physicists learning the math they'll need to represent the universe - mental representations seem to be the key to gaining these incredible levels of skill. As Steve was finding ways to better take advantage of his limited short term memory so too are masters in all of these areas finding solutions that allow them to break through to the next level of extraordinary. 
\newline

However, all of this should be setting alarm bells off in your head. If it's just a matter of learning mental representations, why on earth aren't we all master chess players and elite swimmers? Clearly we're missing something. To see what it is consider the pipe organ.
\newline

Playing the pipe organ (the kind you find in an old church) is no easy feat. There are usually multiple keyboards which you'll end up playing separately with each of your hands, a pedal board for playing notes with your feet, dozens of stops that control what \textit{kind} of sound each keyboard produces, volume levers, and then of course there's the matter of actually playing itself - reading music, coordinating your fingers, keeping tempo, knowing how to interpret the music, and so on. Clearly if you try to actively \textit{think} about all of these things all at once you'll get overwhelmed. Instead you've got to make each of these components instinctive so that you can use your conscious mind to work on coordinating it all rather than trying to work out all the little details in real time. In other words you've got to work the underlying skills into your intuition.
\newline 

Everyone who's practiced something knows that when you first start, whatever you're doing is \textit{hard}. However if you keep up consistent and regular practice over time, what was hard ends up becoming relatively easy. Why is that? Well it has to do with the notion of homeostasis. All your body wants is for normal life to be... well... easy. So when it suddenly finds that every day you, for some unfathomable reason, find yourself running, it figures out that if this is gonna keep up it had better reconfigure things in order to make the run less of a pain. So it starts building blood vessels and musculature to turn your daily run into just another part of the norm. It shifts your statis state - your homeostasis. 
\newline

Turns out the same thing happens for these mental representations we've been talking about. When you first start using them, your brain is rather clunky about it. Like the first time you did algebra or started reading you have to think really, really hard to get things to work. However once your brain realized this was part of the norm it started shifting the mental representation from your conscious thought into your subconscious thought (which is far more effective and powerful) so that reading and algebra became relatively easy. However this takes time because you first have to establish that this mental representation is a new norm \textit{and then} give your body the time to rewire properly. Going back to our pipe organ example we see that there are potentially dozens of individual skills that have to go through this process before we're able to play the pipe organ with any real skill. So take whatever time is required to embed one skill and then multiply it by the number of skills one will need to embed and you've suddenly got a sense of why we aren't all masters of every field out there - layering the skills required to master something takes a hell of a lot of time!
\newline

We can actually get a general sense of the order of magnitude involved in \textit{just} the embedding process by looking at a skill that has an extremely well developed learning sequence by this point in history - classical violin performance. Given it's skill that's been around for at least a few hundred years, people have been able to take all the guess work and creative mental representation building that Steve was doing out of it. Therefore learning to play the violin is really just a process of embed, embed, embed. 
\newline

Ericsson and two collaborators Ralf Krampe and Clemens Tesch-Romer went to a well known musical academy and selected three groups of violinists - those who were the "best", those who were "better", and those who were "good". They then tried to search for what differentiated these groups from one another. Unsurprisingly, given all we've learned so far, the only predictive component they found was the amount of time people spent practicing - if you practice more you're better at playing the violin. However what's incredible is the sheer amount of time spent practicing. For those in the "best" category, given how much they'd been practicing, it was predicting that by the time they were going to be in their thirties (when folks start really rolling in their careers) they'd have put, on average, somewhere near 25,000 hours of practice in! To put that in perspective, even if you assume they've been practicing since they were 5 years old, that means they've been treating practicing as a part time job (20 hours a week) for their entire life! Embedding skills take a lot of time, energy, and dedication indeed.
\newline

This then brings us to Ericsson's conclusion - what leads to skill in any field is simply \textit{deliberate practice}. He, of course, does a much better job of getting at the specifics (and I'd recommend reading his book \textit{Peak} \cite{ericsson} of which this section is a mere summary) but it more or less comes down to the idea that becoming skilled at something is about two things - (1) not reinventing the wheel, i.e. learning the mental representations that others have already invented in an order that they prescribe and (2) pushing your definition of normal deliberately and consistently in order to truly embed the skills you are learning into your body and brain. However if this is true than it also predicts something that would be rather liberating - that anyone can learn any skill. Note that nowhere in deliberate practice is there any notion of what we might call "talent", all there is is hard work and paying close attention to the coaching others can give you. So is he right, can anyone learn anything? The proof is in the pudding, so let's look at a positively wild example.
\newline

Perfect pitch is something that people will often treat more like a sense than a skill. Most assume that you've either got it or you don't. However in 2014 a Japanese psychologist decided to test this assumption. He gathered together a set of children between the ages of 2 and 6 who didn't have perfect pitch, and over the course of several months taught them how to identify 14 chords by just listening to them. For some children it came more easily than for others, but eventually everyone could identify the 14 chords on their own. Then he tested the kids for perfect pitch. What he found was truly astonishing - every single one of the children had developed perfect pitch and could identify individual notes from sound alone! Something that had always just been considered talent was 100\% learn-able!
\newline

The conclusion is clear (and if you want more evidence, once again, check out Ericsson's book \textit{Peak} \cite{ericsson}) - anyone can follow in the footsteps of the masters if they're willing to put in the time and use deliberate practice. However this leaves us with a question - what happens when there are no longer any footsteps to follow? What if like Steven Faloon you're finding yourself having to invent the solutions that will bring you to the next level? What about creativity? Clearly if we want to under exceptionalism overall we will need to understand the creative aspect as well. So let's turn to that next. 

\section{Creativity}
When we think about creativity we think about novelty - something totally new, a paradigm shift. Strokes of inspiration straight from the well of creative genius. As such, it can end up feeling kind of haphazard, as if you must just mull over things until that stroke of inspiration occurs. Indeed it can end up feeling as if creativity is something given in droves only to a select few. Reinforcing this sense is the fact that most narratives about creativity are almost always dominated by a select few - cubism by Picasso and Braque, computing by Turing, relativity by Einstein. However, if we take a closer look at the achievements of these creative geniuses, patterns begin to emerge. 
\newline

As an example, take Picasso. It's often noted that Cubism was inspired by the geometric and abstract nature of traditional African art, especially masks - something that Picasso took a keen interest in \cite{sabine}. However, Picasso was only able to bring this style into the fold of more mainstream painting because he was one of the most technically proficient painters at the time. In similar fashion Jazz, one of the more influential musical developments in history, arose in New Orleans out of a melding of predominantly white, brass, marching bands and the music and dance traditions of recently emancipated African Americans \cite{jazz}.
\newline

On the STEM side there are examples like this as well. Probability and statistics started out as simple accounting for the purposes of playing cards and dice, but has subsequently found applications in pretty much every field in existence. Signals processing is only possible thanks to a kind of mathematics called Fourier Analysis that is more or less the combination of the epicycles that early astronomers used to explain the cosmos and the principles of calculus that Newton developed \cite{gnu}. 
\newline

Then there's all the inspiration taken from nature. One of the most successful optimization strategies out there - genetic algorithms - came from the flash of insight that if you can create a genome for different solutions you can find the best solution using the same principles backing genetics and macro-evolution in biology. Then there's the fact that down jackets are the result of realizing that some birds don't fly south during the winter, modern windmill fan-blades were inspired by humpback whale fins, and there people are studying ants to figure out how to regulate traffic with autonomous vehicles \cite{biomimicry}.
\newline

All in all, the pattern here should be pretty clear - strokes of insight seem to come from taking the ideas from one field and then finding interesting ways of applying them to another. African masks + european painting becomes Cubism. Tricks from ancient astronomy + calculus becomes Fourier analysis. Bird watching + a need to engineer better clothes becomes down jackets. Creativity is the art of synthesis - finding useful and/or interesting connections between previously unconnected things.
\newline 

Now at this point you're probably feeling like this is a gross oversimplification - and you'd be right! For starters, before the insight can happen the tinder for the spark has to be present. There were plenty of artists around in Picasso's day but he was set apart both due to his exceptional technical fluency and his interest in African art. Then there's the fact that the stroke of insight doesn't come of its own accord - clearly there would've been many people aware of genetics, evolution, and the need for better optimizations - yet, in the end, only a handful of people came up with the idea for genetic algorithms. Then there's the process of sorting - many creative people note the fact that for every good idea you have there will be loads of poor ones, and distinguishing one from the other is no trivial matter \cite{kaufman}. Finally, it's a long way from that initial spark to real achievement - Fourier spent a long time formalizing and proving out his ideas before anyone took them seriously, and you can be sure that folks didn't just copy a whale's fin and stick it directly on a windmill turbine. Beyond the initial stroke of insight there are obviously other very important steps involved. Gather, Spark, Sort, Develop in no particular order (because creativity is a messy business) - these are the steps to creating something new. 
\newline

However, for each of these steps, we can learn from those who've already achieved. We can note that for the cubists and the impressionists (who took key inspiration from Japanese prints \cite{impressionism}) their gathering stage was marked by exploring farther afield than most artists in their day. Einstein is often noted to have spent a lot of time performing thought experiments in his head in order to turn problems this way and thereby facilitate sparks of insight and inspiration. Isaac Asimov (a prolific science fiction author) noted the fact that sometimes it is best to be alone when evaluating and sorting through ideas because many of the initial ideas you'll have will be, in his words, "embarrassing" \cite{kaufman}. But these are just a few examples. In general, there are entire books written on this stuff. From Kaufman and Gregoire's \textit{Wired to Create} \cite{kaufman} to Csikszentmihalyi's \textit{Creativity} \cite{creativity} there are loads of books out there chock full of lessons learned from studying what sets high achieving creative people apart.
\newline

In the light of the last section, all of this should be sounding tremendously familiar. We've got a specific goal - synthesis of previously disparate fields - as well as a whole lot of techniques, advice, and, may I say, mental representations from experts. In other words, while what comes out of the creative process is certainly novel, the process of being creative is not novel at all! Synthesis is a skill, with experts, that can be mastered itself! It's just a very special skill in the fact that regardless of the field you find yourself in, you're going to want to master this one as well because creativity is how you adapt and push boundaries. But as we've seen in the last section, mastering a skill is no mystery rooted in talent alone - it is instead the result of deliberate practice. 

\section{The Need for Organizations}
Alright, at this point we know that the key to exceptional skill and creativity is deliberate practice. However, we also know that deliberate practice for even one skill takes an extraordinary amount of time and, as we just pointed out, one must invest time in practicing creativity too. Remember those violinists who'd spent 20 hours a week for decades to master their craft? That was just for practicing the standardized components. All in all, exceptionalism comes at a very specific cost - and that cost is specialization. Each individual has only so much time and given the massive investment required to master something, there will only be a handful of things you can master in one lifetime.
\newline

However, as anyone knows, specialists cannot operate alone. People who operate in sales need things to sell. Musicians need bands or orchestras and places for those bands or orchestras to perform. Mathematicians only really become "useful" when engineers and scientists put their abstractions to use. A doctor is no good without the hospital staff and admin required to keep hospitals running. In short, specialists only become valuable in the coordinated presence of other specialists.
\newline

As a result, I, as an individual, will only be able to see benefits from my specialization if this coordination exists. And so we can see that creating high performers through deliberate practice is not enough. While it is definitely a necessary condition for taking full advantage of human potential, it is certainly not sufficient. What you also need are \textit{organizations that can coordinate those specialists}. Put another way, without organizations, specialists are nigh useless. But this also means that organizations that do a poor job developing and/or coordinating those specialists are also useless. 
\newline 

I like the analogy of catalysts here. Within your cell float about loads of proteins, carbohydrates, salts, sugars, and all sorts of other useful, specialized, substances. However, on their own these chemicals just form a weird, gooey soup because each of these substances on their own don't really do anything. What turns that soup into life are the catalysts that facilitate bringing the various substances together in useful and meaningful ways. This is what transforms a biochemical sludge into one of the most incredible machines in existence - the living cell. Organizations, then, are catalysts for human specialists. Without them there is no way to take full advantage of the human condition and as a consequence no way to tackle and solve our biggest problems or our most daring dreams. And so, if you wish to tackle something huge, if you to achieve in the highest order possible, you must do two things - first you must engender the most exceptional of specialists and then, with equal importance, you must build the organizations that will catalyze them. For without those organizations there will be no mechanism capable of mustering the human skill and energy required to shift a component of our world.

\nocite{duckworth} \nocite{outliers}

\chapter{The Value of Focus}
The world is a really big place with a lot going on. No matter where you look, opportunity and problems abound. On just the matter of sustainability the United Nations has identified 17 separate goals each of which is enormous in scope - number 15 is just titled "Life on Land" \cite{united}. Likewise the Millenium Ecosystem Assessment has identified 25 different ecosystem services, all with their own host of concerns \cite{mas}. And all you need to do is head over to your news feed of choice to get an overwhelming sense of all there is out there - economic recessions, public safety, the fate of democracy, crises in agriculture, issues with education... the list goes on and on.
\newline

Then there's the fact that the "Devil's in the Details". Take literally any of the above topics and you can subdivide it about a million times over to create thousands of different focuses, each with their own host of questions, issues, and jobs to get done. The world is so interesting and diverse that people can make a career out of studying a very specific people, at a very specific point in time, in an equally specific part of the world! 
\newline

And then there's the fact that each of us is juggling so much in our day to day lives. There's school, families, friends, work, hobbies, commutes, cooking, cleaning, exercise... yet another overwhelming list. 
\newline

All of this is enough to make one feel like there's not enough time to go around! So many world issues that it seems like one should care about, so many things to juggle in daily life, and then even just getting through the details of any one of these things can feel overwhelming. And then, invariably, there are those people who just seem to be doing it all without breaking a sweat. You know the ones I'm talking about - the ones who are part of the local government, while also holding a full time job, while also being involved with their local church, while going out every night with their friends, while also having kids, while also... you get the idea. It leaves a person wondering what their secret is - how are they able to get everything out of life while the rest of seem to struggle with a far smaller TODO list?
\newline

Well, Clifford Nass - a professor at Stanford - wondered the same thing. So he brought together a group of researchers and a group of students to find out. He divided the students into high multi-taskers and low multi-taskers and then set about searching for what they specifically excelled at. But what he found surprised him - the high multi taskers just did worse at everything... Since then, lots of other research has followed this up and shown the same thing - there is no such thing as effective multitasking, the key to success is actually focus. \cite{keller}
\newline

To understand why this is requires that we look at several attributes of how our brains work (and don't work). Each of these examples on their own will make a great deal of obvious sense, but together will lead us to some pretty interesting conclusions about how to be effective as an individual. So let's dive in and start with the idea of attention residue.

\section{Context Switching}

We've all had that feeling - the one where you were just playing a game or watching a show and suddenly you're asked to direct your attention elsewhere, like toward homework or something like that. It's a nasty feeling and leaves one feeling frazzled and not totally focused. Well this is more than just a feeling. Sophie Leroy ran a study in which she gave a series of people some difficult puzzles to solve. Then, for some of them she would interrupt them and give them a new task. For for the others, she let them finish first. For those interrupted they performed worse - indicating that context switching in the middle of something isn't some kind of flip of a switch. Leroy named this phenomena - \textit{attention residue}. 
\newline

This attention residue can last for a really long time. Thanks to Bluma Zeigarnki's research we know that unfinished tasks at the end of the day can stick with you and continue interrupting your thoughts throughout the rest of your evening! This phenomena of leftover "background threads" is called the Zeigarnik effect \cite{newport} and getting rid of it requires taking the time to get your mind in a place where it feels like today's tasks are indeed done. Often times this involves planning out tomorrow in such a way that the tasks unfinished today have guaranteed space tomorrow. And as you might expect this takes time to do. 
\newline

But there's not just overhead involved in getting the last task out of your head, there's also overhead in bringing your faculties online for the next one. For anyone who's brainstormed or experienced writer's block, you know that it's really only once you get going that the going gets good - i.e. it takes time to warm up your brain to whatever you are doing. And getting to that warmed up state of mind can take quite a while. So, not only will you be taking time to get your mind off the last thing, but also taking to time to really get into the groove on the new one. 
\newline

And let's be real, context switching is not fun. Having that distracted feeling, experiencing writers block, or feeling like you're really just not in the groove is not at all enjoyable. So both for efficacy and enjoyment's sake reducing overhead is the way to go. 
\newline

This, of course, means making sure you've got large blocks of time to do things that are challenging. For example, suppose you have a 30 minute window open at work and you're tempted to try and get something done during that time. You're going to be spending several minutes pulling your brain of the last task, several minutes getting into the flow for the upcoming task, only to have a few minutes left to actually do anything well. Which is going to leave even more open threads in your brain to distract you during your day! Definitely not worth doing. 
\newline

Also note another conclusion that falls right out of this - multi tasking isn't real! We only ever really do one thing at a time, and every time we switch tasks we pay a heavy penalty in context switching. So we've got our first principle then - context switching takes a considerable amount of time and effort and represents a kind of overhead that should be minimized by instead working serially in large dedicated blocks of time.
\newline

To get at our next set of principles, we'll need to switch gears and consider the nature of willpower.

\section{Willpower}
There are quite a number of studies that demonstrate, rather nicely, that willpower is not only a resource used to redirect ourselves away from our default behavior, but that it is a limited resource at that. For example, in one study students were asked to memorize either a two or seven digit number after which they were presented with a choice to either have chocolate cake or fruit salad as a snack. The minimal increase in cognitive load required memorize a seven digit number instead of a two digit one resulted in twice as many people taking the cake \cite{keller}. 
\newline

In another study, each participant started in a room with the researchers, a bowl of radishes, and a bowl of cookies. The researchers pretended like this was a study on taste and so instructed the participants to eat only a specific food while the researchers went off to do something else - for some it was cookies, and for others it was radishes. More than one radish assignee nearly ate a cookie! Then the researchers said they needed to pass some time to let the "sensory memory of the food fade" and gave each participant a puzzle to solve. However they neglected to tell the participants that the puzzle was actually impossible. At this point the researchers left and watched the students from behind a two way mirror. The cookie eaters worked for about 19 minutes before giving up. The radish eaters on the other hand only averaged 8 \cite{duhigg}.
\newline

Likewise upon studying over a thousand parole board hearings, a clear pattern was found between the proportion of favorable decisions and how long ago the judges had taken a break. The default behavior for such judgment is to give a negative decision and the number of these steadily increased until break time after which the number of positive decisions suddenly jumped through the roof \cite{keller}! 
\newline 

The point of this is clear - willpower is a limited resource that eventually runs out and is therefore something we have to be careful to use wisely. So, two questions immediately - are there ways to increase the availability of this resource and how do we use it wisely?

\subsection{Using Willpower Effectively}
The answer to the latter question comes in two parts and both require a clear understanding of what willpower is - doing something that isn't your default behavior. As Charles Duhigg expands upon in great depth in his book "The Power of Habit" \cite{duhigg}, our default behaviors are driven by habits. To understand this we must dive into what a habit is. To illustrate this Duhigg presents a very simple experiment that was done on rats. Specifically, rats were placed in a kind of "maze" that had chocolate at one end for the happy rat that could find it. When the rats were first introduced to this they would wander around largely at random and their mental activity would be quite high in the new environment. Eventually they would end up finding the chocolate and the researchers would repeat the whole thing. What was very interesting however was what happened as time went on. Eventually the rats would master the maze and know exactly where to go and during this process of learning their brain activity went from very active to barely active at all and actually changed position in their brain - they had internalized the pattern and could now do it without much thinking. They'd moved the activity from the active decision making part of their brain to the default circuits in their brain. 
\newline

However there was something another strange insight that came along with this discovery. While the brain activity during the maze walk decreased, the activity at the beginning and end did not. Specifically it seemed that the brain was looking for cues to start a default sequence, then fell into it, and then ramped again up again once the task was complete. This Cue, Routine, Reward loop turns out to be how habits always work \cite{duhigg} and is a model for how to embed things into your default behavior. 
\newline

Two things come from this. The first is that the best way to use our willpower is to, instead of constantly fighting our default behaviors  create habits in order to change our defaults to suit our needs. Indeed this has led to the observation by several people that no one is \textit{really} disciplined, there are just those that exercise short bursts of discipline to change their default behavior in this way and those who try to fight their default behavior and eventually fall off the wagon. So when you see someone studying or practicing for hours on end, it's not because they are actively exerting willpower - it's that they've built habits that allow them to get into flow during those activities \cite{keller}. 
\newline

The second conclusion from this is that life has to be built around processes, not discreet tasks or goals. Habits are all about routines and therefore in order to build them you have to turn your life into a series of processes that you can repeat day after day in a largely consistent manner. This is quite a divergence from the kinds of chaos that life can regularly throw at you.
\newline

Which brings us to the second point on how to use willpower effectively. As Cal Newport exposes in his book "Deep Work" \cite{newport}, distractions are the bane of productivity. Even if you're trying to establish flow, having your phone buzzing every two moments, or having coworkers popping over to talk to you at random, or having to constantly check email, leads to distractions that very quickly drain whatever willpower you have. So as just a common sense takeaway, it seems reasonable to divide your time into deep and shallow work so that you can group the distractions together and prevent them from draining your willpower. You can let your default circuit run rampant around messaging and communication at a different time from your deep work - during those deep work times just create your own personal "work bunker" \cite{newport} in order to protect yourself from distractions.
\newline

In summary then, to use our willpower effectively we must organize our days and habits to create alignment between our default behaviors and what we want to get out of work while creating clear distinctions between deep and shallow work. In so doing we prevent our willpower from being drained unnecessarily. 

\subsection{How to Get More}
Willpower often gets compared to a muscle - use to much and it gets tired. But the analogy goes further because like a muscle will power can be trained. Two researchers, Megan Oaten and Ken Cheng, did a series of experiments in which they got people to exercise their willpower in one arena (like going to the gym, or doing better money management, etc) and then showed that it created spontaneous changes in other parts of their lives (like eating better, smoking less, or becoming more productive at work) \cite{duhigg}. Exercise your willpower muscles and you strengthen them. But the reverse is also true - give into distractions frequently and your willpower wanes. Cliff Nass's research has shown that the brains of those who are chronic multitaskers or who are chronically distracted just remain overstimulated all the time and have trouble focusing even when they want to \cite{newport}. In order to keep our willpower muscle healthy we have to keep exercising it. So it's important to set aside time to focus deeply and practice using willpower so that you can push your limits and stay strong.
\newline

Another useful note is that like a muscle, giving your willpower rest gives it time to bounce back a little. In one study two groups were asked to go on a walk before doing a task requiring a lot of concentration. For one group, they went for a walk in the woods, for the other they walked through a busy city center where they had to dodge all manner of people, cars, and the like. The group that went on a lovely walk in the woods did significantly better on the task than those who'd been busy trying to make their way through the city center. The lead author Marc Berdman attributes this to the fact that while walking through the woods we no longer direct our attention, whereas in the city center we have to be on guard and so are sapping our willpower as we go about. Taking time to be idle gives us an opportunity to replenish a little bit \cite{newport}.
\newline

Now one note we should make before coming to any conclusions about "unlimited power" is that like any other muscle there's not only diminishing returns to any of these techniques but there are also limits. For example, in his studies on expert performers Anders Ericsson rarely found people who's time spent truly focused on difficult \textit{new} tasks ever exceeded 4 hours \cite{newport}. So while there are certainly ways to extend the time you can spend draining willpower, be sensitive to your limits. Work beyond that time will, by definition, not be the best you can do.

\subsection{When the Tap Runs Out}
So what to do when your willpower tap runs out and you find yourself struggling to focus or realize you are operating suboptimally due to mental fatigue? Embracing idleness can be extremely powerful. People have run studies that show that sleep (or even naps) can allow your subconscious to solve problems that you'd been struggling with all day \cite{kaufman} or that going for a walk or using some other way to get your mind off things will suddenly give you that eureka moment that you've been looking for \cite{newport}. In general studies have found that your subconscious faculties are very powerful on their own and often times better suited to certain kinds of problem solving. So taking a break doesn't actually mean you've stopped thinking or being creative - it just means you've let a different part of your mind come into play. And from this vantage point it's just plain dumb to burn yourself out working suboptimally when a nice rest and good night's sleep would mean you'll have meaningful breakthroughs you can follow up on with your full faculties the next day.
\newline

When the tap runs out, and perhaps even before then, \textit{rest}.

\subsection{Use Willpower Wisely}
All in all then, we've learned a couple of things about using willpower wisely:
\begin{enumerate}
\item Divide your deep and shallow work intentionally, no point draining your willpower unnecessarily
\item Use willpower to create habits not to fight yourself day in and day out
\item Exercise your willpower to keep it strong (i.e. work in large blocks of focused concentration)
\item Take breaks when you feel yourself getting tired 
\item Embrace idleness and don't overwork yourself
\end{enumerate}

One final note is that willpower is not just for work. It's for anything you really deeply care about and want to give your all. So friends, family, hobbies - they will all need willpower too. Which points out something big - even if you had 10 hours of deep work in you everday, with habits and the benefits of practice involved, you only really have 5 or so hours to dedicate to work each day. And as anyone who's worked on tough problems knows - five hours quickly seems like very little. Unless that is, you have \textit{focus}. If you can dedicate that attention to one or two things, fine. Anything more than that and you'll just perpetually feel like you aren't doing enough. So, once again we come to the conclusion that being effective requires focus. Multitasking just spreads you thin.
\newline

But enough of that, onto the growth mindset!

\section{The Growth Mindset}
We often make decisions on the basis of what's happening in there here and now which... makes sense. The here and now is what we can feel and the future is not a given, so this, current moment is certainly of far greater importance than any single moment to come. The problem in this calculus is that we often forget that while each coming moment may have less valuable than the present, there are \textit{way} more future moments. So thinking about how to build upon all of that time-to-come is one of the best places to find ways of being ever more effective.
\newline

The first and most obvious of these "growth mindset" strategies is education. Everything you can do today is thanks to what you learned yesterday, and so what you learn today will make you better tomorrow. If, once out of school, you just stopped learning and worked as hard as you could you'll seem extremely productive in the short term, but in the long term you'll get outpaced by the version of yourself that decided to dedicate time to education as well. However the reverse is also true, a version of yourself that never ever starts working but just learns gets even less far. So, a balance is necessary. Continuing education is one of the best ways to ensure you just get more and more effective over the long run. 
\newline

Then there is what Keller refers to as the "domino effect" \cite{keller}. I find the best way to illustrate this (given what a long play it is) is to refer to one of my favorite kinds of game - simulated city/civilization building. In these games you normally start out with very, very few resources and people. For example, in one of my favorites as a child you start out with a few huts on the Nile river. At this point in the game your options are quite limited - instead of building pyramids, you're organizing farmland, building new huts, and getting cleaner water sources. But as your people get healthier and their work more efficient, a surplus appears and you can start to use that to create healers, roads, and wagons. This in turn provides for even more surplus which allows you to take even larger actions, which creates more surplus, which generates more action, which... you get the idea. Before you know it you've started construction on your first pyramid. This kind of exponential growth comes when each new increment of work we do both takes advantage of what we've already done, and sets us up to be more efficient in the future. We get buoyed by our own past doings! The reason why Keller refers to this as the domino effect is because a falling domino can knock down another domino that's a fixed multiple of its size. So for example if that multiple is 1.5x then if your first domino is an inch tall, the next domino can be an inch and a half, the one after that 2 and a quarter inches, then 3 and a third, then 5 inches and by the time we get to the tenth domino it is roughly 5 feet tall! Compound growth is an exceptionally powerful thing but it requires that you are in fact compounding something on itself. If instead you just redirect your energy in a different direction from one moment to the next, no compounding is going to happen.
\newline

Now when it comes to getting that compound interest to have a quicker return it's wise to remember the Pareto principle (also known as the 80-20 rule). The idea here is that for any effect, 80\% of it is likely the result of only 20\% of the underlying causes. The classic example of this (and the one that led to Pareto coming up with this principle) is wealth. A small fraction of people control the majority of the world's wealth. In Pareto's day it was roughly 80-20. But as it turns out this rule applies pretty much everywhere. I've come across it while teaching people, building software, or even while trying to get a recipe to work just right - a minimal amount of causes creates the most impact. Another way of looking at this is in terms of diminishing returns. Everyone knows that the more you try to optimize something, the harder and harder it is to make it better. Learning how to go from making a sauce so awful that no one wants to eat it to something palatable - relatively easy. Making a sauce that just comes out above the best in the world? Really, really hard. Minority of causes, majority of results. 
\newline

This principle is very useful because it also means that if we take the time to figure out what that hypothetical 20\% is, we can save ourselves a lot of time - we can get to knocking over that next domino that much faster. If instead we don't take the time, chances are we'll end up dipping quite deeply in the 80\% of causes that create only 20\% of the impact - which definitely doesn't help us work more efficiently. Here then is our third and final principle of this group - take the time to figure out the "keystone" causes of whatever it is you're doing so that you can spend less time working and get just as much result.
\newline

So what's in common between education, the domino effect, and using Pareto's principle to our advantage? It's the fact that if all you cared about was the present, you'd never invest time in any one of these! Education takes away time from high priority work that you might be stressing about. The domino effect requires that you stay focused on building on what you've already done even when your boss turns around and tells you there's something higher priority but relatively unrelated that needs to get dealt with urgently. And to find that golden 20\% requires once again taking time away from work to deeply understand what it is you're doing and what its keystone causes are. For those who apply those principles they will \textit{always} look less productive than their peers in the moment. 
\newline

However, over time they get to reap the rewards of compound interest! And this means that their base level of productivity grows exponentially. So while they may look unproductive in comparison to their present-focused peers in one moment, a few years down the line the two won't even be comparable. \textit{Investing} your time is the most important thing you can do to make yourself more productive - everything else we've talked about pales in comparison because this stuff \textit{raises the baseline}. 
\newline

However, as we mentioned before, compound interest only works if you are actually compounding on the same thing. All of these strategies require that you have focus, and focus that you can maintain over \textit{years}. So far we've noted the importance of depth and focus in work, but here we're talking about depth and focus that remains consistent over long periods of time. Yet, without finding that focus we'll never be able to take conscious, deliberate advantage of compound interest on our own capabilities.

\section{People, then Problems}
The first thing that should be clear from everything we've said thus far, is that multitasking, for the sake of multitasking, is not a virtue. It creates more overhead, it reduces our willpower, prevents us from educating ourselves as deeply or consistently, interrupts the compounding of our own work, and in general just takes the potential we have and smears it out so that we become a kind of "master of none" \cite{keller}. Focus is good, and for making real progress, focus is necessary. 
\newline

However when we look at the fact that the whole point of our "compound interest" is to take us beyond what we can perceive or imagine today, it also becomes clear that our focus is not just about the goals of today - instead it has to encompass our growth over many years to come. How on earth can we find focus that will extend past what we know today? Well, imagine once again that we're playing the civilization building game. At no point are our goals fixed, instead what we are doing each and every moment is monitoring what we have, what's been working well, and how we can take advantage of it to launch ourselves forward. In other words we're not so much focusing on a specific outcome but on understanding what is driving our success today and how we can continue to grow \textit{that}. 
\newline

Similarly when it comes to our considerations of willpower we found that the key is to use our discipline not to fight our default behaviors but to rather use that discipline to engineer and redirect our default behaviors through the formation of habits. But as we've seen, habits are rituals, processes, ways of life. And so once again we see a need to look for common, repeatable patterns in our lives that we can turn into habits. 
\newline

In other words as we go about our work, the key to our long term success is in finding this unifying framework around which we can build habits and upon which we can keep compounding our efforts. We are not just a bag of skills but a moving, developing centroid of processes, skills, resources, connections, capabilities, ideas, and perspectives that have been honed over time.
\newline

Why does this matter? In our wildly varied and oftentimes overwhelming world we typically take the problems or opportunities out there and try to figure out how to fit ourselves to them. You see this all over the place. Charitable operations ask "what can you do?". Your work hands you jobs and expects you to be resourceful enough to figure out how to get it done. We ourselves orient around far off dreams that more often than not have very little to do with who we are today. We see the world as a collection of aspirations, and success as our ability to conform to those things. 
\newline

However now that we know what creates success in the individual, it should be clear that this whole mentality should be flipped upon its head. Success comes not from launching ourselves into whatever problem is at hand but instead in looking for the things that we are best equipped to do, and especially those things that will allow us to build upon and develop ourselves even further. It's not about finding the most important problem you but finding the problem you can be most important to. And to be important to that problem you must prevent yourself from being diluted by the world in the process. 
\newline

So, to bring this back to the beginning - yes, the world is overwhelming and often times frightening, but it is not your responsibility to be involved with it all. The best you can do with the time and energy you've got is to follow the part of the adventure well suited to you and devote your heart and soul to it. 
\newline

The problem now is, if rather than being able to specify what needs doing and driving people towards those tasks, each person is just taking up the things best suited to them, how do we ensure the work that needs doing still gets done? This, is a challenge to organizational development theory. 

\chapter{Cheap Failure}
In the summer prior to my senior year at college I got an internship with a satellite communications company where I, and two fellow interns, attempted to build some machine learning models. I ended up getting hired full time and spent the next two and half years working with that group. They were the best engineers I'd ever worked with and I still attribute a lot of my personal success to what I learned and absorbed being around all of them. However I eventually ended up leaving for greener pastures specifically because the project we were working on just never seemed to really go anywhere. Indeed shortly after my departure the group re-branded, switched tactics, and more or less left the work we were doing behind. 
\newline

I think during college (and even before that) I had this notion that if you had the skills and the time to use them - success was just around the corner. Yet here I was having worked with some of the best engineers I've known, for years, and we'd more or less gotten nowhere. For a young professional it was a moment of existential anxiety but as I thought about it more I started realizing how naive I had been.
\newline

If skill was all that was needed, companies would never fail. Yahoo had incredible talent, yet it was taken down by Google. At the time of writing Google is failing to step up to the Artificial Intelligence plate in its tossle with OpenAI and Microsoft, even though the underlying technology (transfomers) was created at Google! Facebook still attracts some of the best talent in the world and yet that business seems to be going down the tube quite quickly. Skill and masterful execution are necessary but not sufficient conditions for success. And so I began wondering what other conditions were missing.
\newline

In the course of my reading and subsequent experience I've come to the following conclusion - the key to success is making failure as cheap as possible. The reason for this is that failure is inevitable, curve-balls will come your way, things you thought you knew well will surprise you, and so failure, over time, just becomes a familiar face. However, failure itself, doesn't mean the end of success, so long as you can pick yourself up, learn your lesson, and move on, you'll eventually make it. The problem comes when failure dashes you to the ground in such a way that you can't get back up - maybe you're out of resources, maybe you're not reading the writing on the wall, maybe you've eroded the trust of those you're working with - it's the magnitude of the consequences, not the failure itself that's the problem. So to find success, make failure cheap.
\newline

The easiest way to do this is to recognize that imagination is cheap \cite{macmillan}. If you discover a mistake in a design that's in your brain, you can just go right ahead and fix the problem - mental mistakes just cost time. So the more you can work things out with pen and paper, the less likely you're going to make the mistake in the real world and suddenly have money, resources, trust, and the like at risk. Proper (and thorough) design thinking is the easiest and most convenient way of making failure cheap. However it doesn't cover everything.
\newline

Your brain and the physical world are two distinct things and while you can get them into pretty close alignment, the fact of the matter is that there's no proof like the pudding. You can only answer questions in your brain that you know you need to ask - however there are a whole suite of questions that you simply won't realize exist. The answers to these questions are unknown unknowns and the only way you find out about them is by making mistakes in the real, physical world where consequences can be nasty. Therefore figuring out how to find these kinds of things early (and thereby minimize their overall impact) is also super important - we'll call this line of "failure cost reduction" iterative execution.
\newline

In my experience these two strategies - iterative execution and design thinking - \textit{are} how to make failure cheap. So, if we want to understand how to turn our skills into success we'll need to dive into each of these in turn. Let's begin with design thinking.

\chapter{Systems Engineers}
\section{Diversity First}
I've always held that games provide excellent ways of demonstrating and learning real world principles. As an example, take the game Settlers of Catan. The game itself is simple enough, players each gather and trade resources and then use those resources to build up and expand their respective settlements. As they do so they earn victory points and the first person to reach a specific threshold of points wins the game. 
\newline

In the game resource tiles are not evenly distributed among the players and who needs what depends on what each player is trying to achieve. As a result of this it doesn't take long before different players end up with disproportionately large stockpiles of certain resources and an absolute dearth of others. This sets things up for possibly one of the most entertaining components of the game - trade. An arena of backstabbing, failed loans, and bribes all trade in the game is based around one thing - differentials in value between different players. For example suppose I happen to have a large surplus of sheep but am in desperate need of stone to build cities. I will end up looking for players who are in the opposite position - having more stone than they really want or need but are in need of sheep. This differential in value sets us up for what is known in game theory as a positive sum game - by cooperating we will both end up with more value than either of us started out with. And if the deal is good enough others might actually look to interrupt our bargain in order to limit our perceived advantage. 
\newline

This simple example demonstrates a very broad and far reaching principle that has driven quite a lot of human development - differentials in what people value creates the opportunity for positive sum games and positive sum games generate growth. Loads of other examples of this principle abound throughout history. The rush to colonize the New World came from sudden access to things Europeans valued a great deal like fancy furs and silver for coinage that the New World had relatively easy access to (unfortunately no one ever thought to cooperate with the Native Americans but that is a diatribe for another essay). In the wake of global trade and global industrialization each city in the world has been able to become hyper-specialized in producing specific kinds of things allowing for loads of positive sum games that have driven a lot of the progress we've seen in the past half a century \cite{zeihan}. Then looking at specialized industries like health care we see this kind of pattern creating a diversification of jobs. Increasingly specialized doctors need increasingly specialized nurses, administrative staff, tools, and facilities.
\newline

In general the loop is clear. As people specialize they generate both specialized kinds of value but also specialized kinds of needs. If these needs and values get matched with other different but compatible specialists through positive sum games, the sum total value ends up larger than its parts. These additional resources (whether they be in the form of knowledge, capability, time, money, etc) can then be used to drive additional specialization to drive even more powerful positive sum games, and so on. 
\newline

In sum, specialists open the opportunity for positive sum games and positive sum games open the opportunity for specialists - a powerful feedback loop that doesn't just redistribute the pie through even trades (zero sum games) but actually results in growth. 
\newline

This, however, points out something really interesting about the nature of work and growth. In my own experience I have oftentimes (if not almost always) seen the focus surrounding achievements go not to the \textit{teams} but to the \textit{stars}. Sports focuses on the star players, music focuses on the star performers, engineering focuses on the folks "at the cutting edge". Obviously high performers are important but positive sum games work because of \textit{differentials in capabilities and value}. What matters is pulling together groups or teams that collaborate in the most powerful of ways across a variety of different skill sets. A team of highly compatible, highly differential players is always going to beat a team of high performers that either don't get along or are too close to one another in capabilities. A team of quarterbacks or strikers wouldn't get very far against a relatively mediocre but well diversified team. 
\newline

High performance only matters in the presence of the diversification required to take advantage of it.

\section{Process not Product}
Alright so we know positive sum games help us generate growth but we've got one problem - single positive sum games don't get us anywhere and are usually pretty hard to setup. Calling back to our game of Settlers of Catan a single trade is not going to win the game for me - I need to make repeated good trades in order to actually grow my settlements. And each trade I make has to be thought through, bargained for, and sealed. That's a lot of overhead. 
\newline

In general this applies to the real world as well. Single sales don't make companies, continued, predictable sales do. So should we renegotiate every single deal, every single time? Rarely. In general it makes sense to save on the overhead as much as possible and instead of thinking about single actions, think about creating resilient processes that will continue to produce positive sum interactions for us. As Meadows points out repeatedly in her book \textit{Thinking in Systems} \cite{meadows} single interactions don't lead to long term reliable patterns of behavior, only systems do so. So rather than spending loads of time hand crafting each and every interaction it makes a lot more sense to put that time and energy into building a \textit{process} that will reliably produce the results we want. Put another way - process over product.

\section{Avoiding the Chaos}
Now what we've talked about so far are pretty limited, small scale interactions - positive sum processes between very specific parties. However usually meeting larger goals takes the coordination of several different positive sum processes. As an example let's take an example out of Meadow's systems zoo - a car dealer \cite{meadows}. In this situation the car dealer has not one but two positive sum games to play - one with the customers and one with the car manufacturers. However, while this situation seems only trivially more complicated, it quickly illustrates what can go wrong with more complex systems. What's important in this example is that the car dealer is having to balance out how quickly she buys cars from the manufacturer with how quickly she sells them.
\newline

Now in this example, as with the real world, there are delays involved. It takes time to place the orders to the car manufacturer and it takes time to get a good sense of how customer behavior has changed. As Meadows demonstrates, these kinds of delays result in oscillations in the stock of cars at the dealership at anytime. Basically, what happens is that if sales increases slightly, given it takes time to notice the shift a drop in inventory will result. This drop in inventory ends up pushing the car dealer to put in a large order to the manufacturer. However because this order needs to compensate not only for the loss in inventory but also the increased rate of sales at some point the inventory overshoots the sales alone. But it takes time to correct this with the manufacturer and by that point the dealer will be trying to correct downward to deal with the excess of stock. This then brings us back to a lack of stock and the cycle continues.
\newline

This however is where things get interesting. The obvious thing to do here is to remove the delays as they seem to be the source of the problem. However there is really only one delay the car dealer has any control over - her own response time to customer sales. But as Meadows demonstrates by actually simulating what happens with a shorted delay, the oscillations only get worse! More or less the quicker response also results in more dramatic and therefore over-compensatory responses. As it turns out the way to dampen the oscillations is not to reduce the perception delay but to increase it so that it better matches the delay from the car manufacturer.
\newline

This demonstrates a simple but profound principle illustrated throughout Meadows' book - rational actors in complex systems do not produce rational results if they only think about their individual interactions with the system. What's required to get the system to work properly is to look at the system as a whole and tune the various interactions to work well \textit{with one another}. And sometimes the required changes seem to make little sense within the specific components that need to make the change. Our dealer was having issues due to delays but it turned out that the solution was actually making one of the delays longer!
\newline

Our car dealership was a toy example but real examples of these principles abound. Politicians in democratic governments are reelected every few years. This makes good sense if you're trying to provide a quick feedback for your voters. However it also leads to politics that has a hard time dealing with long term issues as politicians are motivated by their immediate next upcoming election. Given the risk involved in lending money it only makes sense to receive some kind of return on investment. Yet this results in a self reinforcing feedback loop that inexorably drives money into fewer and fewer hands. The principle is clear - people can act totally rationally with respect to the issue at hand and yet drive a system to horrendous results.
\newline

What this means is quite simple - systems require system engineers. It is not enough to simply throw together a system and hope that everyone doing their part will get you the intended results. You need folks who will monitor the system as a whole and modify and improve it as needed to meet the \textit{actual} aim of the system. Furthermore it is up to these systems engineers to notice when components of the system have grown old and need to be retired or when new components are required to meet or grow the system.
\newline

The people currently in those kinds of positions go by the title - manager. However, (at least in my experience) this is a relatively radical way of looking at a manager's role. Most of the time when thinking about management it seems that people tend to think of a manager like a general during a battle - handing out orders, monitoring progress toward goals, potential risks, etc. In the moment, this kind of work is obviously important, however to our earlier point about single trades not creating a business, single battles don't win a war. And this is where systems thinking comes in. Battles are fought by armies, but wars are fought by militaries. Militaries are composed of all the processes for recruitment, training, promoting, gathering intelligence, selecting engagements, developing new technologies, etc. Battles are just individual expressions of these systems. So if a military wants to stand a chance of winning battles consistently they'd better not just focus on the tactics of individual battles but also work on developing the systems that drive their military. In a similar way, while managers are certainly responsible for the tactics around driving specific deliverables, they should really see those tactics as expressions of the system for which they are ultimately responsible.
\newline

Managers should be systems engineers first and tacticians second. Indeed one might even go so far as to say that any time they have to intervene they've found a flaw in their system that needs to be addressed. 

\section{Recursion through Hierarchy}
The example of a military is a good one because it also illustrates a pretty general problem in systems engineering - how to deal with scale. One cannot expect a single person to manage every single subsystem inside of something so large as a country's military - there's simply not enough time to do so. In general we'd expect this kind of problem to arise in pretty much every successful system. Why? Because remember the whole point of positive sum games is to generate growth. That means positive sum systems grow too and so, if they're successful, at some point scale is going to become an issue. 
\newline

Thankfully we can take a queue from our own biology. To understand your anatomy you don't necessarily need to dive all the way down to biochemistry - understanding the general anatomical systems will do. If you want to understand those better you need only learn about the organs that compose them. Organs in turn are composed of tissues, which are composed of cells, which are composed of organelles, which are, finally, composed of biochemistry. Your body is an exquisite example of hierarchy. At each new level of complexity all of the components have clearly defined interfaces and roles which allows them to be considered, more or less, as black boxes - you don't have to understand their internal systems to understand how they work relative to the system they're a part of. 
\newline

Hierarchies are how systems end up dealing with scale. In general once a system becomes overwhelmingly complicated it must find ways to divide its components into individual, independent subsystems into which all of the underlying complexity can be packed and then hidden. The resulting system will then be back to a manageable level of complexity allowing for further growth of the system. 
\newline

As such, hierarchies require three things - (1) clear, distinct roles for each component, (2) clear interfaces between those components, and (3) systems engineers who can manage and be responsible for the internals of each subsystem. Without these conditions the hierarchy either doesn't effectively resolve the complexity or results in the eventual degradation of the internal subsystems. But with these, hierarchies can be applied recursively to allow systems to scale, at least in theory, indefinitely. 

\section{Pulling it All Together}
Okay so we've noted how positive sum games enable you to use specialists to drive growth, how thinking in terms of systems allows you to turn that growth from an effect and into a pattern, how systems engineers are required to get systems to actually reach their goals, and how the scale that eventually comes from the growth of those systems can be handled by building hierarchies. This provides us a useful map for thinking about how to get the most out of our specialists, drive them towards meeting useful goals, and, in the long run grow the capabilities and value available to us. But of all of this points out something very simple and very profound - \textit{organizational development is positive sum systems engineering}. 
\newline

Why does this matter? If you listen to how our world talks about the "big" issues you'll see the same thing over and over again - here's the solution that will save us. If we just build carbon sequestration technologies we'll solve climate change. If we can just elect the right leaders we'll remove poverty. Let's all get together and clean up the local lake so we don't have to deal with pollution anymore. Now don't get me wrong, all of these things would be great! However none of these solutions address the fact of the matter that the systems that got us into these messes into the first place will still be there. It may seem like the lack of a solution means nothing's there but \textit{systems always exist and the current behavior (or lack thereof) is a result of them}.
\newline

Furthermore as we saw a few sections ago, adding helpful technology can just create even more devastation. For example, our creating better and better fishing vessels has actually resulted in fewer and fewer fish \cite{meadows} because we haven't addressed the underlying lack of sustainability in the underlying system!
\newline

As a result, rather than trying to pull together specialists to create the next grand new technology or power the next great intervention, we should instead be trying to engineer the kinds of systems that wouldn't drive us toward these problems in the first place. Indeed if we build the right systems, we will end up empowering the underlying specialists to self evolve the needed solutions. However if we focus only on the solution and leave the old system in place that same system will eventually just erode our efforts away. Organization development is not about creating solutions, its about engineering effective systems.

\section{What is a System Engineer?}
Okay at this point it should be pretty clear that organizational development is all about creating powerful systems - it's the job of a systems engineer of sorts. Yet we haven't really been very specific about what a systems engineer is. And it's very important we're careful about this because if we're not we'll end up sending folks down paths that will simply overwhelm them (as it has done me several times during my career thus far). 
\newline

To see how this is the case consider for a moment that you have just become the mayor of a town. What is your responsibility? Is it really the well-being of all the folks living in the town along with all of the infrastructure, land, water, and resources? Certainly not, right? I mean that would simply be too much for anyone to really bear. For example suppose there arises an issue with a local water supply. Maybe some kind of pollution is happening. To address the issue (in a systematic way) would mean getting down there, thoroughly understanding what has led to the issue, what other system components relates to those causes, and then a lot of design, implementation, feedback, and trial and error to create and execute upon a robust solution. All of that is a full time job in and of itself and, within a town, there'll be loads of these kinds of problems \textit{all the time}!
\newline

Now at this point you might think to yourself, isn't this where hierarchies come in? To some extent yes - the mayor could simply delegate to the right people to get things fixed. But that still requires that the right people are even present, that the issue has been raised and instrumented, that the resources needed to fix the issue can be granted, and so on. So let's look at this from a slightly different perspective. 
\newline

We know that there needs to be someone who's \textit{full} attention is dedicated to just this pollution problem. I.e. there is an artifact of our system that has been identified as less than desirable and we need someone who will go in, understand the system from the point of view of this artifact, \textit{design} a new set of roles and positive sum processes to go into the system (as well as the ways in which those roles and processes will interact with the broader system), drive implementation of those processes, and see the whole thing to its (hopefully) merry conclusion. Let's suppose that this is the role of a \textit{systems engineer}. I.e. instead of systems engineers being responsible for the whole system we are now saying that they simply latch onto a specific undesired artifact (or desired but not present artifact) of the system and work to solve \textit{that} specific issue by adding in new roles, processes, and interfaces. Does this mode of thinking work?
\newline

Well you may point out, and rightfully so, that our mayor had to go and try to do this by themselves because there were no systems designers there in the first place. So what good is a systems engineer if they aren't being directed to the right problems? Well instead of looking at this as a need for specific centralized direction, we could instead look at this lack of systems engineers focusing on these kinds of problems as a bad artifact of the system - i.e. we just a need a system engineer to solve this artifact and then the "mayor" need no longer worry about these kinds of problems because the system will "self heal", so to speak. 
\newline

And if systems engineers can patch up issues in a lack of systems engineers then we find ourselves in a system that can \textit{self organize}! To be specific the "algorithm" would be as follows:

\begin{enumerate}
\item Become aware of an artifact that needs resolving (or a value proposition that needs adding)
\item Ask yourself if your awareness is thanks to the system or clearly in contrast to it
\item If it's the former simply address the artifact you started with
\item If it's the latter you've found an artifact of systems engineer not being driven towards problems of these kind - repeat from step 1.
\end{enumerate}

Now note that while this system is self organizing, there's no reason why it won't lead to things like "mayors". However the roles of these central figures of authority will be to operate within the bounds of responsibility set out by these systems engineers rather than being responsible for the system in its \textit{utter entirety}.
\newline

Furthermore note that before this the central authority had to understand the \textit{whole} system \textit{in full} - a task that quickly becomes impossible for any one human being as we rise to higher and higher levels of complexity. But now, systems engineers only need to understand the parts of systems \textit{pertinent to the value they are trying to bring}. A much more contained and manageable problem!
\newline 

So at this point we know what our systems engineers are - they seek out new value that needs adding to a system, design robust modifications to the system, drive execution of those modifications, and then seek out the next value add. And assuming they continue to ask the meta question in step 2 above, they will self organize and needn't worry about the system as a whole. Certainly now a manageable role that has specific kinds of skills that go along with it - far better than our imaginary and completely overwhelmed mayor. But we can now see why it was so important to nail this down because it should be clear that organization developers do not \textit{run} organizations, they move about and update individual components of the system and in so doing build, overtime, a \textit{self running} system. Only by approaching it this way do we have any hope of being able to scale a system beyond the bounds of any one person's ability to think and operate. Something that's totally necessary in our increasingly complex and wonderful world. 

\part{Action}

\chapter{Systems Thinking}
\section{The Building Blocks}
Before we can start building systems we first need to understand what components are available to us and what each is for. To start off it should be clear that everything you put into your system needs to be measurable - if it's not measurable there's no way to simulate things out or know the state of your system. 
\newline

Next the only things you'll really care about in your system are things that can change. Why is this?

\begin{enumerate}
\item Your system is here to do work and work can only be done on things that have the potential to change. Even if you're trying to hold something constant like the heat in your house it's only the fact that the heat can change that makes the system interesting and means it is doing work.
\item The whole point of looking at the system is to tune knobs and add additional subsystems to get things to work, so if something is just constant it's really of no interest as it's not something you can adjust. 
\end{enumerate}

Of your measurable, variable things you're going to have two kinds - stocks and rates. Stocks are things you can just measure in the moment - like how many shoes you have in your inventory. Rates require the passage of time to measure - you don't know how quickly your inventory is increasing until you measure that increase over time. You can of course have rates of rates or rates of rates of rates. So with stocks and rates in hand you've got every kind of variable possible.
\newline

Now, as was mentioned in \textit{Intuition} all the weird behaviors in systems arise from things like feedback loops, delays, etc. However, to truly understand \textit{why} these lead to weird behaviors we must introduce another distinction in our building blocks - how change to our variable measures happens. There are essentially two kinds of change our stocks and measures can have - memory-less or memory-full. 
\newline

Memory-less changes are the easier to understand. These kinds of changes happen when the \textit{instantaneous state} of your system is all you need to know what the value of a stock or rate is. Another way of looking at this is that the state of that measurable changes instantaneously with the change in system state. And this is what makes it memory-less - that instantaneous change wipes the "memory" of the past system from that variable.
\newline

Alternatively, rates or stocks can be memory-full. This happens when change is not instantaneous and therefore the past states of the system do matter. 
\newline

To illustrate these consider the thermostat example given in \textit{Thinking in Systems} \cite{meadows}. A memory-less stock is the difference in temperature between the outside and inside - it immediately changes whenever either of those stocks change. A memory-full stock is the inside temperature, it takes time for the furnace to adjust it for conduction to cause heat to escape to the outside. A memory-less rate is the rate of conduction - physics is real fast. However the heating rate is memory-full because there's a delay between the measurement and the furnace actually heating up to the needed temperature. 
\newline

At this point it should now be clear that all the weirdness in systems arises, ultimately, from the memoryfull components. The oscillations in car sales mentioned in \textit{Intuition} was due to delays and similarly the gap between the temperature you set and the temperature you get is due to memory-fullness in the thermostat system. All of this means that identifying what is memory-full vs memory-less is extremely important. But it's also relatively easy - things that take time to change or have delays involved are memory-full whereas things that act instantaneously are memory-less. 
\newline

Alright let's summarize all of that for easier reference:

\begin{enumerate}
\item All components in your system should be measurable and variable as these are the only things we can do work with/on.
\item There are two kinds of measurable in your system - stocks and rates. Stocks can be measured instantaneously whereas rates describe a rate of change of a stock.
\item Your measurables will either be memory-full or memory-less - which is determined by whether changes can happen instantaneously or not
\end{enumerate}

Finally, how should we represent all of these when diagramming systems out? 

\begin{enumerate}
\item Each measureable should have a block dedicated to itself. 
\item For memory-full variables they should have indicated each of the inflows and outflows (which will each be a component themselves).
\item Each memory-less variable should just point back to what it's a direct function of and that function should be indicated.
\end{enumerate}

Note that this now gives us a relatively straightforward recursive strategy to roughing out a system graph. 

\begin{enumerate}
\item Start with the stocks that you'd like to control/change
\item Determine if the stocks are memory-full or memory-less
\item For those that are memory-less, tie them back to their independent variables
\item For those that are memory-full, determine the rates associated with them and repeat this process for those rates
\end{enumerate}

\section{Obtaining Comprehension}
When first capturing or designing systems it can quickly get overwhelming just because there are so many possible systems flying about. For example in my line of work there's the products we're building, the ecosystems they're embedded in, how we deploy things, how we test things, how we develop things, how we continue updating things, how we run product management, etc. As a result of all of this a lot of anxiety can just come from worrying whether you've captured all the relevant systems or not. Thankfully there are some relatively straightforward ways to deal with growing comprehension. 
\newline

The first has to do with recognizing what systems do - the do work. As a result, anytime you see anyone within your domain \textit{doing something} whatever they are doing is probably part of some system. When they learn something new - there's a system. When they apply that knowledge - there's a system. When they interact with customers - there's a system. When they are originally hired - there's a system. Anytime change of any kind propagates through your domain you've got an indication that there's a system there.
\newline

This means that as a first step you can simply start cataloging work/change being done inside of your system and use that catalog to drive system development/design. Overtime what will happen is that you'll just end up finding that what you put into your catalog each day just ends up being encompassed by systems you've already drawn out and understand. At that point you'll know you've gotten close to comprehension and can switch to just trying to find outliers that haven't yet been captured. 
\newline

Now all of this may still seem a little bit overwhelming given that we're going to have \textit{a lot} of different kinds of work to incorporate into systems and building systems for them all seems a little over the top. This then is where the second piece of insight comes in - systems usually come in relatively standard, repeatable forms. For example education is a system present in pretty much every professional system. Specific kinds of management also form systems that are pretty commonplace. These archetypes therefore allow you to start placing the kind of work you're observing in different general "bins" that can each be turned into a system by taking a base version of that archetype and modifying it to your specific situation. In other words, rather than always starting from scratch we can reuse archetypes found in other situations and simply modify them. Question is, how do you know which archetype(s) to use?
\newline

Here we'll bring in a third idea - purpose. Every system has a specific kind of goal. It's the stock or stocks that are central to the system and specifically how those stocks are trying to be controlled. For example the thermostat system we've talked about before is trying to maintain a specific temperature, wall street is trying to create economic growth, and our political system is trying to keep our laws relevant to our times (at least in theory...). 
\newline

When we find some new kind of work being done we can similarly ask - to what purpose are we doing this? I.e. what stock(s) or rate(s) are we attempting to control using these actions. Then we can look back at our catalog of archetypes and look to see two things - (1) what archetypes share similar purposes and (2) which archetypes take advantage of such purpose as input. With these in hand we'll have a selection of systems we can be pretty sure our work belongs to.
\newline

Finally, regardless of how well the prior steps work, we're going to likely end up with a lot of different stocks, rates, and systems in all sorts of webs. So equally important to getting a handle of all of this is to recognize and act upon opportunities for hierarchy. Specifically it's a good idea to outline the broad view of the overall system before getting into the details of subsystems. So what distinguishes the high level from the low level? Purpose. 
\newline

The key here is to recognize means as opposed to ends. Ends are stocks or rates that are simply just valuable, not questions asked. Means are the things that help us build out those ends. Each subsystem is more or less defined by the fact that its ends are the means for some larger, "higher level" system. For example a machine learning model's goal is just to produce really good predictions. However really good predictions on their own don't produce value outright so there will be a higher level system using those predictions to themselves produce another kind of end. 
\newline

Capturing this chain of means to ends is the first step in outlining a general hierarchy and organizing the systems design work you'll be doing. Absolutely take advantage of this to organize yourself and work on obtaining breadth before depth.
\newline

Alright let's summarize this process.

\begin{enumerate}
\item Begin maintaining a catalog of work you observe that you haven't yet placed in a system by picking items off of here you'll work your way towards comprehension
\item For each of those items ask \textit{why} the work exists - i.e. to what end.
\item Then ask if the purpose of the work is really a true end or just a means, for each level you jump up (until you get to your specific role) note that you have a new level of hierarchy
\item Repeat for each piece of work in your backlog until you've exhausted the list and have a sense of all the levels of hierarchy in your system
\item For each level of the hierarchy you find using its purpose find relevant archetypes (having the same purpose or having that purpose as an input)
\item Plug those archetypes in and start modifying them as needed
\end{enumerate}

By performing this exercise consistently you will end up driving your way towards the comprehension you want. 

\section{Choosing Stocks}
When you're first identifying the kinds of work that are going on within the systems you are designing or managing it can be hard to figure out what the stocks really are. This section goes through some helpful questions to ask yourself while doing so.

\subsection{Is it a Currency of Exchange}
There are many ways to describe the same kind of change. For example if there's an optimization I have a team developing, what exactly are the stocks that are changing? Clearly I'm hopeful that the optimization is getting better, but what does better mean here and how does that translate into stocks? For example I could say the optimization might be getting more relevant, but I can assure you that trying to stick that into a systems design is going to lead to a lot of befuddlement and confusion. Why? Because no one is actually exchanging relevance - relevance is an (intended) outcome of what is actually being exchanged - which is development work on a backlog. 
\newline

When you are trying to determine what stocks are associated with the work in question ask yourself this, what is the currency of exchange here. What is being moved out of one person's queue into another's. Systems move and transform things, so without having the specific thing being moved or transformed it'll be impossible to build out a coherent system. So:

\begin{enumerate}
\item Identify the work in question
\item Ask to what purpose it is being done
\item Then determine the currencies of exchange to meet those goals
\end{enumerate}

\subsection{Look to the Catalog}
The weird behaviors that show up in systems arise from feedback loops. But feedback loops only happen in the presence of variable stocks because if the thing doesn't vary it can't drive a feedback loop \cite{meadows}. Therefore the first step in understanding \textit{non-linear} aspects of a system is to figure out what the stocks are.
\newline

Now some stocks are pretty obvious - inventory, money, heat - there's a clearly quantifiable things that come in varying degrees. Other stocks may not be so apparent right off the bat, largely because there are many less tangible stocks out there. For example, suppose that someone is asking you to make a bot that takes in queries and returns responses  about good restaurants in my area. There are two very clear stocks here in the queries and responses but at first blush these don't seem to be driving any feedback loops. However, let's pull in a less tangible stock - relevance. This is also something that is variable in a nature - depending on the kinds of data you have about the nearby restaurants (as well as how well you're using that information) it will either be high or low. If it's high you'd expect people to get their answers quickly and stop querying the service. But if the relevance is low then people might ask the bot increasingly specific questions to get what it is they want. Boom! Suddenly we have a feedback loop. 
\newline

What's key here is that there are a \textit{lot} of different kinds of stocks (and rates) out there which means there's also a lot of potential for getting overwhelmed by the variety. Therefore it's useful to keep and use a sort of cataloged taxonomy of the various kinds of stocks out there. Oftentimes you've already discovered the stock or rate that you're looking for. 

\section{Timescales}
The whole reason systems can end up having really weird behaviors is by and large due to the non-linearity that comes about from feedback loops. Our human brains see pretty much everything around us in terms of linear relationships - if you do something and it causes a small change, do that same thing more and you'll see a bigger change of the same kind. You see this same kind of reasoning in math - around any point on a curve you can approximate the curve by a simple straight line. However if you keep moving farther and farther from your starting point, the straight line approximation and the actual curve are going to get more and more out of whack. Similarly while a system may act linearly over small timescales, over longer timescales the non-linearity is going to dominate.
\newline

A good example of this is something like poverty. You can go ahead and redistribute wealth to try and remove poverty and for a short while it will. However over time wealth will begin to accumulate again until you're back in the same spot (just maybe it isn't precisely the same people who are poor this time around). 
\newline

Why bring this up? Because if you imagined a group getting evaluated on short time scales, then this single redistribution event would be considered a grand success and then everyone would pack up and going home while patting themselves on the back. Systems behavior only shows up when you are considering \textit{the appropriate timescales}. 
\newline

In general then unless we consider our systems in the contexts of many different timescales, we're not actually going to see all the behaviors that could result and either help us toward our goal or push away from it. So when considering systems \textit{you must} consider as many orders of magnitude of time as you can.
\newline

Obviously this can seem a bit ridiculous. If I'm trying to build a tool that people interface with only for a matter of moments, why would I care about systems timescales on the order of months or years? Because the parts of your system that matter over months or years won't be obvious until you do so. Only consider the short term and you'll end up building things that don't last because there was nothing built into your systems to keep the thing up to date or interesting to your consumers. Or similarly you'll build a system that, like our few-year term political system, responds to local feedback so well that it never actually deals with overarching, long term issues. 
\newline

Consider your system within each order of timescale because you won't see what might matter until you do.

\section{Multi-Actor Systems}
One of the biggest things to look out for when building systems are limited resources for which multiple actors are competing. For example each year there is only so much wealth going around and having wealth makes it easier to acquire more wealth. This means the rich get richer and the poor get poorer. This is an example of parallel system actors acting on a limited common resource. What do I mean by a parallel system actor? I mean that each entity in this economy has the same set of stocks and rates (their own personal wealth, their ability to work, etc) which influence how they are able to act on the overall system. Note in this case we can't necessarily draw things out because the number of actors is not necessarily given to us (or large enough to make the drawing cumbersome).
\newline

The reason to look for such situations is that each actor is going to evolve (or go extinct) with respect to how they compete. As we just pointed out the rich get richer in a system with nothing in place to regulate the obvious feedback loop of lots of wealth leads to more wealth. So if you want to understand what features of your limited resources are going to evolve you have to look at how people compete for them. 

\section{When Systems aren't Relevant}
Systems, as we've discussed so far, occur anytime we have variable stocks or rates that are doing work. And while this may seem like pretty much everything out there, as with anything else, there are boundaries to the relevance of thinking in this way. 
\newline

The boundaries are both fundamental and practical. For example when I'm thinking about an organization at a very high level, trying to capture every little detail of the system just becomes overwhelming. So, as we've pointed at before, we start to build hierarchies of subsystems. But at the higher levels those subsystems are not really thought of as systems in and of themselves. Instead they are thought of as black boxes that simply take in certain inputs, provide certain outputs, and may or may not be modified by various stocks in the high level system. But by and large we can think of these black boxes as taking one kind of stock and turning it into other kinds of stocks - these black boxes are transformers. 
\newline

However sometimes transformers arise out of more than just simple practicality. For example in the section \textit{The Building Blocks} we pointed out that sometimes things are just direct (memory-less) functions of other parts of the system. The changes in temperature due to conduction in our thermostat example was one of them. Another example would be something like a physical lever - use it one way and it'll always respond in kind. Clearly then, when something acts simply like a mathematical function - provide input, get output (even if there is a delay in getting that output), then we can think of that thing as a transformer. 
\newline

To expand on this, certain systems \textit{become} transformers at higher levels because their non-linearities aren't relevant to the higher system. For example the applications on your phone are systems that have loads of non-linearities in them that require systems engineers to think carefully about queues, cpu and memory usage, and so on. However at the level of the user, all of these things (at least during nominal operations) are hidden. Therefore to the user this simply becomes a memory-less transformer. There's a lot going on under the covers to make it memory-less, but if it's done well it achieves the goal and the user never knows about it. 
\newline

So when we're building systems the boundary of where that thinking remains valuable comes down to where we get to memory-less components (whether they are that way by scale, design, or nature). In those cases rather than systems we have transformers, and understanding transformers is a little different than understanding systems.

\section{The Archetype Catalog}
\subsection{Education}
\subsection{Development}
\subsection{Product Management}
\subsection{Applications}

\section{The Currencies Catalog}
\subsection{Backlogs and Queues}
\subsection{Standards and Rates}

\chapter{The Art of Being Less Wrong}  
\section{Why Precision Matters}
This is one of those things that is superbly obvious when you think about it but somehow always falls through the cracks. When we're trying to create anything of value we (hopefully) start with a design process that helps us think through what we're going to build, how we're going to build it, what obstacles we may face, etc. In theory what we want at the end of that process is more or less a proof that the design we have will result in the value we want. However the rigor of that proof \textit{completely} depends on the precision in our representation and language. 
\newline

As a simple example if I'm imagining have a restaurant I could have a design which outlines what kinds of cuisine I'm going to sell, what kind of setting, where it'll be, etc. However this design is incredibly high level and misses a lot of detail that's critical to actually getting the restaurant off the ground - how much will all of this cost, how will I manage inventory, what'll be my hours, how will I manage my staff, etc. As the precision in both my questions and answers increases my ability to note faults in my thinking grows and the confidence that I will hit my target increases as well.
\newline

So if it's so obvious that precision is needed, why is it so often overlooked? Because being precise means \textit{knowing the right kinds of questions to ask and what good answers look like}. Therefore as with many of the other practical guides here in this book this once again comes down to creating a catalog of the kinds of questions to ask as well as what counts as an answer. That then is what the rest of the chapter is all about.

\begin{enumerate}
\item Units
\item Inputs/Outputs
\item Time
\item Explicitly drawing out the proof
\end{enumerate}

\chapter{Things I'd Like to Include}
\begin{enumerate}
\item The fact that lots of times we think of building products and the product as the system we are managing when really the system we are managing is the organization doing the work of building the system (everyone at Viasat doesn't think about development as a system at all, the only systems we think about are technical ones)
\item How does all of this practical stuff tie back to our positive sum systems idea?
\item The need to include specific feedback loops (Network Earth needs attachment to the actions it's trying to inform otherwise it'll produce no useful information)
\item Accuracy as well as precision (i.e. backing what you see empirically) [the issues with EwA and low impact]
\end{enumerate}

\bibliographystyle{plain}
\bibliography{reference}
\end{document}