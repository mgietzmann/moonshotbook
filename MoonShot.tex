\documentclass[11pt,a5paper]{book}
\usepackage[utf8]{inputenc}
\usepackage{amsmath}
\usepackage{amsfonts}
\usepackage{amssymb}
\usepackage{graphicx}
\usepackage[super]{nth}

\title{How to Shoot for the Moon and Land in the Stars}
\author{Marcel Gietzmann-Sanders}
\date{}
\setcounter{tocdepth}{1}
\begin{document}
\maketitle
\tableofcontents
\newpage
\part{Introduction}
\chapter{What is This Book?}
No one person alone can shoot for the moon and hope to succeed. Yet that is exactly where we begin - alone with a dream. This book is an attempt to understand how to take that dream and build an organization around it.
\chapter{My Dream - Network Earth}
\section{The Biosphere}
Take a breath. 
\newline

Washing into your lungs is a rich mixture of gasses containing the all important and life giving element - oxygen. While you may traditionally think of oxygen as coming from the plants around you, it turns out that the mixture swirling about in your lungs comes from a much wider and more exotic set of sources. The majority of the oxygen you are breathing comes from our oceans - one in five breaths from a cyanobacterium named Prochlorococcus \cite{kmorsink}. On the complete other end of spectrum are rain-forests like the Amazon - a place that produces something close to one in ten of your breaths \cite{ymalhi}. The result of this is that the brew of gasses swirling around in your lungs has come from creatures smaller than the single strands of hair on your head to those as tall as multistory buildings. However, the story doesn't end there.
\newline

If you're near some kind of urban center, you probably deal with some level of pollution. The daily goings of urban living produce loads of pollutants such as nitrogen and sulfur dioxide. However, plants have a habit of pulling such pollutants out of the air and storing them safely away in their tissues. In fact a 2006 estimate found that in the United States alone urban forests were responsible for more than 3 billion dollars of air purification \cite{dnowak}. So while most of your oxygen was sourced from our oceans and rainforests, it's purification was local and done by the trees and shrubs in your back yard. 
\newline

Take a bite. 
\newline 

If you happen to be eating cassava from South East Asia then you can thank a tiny parasitic wasp - \textit{Anagyrus lopezi} - that is reared in the millions to control populations of a devastating pest - the mealybug \cite{wpark}. An enormous boon for both the financial and food security in the region, the economic value of this millimeter long wasp is more than \$14 billion a year. 
\newline

If instead you're dining on a lovely, flaky slab of wild salmon, you are effectively eating the entire ocean. As predators that sit relatively high in the food chain, they depend on "bait fish" like herring as their primary source of food. Herring feed on things like krill, who in turn eat the very plankton providing us with the majority of our oxygen. If any one of these levels in the food chain starts to experience issues so too does your flaky slab of salmon. 
\newline

Finally, if there are plants on your plate there's a reasonable chance you can thank the leagues of beetles, flies, butterflies, and wasps managing pollination in our world. These services are supplied to an estimated 75\% of crop species, and are worth something upwards of \$200 billion every year \cite{avanbergen}! Without the insects pollinating the plants that produce our food the variety in our diet would shrink dramatically.
\newline

Go on vacation. 
\newline

Diving at a coral reef? The colors, the vivacity, the sheer brilliance of all the fish darting this way and that around you? It's thanks to the sharks. Sharks specifically target sick or unhealthy fish, thereby improving the overall health of fish populations and possibly preventing outbreaks of disease \cite{reefcause}. Furthermore, they help prevent the kinds of population explosions that lead to mass die offs. Just another way they preserve the fish we all love seeing. 
\newline

Hiking alongside giant sequoias or passing through the incredibly diverse south african fynbos? Both of these are only possible thanks to the drama of fire and lightning. Sequoias depend on fires to open their tightly sealed cones and the resultant sprouts depend on the brush clearing power of the fires to give them any chance of seeing the sun \cite{california}. Then, of the more than 9,000 plant species found in South Africa alone, many, like the proteas, also depend on fire to release their seeds and prepare the ground for the next generation of amazing \cite{shoek}. So why do these fires occur? Lightning is the spark, but the reason the fires spread is thanks to the vast diversity of plants that leave behind dried out remains. When it comes to sequoias and proteas, it truly takes a village. 
\newline

Enjoying all the ocean has to offer in Florida? Thank the mangroves. Not only do mangroves provide nurseries for an incredible diversity of creatures including the fish that crowd coral reefs and the manatees and birds that people flock to see (pun intended), but they are also veritable buttresses against coastal damage. One estimate suggested that during Huricane Irma in 2017 they prevented over a billion dollars of flood damage! That's a lot of insurance. 
\newline

Take a step back. 
\newline

These examples are just a tiny taste of how \textit{every single aspect of our lives} is deeply, intrinsically dependent on the living world around us - the biosphere. The Millenium Ecosystem Assessment identified 25 different categories of ecosystem services - and these are as broad as things like food provisioning and erosion control \cite{mas}! From millimeter wasps to three hundred foot tall trees, from ecosystems half a world away to the shrubs in your back yard, our world provides our food, water, air, health, happiness, and so much more. It is our life support system, our spaceship as we hurtle through the vast, cold emptiness that is space and for the first time in our history we have the power to overwhelm it. 

\section{The Anthropocene}
Let's start with some perspective. We as human beings have a very hard time thinking about pretty much any period of time longer than the ones we've already lived. Tell me something is ten years old - no problem I can grasp that. Tell me something is one thousand years old? While I can make it out mathematically I really don't have any way to grasp just how much longer that is. 100,000 years? Forget about it. So let's use a frame of reference that's a little easier to understand. Let's pretend all of our universe's history fits in a single year \cite{csagan}. 

From this point of view (and if we accept standard scientific time lines) the big bang starts us off at midnight January 1. The next four months are spent in a dizzying dance of expanding hot gas, newly forming stars, and, finally, galaxy formation with the milky way only showing up around May \cite{eellis}. But don't assume that because of this our time line is about to speed up because we still have to wait until September before our solar system even forms! We've gotten roughly three quarters through our year and our planet has only \textit{just} shown up. Surprisingly, given how long it's taken to get to this point, it only takes 20 days for the first things we'd deem living to show up, but then things slow down again as we have to wait until December (around 70 more days) before the first multi-cellular organism joins the scene. Now things really start to take off. 

15 days later we've got plants colonizing the land masses of our planet. 3 days after that the first reptiles join them. Only a mere two days later and we've entered the age of the dinosaurs. Think about that, in less than a third of the time it took to figure out how to make multi-cellular organisms we've gone from nothing macroscopic on land to dinosaurs! A day later, our ancestors - the first mammals - join the production and by December 30th (4 days later) the first primates appear. 

But hold on one second. Did I just say December 30th? Does that mean we've got 1 day left in our year and humans haven't even shown up?! That's right. But it gets even wilder. Homo sapiens, us, don't show up until 12 \textit{minutes} to midnight. Just chew on that for a second. All of human history, finding fire, inventing tools, starting the first cities, all the empires, all the wars, absolutely \textit{everything} that our species has ever done occurred in the last 12 minutes of our universe's year. That's the amount of time it takes to heat up a microwave dinner!

But it gets even wilder because we, as a species, didn't figure out how to farm until 28 \textit{seconds} before midnight. So really, everything you've learned about in your human history courses gets packed into about as much time as you spend watching a \textit{single} tiktok. We are but a blip in the history of our universe. But, oh, what a blip we are. 
\newline

From nature's perspective, humans have generally been, with few exceptions, a footnote. There's a reason why the oceans were terrifying to early humans, or why people prayed to gods to stave off floods or famine. If nature decided to have its way, there was little we could do about it. But post the industrial revolution and especially since 1950, that relationship has \textit{utterly} changed \cite{eellis}.

Let's start with some simple numbers. Around 71\% of the land on our planet is considered habitable \cite{hritchie}. Of that habitable land we have put around 46\% of it into agricultural production. That means we use a solid \textit{third} of the planet's surface (not covered by ocean) to just feed ourselves. Except that we eat more than what's on the land. When it comes to fishing the picture is even wilder. Of global fish stocks we overfish 34\% percent and maximally fish 60\% \cite{mroser}. That means there are only 6\% of fish stocks that we could fish more without starting to deplete their numbers!

The amount of water we use is no less incredible than the amount of area we farm. Each year approximately 40,000 $km^3$ of freshwater flows into the ocean \cite{eellis}. Of this only around 13,000 $km^3$ is accessible to humans with our current technology. Yet nearly half of that flows through human engineered systems! So not only are we using half of the accessible land, and pretty much all of the accessible fish, we're using half the available fresh water too. 

In the 2004 the International Geosphere-Biosphere Programme published a report that illustrated what is now called the Great Acceleration \cite{wsteffen}. Across 24 distinct measures from total GDP to atmospheric carbon to the number of rivers dammed they found an explosive spike in activity starting in the 1950's. We've already talked about a few of them (farm, fish, and water use) but let's sprinkle a few more in for color. 

Since around the beginning of the 1900's we've lost 30\% of our forests. We've now dammed close to 25 thousand rivers. We consume about 300 million tons of fertilizer per year. Our global population has gone from 2 to 7 billion. And we have more than 3 times as many floods as we used to per decade. But here's the one that kind of summarizes it all for me - extinction rates have gone through the roof.  
\newline

Extinctions are nothing new and in general pretty normal. As time marches on species come and go and that's... well... just evolution doing its thing. So what matters is not \textit{whether} extinctions are happening but \textit{how many} are happening. 

In the past we've had some pretty spectacular extinction events. Around 444 million years ago the Ordovician-Silurian extinction took out around 70\% of everything living at the time \cite{adubey}. 372 million years ago the Late Devonian extinction took out some 70\% of all marine species. The Permian-Triassic extinction, which occurred roughly 252 million years ago, took out over 70\% of everything alive again. Then in the Late Triassic (208 million years ago) we had yet another mass extinction. Finally we have the most famous of them all, the Cretaceous-Paleogene extinction where 66 million years ago over 60\% of all species died as the result of a massive asteroid smashing into our planet and creating the equivalent of nuclear fallout. Bad for the dinosaurs but great for us as mammals often are attributed their place center stage thanks to that big old space rock. 

Since we were visited by Armageddon-a-la-space extinction rates have pretty much held at their normal, background levels. Until, that is, the Great Acceleration. Since the 1950's extinction rates have skyrocketed to more than 1000 times their background rate \cite{wwf}! At the current rate scientists estimate that 50\% of all species could be gone by 2100. We have entered the sixth extinction. 

Here is where our current impact on our biosphere really comes into perspective. Mass extinctions in the past have occurred as the result of insanely powerful events. Country sized asteroids smashing into the planet. Massive anomalies in global temperatures (sound familiar?). Such incredible dips in the amount of oxygen that more than 70\% of all living species just... well... died of oxygen deprivation. Causing mass extinctions requires planet shaping power. \textit{We now possess that kind of power}. And as a result we've entered an entirely new geological epoch, an epoch where humankind is the domineering, world shaping, power - the Anthropocene \cite{eellis}. And in the perspective of our "universe in a year" we managed this power grab in less time than it takes to watch a tiktok.
\newline

Contrary to how you may be feeling, none of this is meant to scare you (although fear and awe are perfectly appropriate reactions). Instead it's meant to impress upon you the gravity of our situation. In the first section we got a glimpse of just how important the biosphere is to us and now you've got a taste of just how powerful we, as a species, have become - that asteroid that took out the dinosaurs is taking notes from us now. Thankfully this power means that in so much as we can absolutely wreck our world, we can also take care of it too. The problem is not that our world is doomed, but rather that we can no longer just take it for granted. Given our planet shaping capabilities we have no choice but to become active, conscious stewards of our planet. So the question is simple, how do we do that?

\section{A Revolution}
Suppose for a moment you have a garden. While at first it may seem like all you need to do is water your plants and let them flourish, anyone who's tried gardening before knows there's a lot more to it than that. For one thing watering is a tricky business. Water too much and your plants can get water logged and die. Water too little and your plants will wither up and die. Plant your plants too close together and they won't be able to grow properly. Plant them too far apart and you'll be wasting loads of garden space. Then there are the pests. Caterpillars, beetles, mice, deer, you name it it's probably trying to eat your plants and every single one of these needs a different strategy for dealing with. Then there's disease. Plants, like us, can get sick and it's especially likely if they're not getting the right balance of nutrients and care. Diagnosing these sicknesses is difficult enough that there are apps to help you with the task and then there are a whole slew of different methodologies to deal with the various ailments. And we haven't even gotten to the fact that choosing plants can be its own nightmare. Do you have enough sun? Too much sun? Direct or indirect? What's the soil quality, what's your growing zone? On and on it goes. Point is, stewardship of your garden requires a lot of specific knowledge about the plants you a growing, their life histories, the life histories of everything they interact with, the place you live in, and so on. 

So now replace your garden with the entire world. Obviously the amount of knowledge you need is going to skyrocket but something else changes as well. In your garden you've got lots of room to learn by trial and error - if you manage to kill a plant (like I tend to do) it's no big deal, just try again next year or in a different plot in your garden. We, however, have only one world. While we can certainly try little things here and there, at the end of the day, we've got to be pretty sure we know what we're doing before we launch large scale efforts to shape our world one way or another. And that requires models sophisticated enough to let us trial things in simulation. So not only do we need a lot of knowledge about how our biosphere works, we also need to make sure it's the kind of knowledge that allow us to build simulations. Turns out, this kind of science is pretty new. 
\newline

Some history. The scientific revolution started in the 1500's and was set off by Copernicus' now famous idea that perhaps we aren't at the center of the universe \cite{sbrush}. What followed was a massive explosion in all things astronomy and over the course of a few hundred years everything we thought we knew about our place in the universe changed. In 1620 Robert Hooke discovered cells for the first time launching the whole field of microbiology \cite{wsd}. In 1735 Carl Linnaeus put together the system for classifying plants and animals that we use to this day and in 1745 Ewald Jürgen Georg von Kleist invented the first capacitor. Fast forwarding to the 1900's and we've got Max Planck's theory of black body radiation in 1900 followed quickly by Einstein's special relatively and his explanation of the photoelectric effect in 1915 - all theories that were seminal in the creation of modern physics. In 1953 Wilkins, Franklin, Watson, and Crick made the now famous discovery that DNA has a helical structure - thereby totally changing our understanding of the stuff - and in 1996 we cloned a sheep for the first time. So, given all of this progress, you might be surprised to learn that it was only in 1999 that Hans Schellnhuber published a "groundbreaking" paper on earth system modeling that spoke of a "second Copernican revolution" in our ability to model the biosphere \cite{hschellnhuber}. Our first thoughts on this issue came around the same time as our ability to clone things and well after we came up with the idea of black holes in space. 

To understand this we need to look at what was going on alongside each of these revolutions in science. The scientific revolution, with all of its progress in astronomy, came at the same time as the first telescopes were being invented. Hooke made his discoveries about the cell thanks to the newly invented compound microscope. Linnaeus only came up with his system of taxonomy after extensively traveling the world - something only possible thanks to all the nautical inventions that also created the "age of discovery". The first electrical inventions were all thanks to developments in what materials were becoming available as the industrial revolution unfolded \cite{tkuhn}. The discoveries of modern physics were all thanks to continuous improvements in our ability to poke at smaller and smaller things using tools like the x-ray diffraction techniques that allowed Wilkins, Franklin, Watson, and Crick to infer the structure of DNA. The pattern from all of these examples should be pretty clear - to understand our world we need the tools to probe it and each scientific revolution has been attended by updated capabilities in our ability to gather data \cite{tkuhn}. So it should come as no surprise then that one of the major tools used by earth system modelers - satellites - are a relatively recent development. As an example, the Landsat program, one of the largest programs to provide satellite imagery of Earth, only started in 1972 \cite{wls}.

But satellites are only part of the picture and the need for further instrumentation and tooling is clear. Whether it's land management \cite{jpongratz}, pollination services \cite{ibartomeus}, or general biodiversity management \cite{hkuhl} (just to name a few) scientists in recent years continue to point to a lack of comprehensive data as one of the primary challenges they face world over. Science is driven by data, and the data is sorely lacking. 
\newline

So if we want to spark another scientific revolution, one that will give us the tools we need to be good stewards of our planet, we need to start providing the instrumentation required. So question is - how do we do it?

\section{An Aside}
Before we dive into how to develop instrumentation I want to stress something. The last section argues that we \textit{need} models of our biosphere and all the attendant knowledge in order to be good stewards of our planet. I firmly believe this is true. However (and this is important) I \textit{do not} believe that such knowledge should block actions that are already supported by the science that exists \textit{today}. I \textit{do not} believe we should wait to decarbonize or wait to stop wasting as much as we do or wait to stop polluting. These are all good, clear actions that should be taken \textit{now} because the science clearly backs them. In general there will always be gaps in our knowledge that need filling, so if we ever want to be good stewards who actually take action we have to learn to listen to the knowledge we have. 

Alright, back to the scheduled programming. 

\section{An Industry}
We've got a problem. While pointing out the immediate need for biosphere instrumentation is cool and all, it's also laughably vague. How on earth are we supposed to execute on a mission statement that's that high level? Fact is, we need a strategy. 

Strategies start with understanding our goals and, especially in our case, clarifying goals is really important. The history of conservation and "biosphere activism" has... well... a checkered past. Early biologists exploring the pacific islands sometimes caused extinctions in "the name of science" and part of the reason we're having to reintroduce wolves all over the place is that early conservationists helped get rid of them in the name of creating a more bounteous United States \cite{mnijhuis}. At the other extreme, in my own experience, I've found that frustration with human negligence often leads to a desire to take humans out of the picture entirely, a philosophy that culminated in the Half-Earth Hypothesis - a proposal that we should set aside half of the planet as a human free zone (and they're not talking about totally uninhabitable places) \cite{ewilson}. 

But let's look at intentions. The folks causing extinctions in the pacific islands? They just wanted to study birds. Unfortunately, at the time, studying birds was synonymous with shooting them. So while studying birds is a great idea, shooting them all as a means to an end is a little more than short sighted. 

How about the wolves? Well quite a lot of the money in conservation coffers comes from hunters. So making our forests more bounteous isn't a half bad idea. The problems are twofold. First, it turns out wolves are needed to keep deer populations healthy (ecology!). Second, extirpation is a really extreme take on "reducing" wolf populations. So once again, reasonable intentions with a short sighted strategy.

Finally I get the frustration with people seemingly just bulldozing everything that isn't "producing human good" or treating the natural world with less than respect. But the problem here is one of education - rather than understanding that rainforests are useful natural resources, most economies see them as unrealized cattle farms instead. The problem isn't that humans are included, it's that those same humans aren't including the natural world in their calculus. 

Regardless of the various strategies taken, one thing is clear from all of these examples. The goal is to nurture the value that our biosphere provides us. Unfortunately this is still too vague for our purposes so let's dig deeper.
\newline

The Millennium Ecosystem Assessment \cite{mes} suggests a rather lovely framework for thinking about all of this. They suggest (and the suggestion is backed by data) that human well being exists on a continuous spectrum with poverty that is made up of five key components - the necessary material for a good life, health, good social relations, security, and freedom and choice. So while you may have a lot of material wealth (1) if your health (2) is extremely poor because of, say, pollution, you are still impoverished to some degree. In other words there is no such thing as poor or wealthy. Instead there is a five dimensional spectrum that we have to keep our eyes on at all times. 

As a quick example of why this is important, almost every assessment of ecosystems that I have read is stated in dollars. Even the first section of this essay had a dollar bent! But how do you put dollar values on social relations, feelings of security, or autonomy? You obviously can't! Having these five dimensions of human well-being helps us break out of the habit of only seeing value as economic. And this level of specificity and holism is exactly what we need in our goals.
\newline

Okay so that gives us a better sense of how to make sure we're measuring value completely, but how on earth are we supposed to connect this back to the literal jungle's worth of complexity in our biosphere? Trying to work out all the ways in which our biosphere provides for, say, our health brings the phrase - boiling the ocean - to mind. Clearly we need to divide the problem up into more manageable parts. Once again, the Millennium Ecosystem Assessment is here to help \cite{mas}. 

As I mentioned before they have already done the work of divvying up ecosystem services into nice neat bins. At the top level we have three overarching categories - provisioning services, regulating services, and cultural services - which are each then divided up into specific sub categories such as food and fiber, water regulation, pollination, sense of place, disease regulation, and so forth. These, obviously, are much more manageable pieces of the overall biosphere pie.
\newline

Alright, let's bring this all back. How does this help clarify our path to building biospheric instrumentation? Well first we know now that there are 25 specific models that we are trying to build - one for each ecosystem service category. Second we know that in order to be complete, those models need to be capable of helping us predict the effects on each of the 5 dimensions of human well-being. Finally we know that those models need to be comprehensive. So we can clarify our path forward as such - we need to help scientists obtain the data required to build comprehensive models that connect each of the 25 ecosystem service categories outlined by the Millennium Ecosystem Assessment to the 5 dimensions of human well-being. Which leaves a much simpler question - what data is required?
\newline

Here, we come to the kicker. Scientists have not exactly been twiddling their thumbs. Huge strides have been taken in learning how to think about and model this kind of stuff. Remember, the issue scientists have is not in having \textit{no} data, it's in not having \textit{enough} of it. Pull down the journal Ecological Informatics and you will find \textit{hundreds} of examples of instrumentation techniques that scientists have already developed and proven useful. They've already done the de-risking work and know what data they need - they just need help deploying the instrumentation at scale! And doing this, would be a real value add. 

Think about it this way. Most of us could, if we really wanted to, wire our own homes, do our own plumbing, and so forth. But doing so requires buying all sorts of tools, learning all sorts of techniques, going through all sorts of safety training, and so on. That's a lot of overhead for one house. A professional electrician also has to go through all of this overhead but then they apply their tools and training to loads of homes. This is the economy of scale - and it makes a big difference. We absolutely need this same economy of scale if we're going to attempt biosphere-wide instrumentation. And that means we need professional instrumentors. Put a slightly different way, to achieve the economy of scale, biosphere instrumentation can't just be limited to a research activity - it needs to be its own industry, driven by its own professional workforce. To trigger the scientific revolution we want, we need to build an entire industry.
\newline

\section{Conclusion}

If we're going to be sophisticated stewards of our planet, we need a scientific revolution in biosphere modeling. Scientific revolutions in the past were driven by revolutions in instrumentation. Today is no exception. Our strategy? Using the 5 dimensions of human well being and the 25 ecosystem service categories outlined by the Millennium Ecosystem Assessment as a guide, find the researchers already building the instrumentation we need and create professional industries dedicated to deploying and maintaining that instrumentation at scale. Let's create a biosphere instrumentation industry - let's create Network Earth.

\chapter{Why This Book?}
When I was a child, my family lived across the street from a wonderful little pond. Every time it rained our yard, the road, and, of course, the pond itself would become chock full of amphibians. I'd run out there in the rain and catch several of them to keep for a few a days in the bank of terrariums that my parents had purchased for my brother and I.
\newline

My brother and I just liked playing with them. We'd line em up and see which could jump the farthest, we'd build places for them to live with Legos, and before we let them go I'd take pictures of the patterns on their skin so I could check if we'd met before. Then, after their short stay was over, we'd bring 'em back to where we found them and let them go on their merry way.
\newline

I like to think that the stay with us was pleasant for them as much as for me, but I suspect that I got the lion's share of joy out of those interactions. Indeed catching critters was such a formative part of my life. I became obsessed with learning everything I could about the living world and had stacks upon stacks of National Geographic issues, natural history books, and field guides. It seemed obvious to everyone around me that I'd end up becoming a biologist of some kind. However fate had other things in mind.
\newline

Through a whole of series of events and existential crises that seem as bizarre to me now as they did when I was a hormonal teenager, at the age of 16 I found myself in a dormitory at Worcester Polytechnic Institute (WPI) to study physics of all things. But physics soon turned into math and by the end I'd given up on that too and was teaching myself software engineering and machine learning. Upon graduating with a math degree (I switched to computers too late) I took up a role as a "Member of the Technical Staff" at a telecommunications company that I'd interned with the prior summer.
\newline

Over the next couple years I ended up being relatively successful as an engineer and even ended up co-managing the team I was on for a while (something I was wholly unprepared for). However throughout that period I just felt like there was something missing in my life and so,after another existential crisis or two, I ended up wondering if it was time to return to my roots - wildlife biology and conservation. 
\newline

Thankfully I did the rational thing and decided to run a small experiment instead of just quitting my job wholesale - over the course of an academic year I took a marine biology course and a class in conservation. In the process of taking these classes I came to the realization that the kinds of technology I had been building over the past few years were almost completely absent from both biology and conservation. It was clear that those in the field could stand to take advantage of what I knew and so (at the advice of my professor) I started looking around for ways to volunteer my skills.
\newline

Over the next couple of years I volunteered with a bunch of different groups. At one point I helped a couple biologists automate a machine learning pipeline. At the end of that project it was clear that what we'd just built could be applied to so many other things, yet that project (for reasons I still wonder about) just sort of disappeared. I ended up working with a marine rehabilitation group and after a couple of months doing a lot of manual, analog stuff that definitely could get automated started floating the idea of modernizing some things. I was very quickly shut down and left a few months later. At another point I started contributing to an open source project that was dear to me but quickly found that they were so overwhelmed they could only really hand off  more trivial work to outsiders. Indeed in perusing their forums I quickly discovered that due to a lack of funding they had to shut down most of the more interesting features people using the tools wanted. All of this left a rather sour taste in my mouth. I couldn't get away from this feeling that I had so much to offer but no one seemed to have the time nor resources (and in some cases vision) to take advantage of it. Everyone was happy to take my money, or put my time to menial tasks, but no one really had a place for my \textit{actual} skills.
\newline

So, after a time, I decided to take things into my own hands. If others weren't going to put my passion and brain to use I would - and after months of reading, talking to people, writing, prototyping, and thinking I came up with the idea for Network Earth. But as I began working on that project something clicked in my brain. 
\newline

The whole idea behind network earth is to create industries around the kinds of technology that can help us instrument our biosphere. There are 25 different ecosystem services to monitor and loads of technologies to manage in each. I suddenly realized that Network Earth didn't just solve my own problem, but it could provide a direction for anyone else in my position - and goodness had I met a lot of those people. When working at the rehabilitation center I'd been surrounded by \textit{actual medical staff} who were being tasked with nothing more than doing laundry and mopping floors. In both of the classes I took nearly everyone there had loads of energy and ideas and each of them was looking for a way to help out. At work itself I was surrounded by colleagues anxious about the state of our world but unsure of how to become involved. So much skill, talent, passion, time, money, and energy with nowhere to go. 
\newline

And so I had an epiphany that I probably should have had ages ago (but as they say, better late than never) - the real problem to solve here is not the biosphere stewardship crisis but how to harness all the energy and skill of those who \textit{already want to do something about it}! If I could build the organization that I had spent the last few years looking for I could harness that energy and end up being a much more effective force of change than if I built one specific technology to solve one specific part of the problem. 
\newline

This, however, presented a problem. While I had some small amount of management experience I was totally unprepared to \textit{build an organization from scratch}! I clearly needed to learn organization development and quickly. Thankfully the stars happened to align in my favor.
\newline

Learning anything comes in two parts - theory and practice. Without the theory there's no way for you to "stand on the shoulders of giants" and learn from those who've come before you. But without practice it's all in one ear and out the other. Theory I knew I could do, I'd just simply read everything I could about organizational development and management until I felt confident there wasn't much left that would surprise me. But practice was going to be the harder thing to sort out. Happily, at around the same time as this epiphany I ended up getting something of a promotion at work and found myself co-leading a team of about a dozen people. And that team was tasked with building out a component of our broader organization that, up until that point, never really existed in any clear form. I found myself with a perfect mirror to the problem I wanted to solve - I had a group of brilliant people who's energy needed to be harnessed toward an organization that did not yet exist. Ample opportunity for practice had just landed in my lap!
\newline

Which brings us to why I started writing this book.
\newline

This book serves two purposes. First, I'm a firm believer that you don't really understand something until you can explain it to someone else. Therefore I am using the process of writing this book to firm up my own understanding of organization development. However I am also a firm believer in self organizing systems. To my mind the best systems are ones that can build themselves and so, if I do end up being successful in building organizations, part of that success is going to come through teaching others, within the organization, how to build organizations themselves. When that time comes I hope this book will be a central resource.
\newline

So, this book is nothing more than the compendium of the things I've learned about organizational development. And while a lot of it is going to feel philosophical or abstract in nature you can rest assured that I myself lean upon these principles each and every day I spend leading teams, building organizations, and dreaming of a better world. 


\part{Intuition}
\chapter{Creating Catalysts}
\section{Skill}
It's a reasonably well established fact that the average human being can only hold around 7 things in short term memory at once. Put another way, if I start sounding off digits to you in quick succession you'll only reliably be able to remember 7 of them at a time. However there are a rare handful of people out there who blow this number straight out of the water. There have been people who've been able to recite upward of 80 digits consistently and others who've been able to demonstrate this trick for 100's of digits at a time! Given the rather measly limitations on the average human's brain you may be wondering what makes them so special. The answer? Nothing at all.
\newline

For example take Steve Faloon - a student the psychologist Anders Ericsson recruited to study how far people can push their short term memory. When Steve first started he, like the rest of us, could only reliably recite 7 or 8 digits from memory. Yet by the end of this experiment he was reliably reciting 82 digits at a time! How had he made this jump? By using mnemonics. Being a runner, Steve realized that he could group series of numbers into running times. So if he saw the digits 1, 0, 3, 5 he'd remember the running time 10 minutes and 35 seconds. This alone allowed him to focus on remembering 7 running times instead of 7 digits which translates into around 28 digits overall. A huge leap forward! By continuing to create structures like these that made better use of his normal short term memory, he was effectively able to extend his short term memory by over 10 times. 
\newline

Another feat of intellect that seems near miraculous to the rest of us is the ability of master chess players to play blindfolded or without a board. Indeed many master chess players have taken this to the extreme with some playing a couple dozen games simultaneously all while unable to see or touch any of the boards! Wondering what sets them apart, a set of psychologists performed a rather interesting experiment. They took a national-level, mid-level, and novice chess player and showed them a board with a whole load of chess pieces arranged in a way that might arise from a game of chess. Then with the board removed they asked them to recall the arrangement of the pieces. Unsurprisingly, the better the person was at chess the better they did at remembering the pieces and where they were. But then they showed those same people a chess board with \textit{randomly} arranged pieces, took the board away, and asked them to recall again. Surprisingly in this case there was no longer any difference in performance! The national-level chess player no longer had any advantage. 
\newline

In interviews with a variety of experienced chess players, it has become clear that what they are seeing in their head when they play chess with no board are "chunks" of interacting pieces. They'll see setups that they are familiar with from prior games, lines of attack or pressure that have been important in the past, etc. And so when you randomly arrange the pieces on the board, none of these chunks can be found and so the ability to remember it with any level of real skill goes away - once again the exceptionalism is coming from specific mental representations that they've built up. 
\newline 

Ericsson and others studying exceptionalism have found this pattern over and over again pretty much everywhere they've looked. From swimmers being taught ways of perceiving their body and performance as they swim to musicians learning music theory to physicists learning the math they'll need to represent the universe - mental representations seem to be the key to gaining these incredible levels of skill. As Steve was finding ways to better take advantage of his limited short term memory so too are masters in all of these areas finding solutions that allow them to break through to the next level of extraordinary. 
\newline

However, all of this should be setting alarm bells off in your head. If it's just a matter of learning mental representations, why on earth aren't we all master chess players and elite swimmers? Clearly we're missing something. To see what it is consider the pipe organ.
\newline

Playing the pipe organ (the kind you find in an old church) is no easy feat. There are usually multiple keyboards which you'll end up playing separately with each of your hands, a pedal board for playing notes with your feet, dozens of stops that control what \textit{kind} of sound each keyboard produces, volume levers, and then of course there's the matter of actually playing itself - reading music, coordinating your fingers, keeping tempo, knowing how to interpret the music, and so on. Clearly if you try to actively \textit{think} about all of these things all at once you'll get overwhelmed. Instead you've got to make each of these components instinctive so that you can use your conscious mind to work on coordinating it all rather than trying to work out all the little details in real time. In other words you've got to work the underlying skills into your intuition.
\newline 

Everyone who's practiced something knows that when you first start, whatever you're doing is \textit{hard}. However if you keep up consistent and regular practice over time, what was hard ends up becoming relatively easy. Why is that? Well it has to do with the notion of homeostasis. All your body wants is for normal life to be... well... easy. So when it suddenly finds that every day you, for some unfathomable reason, find yourself running, it figures out that if this is gonna keep up it had better reconfigure things in order to make the run less of a pain. So it starts building blood vessels and musculature to turn your daily run into just another part of the norm. It shifts your statis state - your homeostasis. 
\newline

Turns out the same thing happens for these mental representations we've been talking about. When you first start using them, your brain is rather clunky about it. Like the first time you did algebra or started reading you have to think really, really hard to get things to work. However once your brain realized this was part of the norm it started shifting the mental representation from your conscious thought into your subconscious thought (which is far more effective and powerful) so that reading and algebra became relatively easy. However this takes time because you first have to establish that this mental representation is a new norm \textit{and then} give your body the time to rewire properly. Going back to our pipe organ example we see that there are potentially dozens of individual skills that have to go through this process before we're able to play the pipe organ with any real skill. So take whatever time is required to embed one skill and then multiply it by the number of skills one will need to embed and you've suddenly got a sense of why we aren't all masters of every field out there - layering the skills required to master something takes a hell of a lot of time!
\newline

We can actually get a general sense of the order of magnitude involved in \textit{just} the embedding process by looking at a skill that has an extremely well developed learning sequence by this point in history - classical violin performance. Given it's skill that's been around for at least a few hundred years, people have been able to take all the guess work and creative mental representation building that Steve was doing out of it. Therefore learning to play the violin is really just a process of embed, embed, embed. 
\newline

Ericsson and two collaborators Ralf Krampe and Clemens Tesch-Romer went to a well known musical academy and selected three groups of violinists - those who were the "best", those who were "better", and those who were "good". They then tried to search for what differentiated these groups from one another. Unsurprisingly, given all we've learned so far, the only predictive component they found was the amount of time people spent practicing - if you practice more you're better at playing the violin. However what's incredible is the sheer amount of time spent practicing. For those in the "best" category, given how much they'd been practicing, it was predicting that by the time they were going to be in their thirties (when folks start really rolling in their careers) they'd have put, on average, somewhere near 25,000 hours of practice in! To put that in perspective, even if you assume they've been practicing since they were 5 years old, that means they've been treating practicing as a part time job (20 hours a week) for their entire life! Embedding skills take a lot of time, energy, and dedication indeed.
\newline

This then brings us to Ericsson's conclusion - what leads to skill in any field is simply \textit{deliberate practice}. He, of course, does a much better job of getting at the specifics (and I'd recommend reading his book \textit{Peak} \cite{ericsson} of which this section is a mere summary) but it more or less comes down to the idea that becoming skilled at something is about two things - (1) not reinventing the wheel, i.e. learning the mental representations that others have already invented in an order that they prescribe and (2) pushing your definition of normal deliberately and consistently in order to truly embed the skills you are learning into your body and brain. However if this is true than it also predicts something that would be rather liberating - that anyone can learn any skill. Note that nowhere in deliberate practice is there any notion of what we might call "talent", all there is is hard work and paying close attention to the coaching others can give you. So is he right, can anyone learn anything? The proof is in the pudding, so let's look at a positively wild example.
\newline

Perfect pitch is something that people will often treat more like a sense than a skill. Most assume that you've either got it or you don't. However in 2014 a Japanese psychologist decided to test this assumption. He gathered together a set of children between the ages of 2 and 6 who didn't have perfect pitch, and over the course of several months taught them how to identify 14 chords by just listening to them. For some children it came more easily than for others, but eventually everyone could identify the 14 chords on their own. Then he tested the kids for perfect pitch. What he found was truly astonishing - every single one of the children had developed perfect pitch and could identify individual notes from sound alone! Something that had always just been considered talent was 100\% learn-able!
\newline

The conclusion is clear (and if you want more evidence, once again, check out Ericsson's book \textit{Peak} \cite{ericsson}) - anyone can follow in the footsteps of the masters if they're willing to put in the time and use deliberate practice. However this leaves us with a question - what happens when there are no longer any footsteps to follow? What if like Steven Faloon you're finding yourself having to invent the solutions that will bring you to the next level? What about creativity? Clearly if we want to under exceptionalism overall we will need to understand the creative aspect as well. So let's turn to that next. 

\section{Creativity}
When we think about creativity we think about novelty - something totally new, a paradigm shift. Strokes of inspiration straight from the well of creative genius. As such, it can end up feeling kind of haphazard, as if you must just mull over things until that stroke of inspiration occurs. Indeed it can end up feeling as if creativity is something given in droves only to a select few. Reinforcing this sense is the fact that most narratives about creativity are almost always dominated by a select few - cubism by Picasso and Braque, computing by Turing, relativity by Einstein. However, if we take a closer look at the achievements of these creative geniuses, patterns begin to emerge. 
\newline

As an example, take Picasso. It's often noted that Cubism was inspired by the geometric and abstract nature of traditional African art, especially masks - something that Picasso took a keen interest in \cite{sabine}. However, Picasso was only able to bring this style into the fold of more mainstream painting because he was one of the most technically proficient painters at the time. In similar fashion Jazz, one of the more influential musical developments in history, arose in New Orleans out of a melding of predominantly white, brass, marching bands and the music and dance traditions of recently emancipated African Americans \cite{jazz}.
\newline

On the STEM side there are examples like this as well. Probability and statistics started out as simple accounting for the purposes of playing cards and dice, but has subsequently found applications in pretty much every field in existence. Signals processing is only possible thanks to a kind of mathematics called Fourier Analysis that is more or less the combination of the epicycles that early astronomers used to explain the cosmos and the principles of calculus that Newton developed \cite{gnu}. 
\newline

Then there's all the inspiration taken from nature. One of the most successful optimization strategies out there - genetic algorithms - came from the flash of insight that if you can create a genome for different solutions you can find the best solution using the same principles backing genetics and macro-evolution in biology. Then there's the fact that down jackets are the result of realizing that some birds don't fly south during the winter, modern windmill fan-blades were inspired by humpback whale fins, and there people are studying ants to figure out how to regulate traffic with autonomous vehicles \cite{biomimicry}.
\newline

All in all, the pattern here should be pretty clear - strokes of insight seem to come from taking the ideas from one field and then finding interesting ways of applying them to another. African masks + european painting becomes Cubism. Tricks from ancient astronomy + calculus becomes Fourier analysis. Bird watching + a need to engineer better clothes becomes down jackets. Creativity is the art of synthesis - finding useful and/or interesting connections between previously unconnected things.
\newline 

Now at this point you're probably feeling like this is a gross oversimplification - and you'd be right! For starters, before the insight can happen the tinder for the spark has to be present. There were plenty of artists around in Picasso's day but he was set apart both due to his exceptional technical fluency and his interest in African art. Then there's the fact that the stroke of insight doesn't come of its own accord - clearly there would've been many people aware of genetics, evolution, and the need for better optimizations - yet, in the end, only a handful of people came up with the idea for genetic algorithms. Then there's the process of sorting - many creative people note the fact that for every good idea you have there will be loads of poor ones, and distinguishing one from the other is no trivial matter \cite{kaufman}. Finally, it's a long way from that initial spark to real achievement - Fourier spent a long time formalizing and proving out his ideas before anyone took them seriously, and you can be sure that folks didn't just copy a whale's fin and stick it directly on a windmill turbine. Beyond the initial stroke of insight there are obviously other very important steps involved. Gather, Spark, Sort, Develop in no particular order (because creativity is a messy business) - these are the steps to creating something new. 
\newline

However, for each of these steps, we can learn from those who've already achieved. We can note that for the cubists and the impressionists (who took key inspiration from Japanese prints \cite{impressionism}) their gathering stage was marked by exploring farther afield than most artists in their day. Einstein is often noted to have spent a lot of time performing thought experiments in his head in order to turn problems this way and thereby facilitate sparks of insight and inspiration. Isaac Asimov (a prolific science fiction author) noted the fact that sometimes it is best to be alone when evaluating and sorting through ideas because many of the initial ideas you'll have will be, in his words, "embarrassing" \cite{kaufman}. But these are just a few examples. In general, there are entire books written on this stuff. From Kaufman and Gregoire's \textit{Wired to Create} \cite{kaufman} to Csikszentmihalyi's \textit{Creativity} \cite{creativity} there are loads of books out there chock full of lessons learned from studying what sets high achieving creative people apart.
\newline

In the light of the last section, all of this should be sounding tremendously familiar. We've got a specific goal - synthesis of previously disparate fields - as well as a whole lot of techniques, advice, and, may I say, mental representations from experts. In other words, while what comes out of the creative process is certainly novel, the process of being creative is not novel at all! Synthesis is a skill, with experts, that can be mastered itself! It's just a very special skill in the fact that regardless of the field you find yourself in, you're going to want to master this one as well because creativity is how you adapt and push boundaries. But as we've seen in the last section, mastering a skill is no mystery rooted in talent alone - it is instead the result of deliberate practice. 

\section{The Need for Organizations}
Alright, at this point we know that the key to exceptional skill and creativity is deliberate practice. However, we also know that deliberate practice for even one skill takes an extraordinary amount of time and, as we just pointed out, one must invest time in practicing creativity too. Remember those violinists who'd spent 20 hours a week for decades to master their craft? That was just for practicing the standardized components. All in all, exceptionalism comes at a very specific cost - and that cost is specialization. Each individual has only so much time and given the massive investment required to master something, there will only be a handful of things you can master in one lifetime.
\newline

However, as anyone knows, specialists cannot operate alone. People who operate in sales need things to sell. Musicians need bands or orchestras and places for those bands or orchestras to perform. Mathematicians only really become "useful" when engineers and scientists put their abstractions to use. A doctor is no good without the hospital staff and admin required to keep hospitals running. In short, specialists only become valuable in the coordinated presence of other specialists.
\newline

As a result, I, as an individual, will only be able to see benefits from my specialization if this coordination exists. And so we can see that creating high performers through deliberate practice is not enough. While it is definitely a necessary condition for taking full advantage of human potential, it is certainly not sufficient. What you also need are \textit{organizations that can coordinate those specialists}. Put another way, without organizations, specialists are nigh useless. But this also means that organizations that do a poor job developing and/or coordinating those specialists are also useless. 
\newline 

I like the analogy of catalysts here. Within your cell float about loads of proteins, carbohydrates, salts, sugars, and all sorts of other useful, specialized, substances. However, on their own these chemicals just form a weird, gooey soup because each of these substances on their own don't really do anything. What turns that soup into life are the catalysts that facilitate bringing the various substances together in useful and meaningful ways. This is what transforms a biochemical sludge into one of the most incredible machines in existence - the living cell. Organizations, then, are catalysts for human specialists. Without them there is no way to take full advantage of the human condition and as a consequence no way to tackle and solve our biggest problems or our most daring dreams. And so, if you wish to tackle something huge, if you to achieve in the highest order possible, you must do two things - first you must engender the most exceptional of specialists and then, with equal importance, you must build the organizations that will catalyze them. For without those organizations there will be no mechanism capable of mustering the human skill and energy required to shift a component of our world.

\nocite{duckworth} \nocite{outliers}

\chapter{The Value of Focus}
The world is a really big place with a lot going on. No matter where you look, opportunity and problems abound. On just the matter of sustainability the United Nations has identified 17 separate goals each of which is enormous in scope - number 15 is just titled "Life on Land" \cite{united}. Likewise the Millenium Ecosystem Assessment has identified 25 different ecosystem services, all with their own host of concerns \cite{mas}. And all you need to do is head over to your news feed of choice to get an overwhelming sense of all there is out there - economic recessions, public safety, the fate of democracy, crises in agriculture, issues with education... the list goes on and on.
\newline

Then there's the fact that the "Devil's in the Details". Take literally any of the above topics and you can subdivide it about a million times over to create thousands of different focuses, each with their own host of questions, issues, and jobs to get done. The world is so interesting and diverse that people can make a career out of studying a very specific people, at a very specific point in time, in an equally specific part of the world! 
\newline

And then there's the fact that each of us is juggling so much in our day to day lives. There's school, families, friends, work, hobbies, commutes, cooking, cleaning, exercise... yet another overwhelming list. 
\newline

All of this is enough to make one feel like there's not enough time to go around! So many world issues that it seems like one should care about, so many things to juggle in daily life, and then even just getting through the details of any one of these things can feel overwhelming. And then, invariably, there are those people who just seem to be doing it all without breaking a sweat. You know the ones I'm talking about - the ones who are part of the local government, while also holding a full time job, while also being involved with their local church, while going out every night with their friends, while also having kids, while also... you get the idea. It leaves a person wondering what their secret is - how are they able to get everything out of life while the rest of seem to struggle with a far smaller TODO list?
\newline

Well, Clifford Nass - a professor at Stanford - wondered the same thing. So he brought together a group of researchers and a group of students to find out. He divided the students into high multi-taskers and low multi-taskers and then set about searching for what they specifically excelled at. But what he found surprised him - the high multi taskers just did worse at everything... Since then, lots of other research has followed this up and shown the same thing - there is no such thing as effective multitasking, the key to success is actually focus. \cite{keller}
\newline

To understand why this is requires that we look at several attributes of how our brains work (and don't work). Each of these examples on their own will make a great deal of obvious sense, but together will lead us to some pretty interesting conclusions about how to be effective as an individual. So let's dive in and start with the idea of attention residue.

\section{Context Switching}

We've all had that feeling - the one where you were just playing a game or watching a show and suddenly you're asked to direct your attention elsewhere, like toward homework or something like that. It's a nasty feeling and leaves one feeling frazzled and not totally focused. Well this is more than just a feeling. Sophie Leroy ran a study in which she gave a series of people some difficult puzzles to solve. Then, for some of them she would interrupt them and give them a new task. For for the others, she let them finish first. For those interrupted they performed worse - indicating that context switching in the middle of something isn't some kind of flip of a switch. Leroy named this phenomena - \textit{attention residue}. 
\newline

This attention residue can last for a really long time. Thanks to Bluma Zeigarnki's research we know that unfinished tasks at the end of the day can stick with you and continue interrupting your thoughts throughout the rest of your evening! This phenomena of leftover "background threads" is called the Zeigarnik effect \cite{newport} and getting rid of it requires taking the time to get your mind in a place where it feels like today's tasks are indeed done. Often times this involves planning out tomorrow in such a way that the tasks unfinished today have guaranteed space tomorrow. And as you might expect this takes time to do. 
\newline

But there's not just overhead involved in getting the last task out of your head, there's also overhead in bringing your faculties online for the next one. For anyone who's brainstormed or experienced writer's block, you know that it's really only once you get going that the going gets good - i.e. it takes time to warm up your brain to whatever you are doing. And getting to that warmed up state of mind can take quite a while. So, not only will you be taking time to get your mind off the last thing, but also taking to time to really get into the groove on the new one. 
\newline

And let's be real, context switching is not fun. Having that distracted feeling, experiencing writers block, or feeling like you're really just not in the groove is not at all enjoyable. So both for efficacy and enjoyment's sake reducing overhead is the way to go. 
\newline

This, of course, means making sure you've got large blocks of time to do things that are challenging. For example, suppose you have a 30 minute window open at work and you're tempted to try and get something done during that time. You're going to be spending several minutes pulling your brain of the last task, several minutes getting into the flow for the upcoming task, only to have a few minutes left to actually do anything well. Which is going to leave even more open threads in your brain to distract you during your day! Definitely not worth doing. 
\newline

Also note another conclusion that falls right out of this - multi tasking isn't real! We only ever really do one thing at a time, and every time we switch tasks we pay a heavy penalty in context switching. So we've got our first principle then - context switching takes a considerable amount of time and effort and represents a kind of overhead that should be minimized by instead working serially in large dedicated blocks of time.
\newline

To get at our next set of principles, we'll need to switch gears and consider the nature of willpower.

\section{Willpower}
There are quite a number of studies that demonstrate, rather nicely, that willpower is not only a resource used to redirect ourselves away from our default behavior, but that it is a limited resource at that. For example, in one study students were asked to memorize either a two or seven digit number after which they were presented with a choice to either have chocolate cake or fruit salad as a snack. The minimal increase in cognitive load required memorize a seven digit number instead of a two digit one resulted in twice as many people taking the cake \cite{keller}. 
\newline

In another study, each participant started in a room with the researchers, a bowl of radishes, and a bowl of cookies. The researchers pretended like this was a study on taste and so instructed the participants to eat only a specific food while the researchers went off to do something else - for some it was cookies, and for others it was radishes. More than one radish assignee nearly ate a cookie! Then the researchers said they needed to pass some time to let the "sensory memory of the food fade" and gave each participant a puzzle to solve. However they neglected to tell the participants that the puzzle was actually impossible. At this point the researchers left and watched the students from behind a two way mirror. The cookie eaters worked for about 19 minutes before giving up. The radish eaters on the other hand only averaged 8 \cite{duhigg}.
\newline

Likewise upon studying over a thousand parole board hearings, a clear pattern was found between the proportion of favorable decisions and how long ago the judges had taken a break. The default behavior for such judgment is to give a negative decision and the number of these steadily increased until break time after which the number of positive decisions suddenly jumped through the roof \cite{keller}! 
\newline 

The point of this is clear - willpower is a limited resource that eventually runs out and is therefore something we have to be careful to use wisely. So, two questions immediately - are there ways to increase the availability of this resource and how do we use it wisely?

\subsection{Using Willpower Effectively}
The answer to the latter question comes in two parts and both require a clear understanding of what willpower is - doing something that isn't your default behavior. As Charles Duhigg expands upon in great depth in his book "The Power of Habit" \cite{duhigg}, our default behaviors are driven by habits. To understand this we must dive into what a habit is. To illustrate this Duhigg presents a very simple experiment that was done on rats. Specifically, rats were placed in a kind of "maze" that had chocolate at one end for the happy rat that could find it. When the rats were first introduced to this they would wander around largely at random and their mental activity would be quite high in the new environment. Eventually they would end up finding the chocolate and the researchers would repeat the whole thing. What was very interesting however was what happened as time went on. Eventually the rats would master the maze and know exactly where to go and during this process of learning their brain activity went from very active to barely active at all and actually changed position in their brain - they had internalized the pattern and could now do it without much thinking. They'd moved the activity from the active decision making part of their brain to the default circuits in their brain. 
\newline

However there was something another strange insight that came along with this discovery. While the brain activity during the maze walk decreased, the activity at the beginning and end did not. Specifically it seemed that the brain was looking for cues to start a default sequence, then fell into it, and then ramped again up again once the task was complete. This Cue, Routine, Reward loop turns out to be how habits always work \cite{duhigg} and is a model for how to embed things into your default behavior. 
\newline

Two things come from this. The first is that the best way to use our willpower is to, instead of constantly fighting our default behaviors  create habits in order to change our defaults to suit our needs. Indeed this has led to the observation by several people that no one is \textit{really} disciplined, there are just those that exercise short bursts of discipline to change their default behavior in this way and those who try to fight their default behavior and eventually fall off the wagon. So when you see someone studying or practicing for hours on end, it's not because they are actively exerting willpower - it's that they've built habits that allow them to get into flow during those activities \cite{keller}. 
\newline

The second conclusion from this is that life has to be built around processes, not discreet tasks or goals. Habits are all about routines and therefore in order to build them you have to turn your life into a series of processes that you can repeat day after day in a largely consistent manner. This is quite a divergence from the kinds of chaos that life can regularly throw at you.
\newline

Which brings us to the second point on how to use willpower effectively. As Cal Newport exposes in his book "Deep Work" \cite{newport}, distractions are the bane of productivity. Even if you're trying to establish flow, having your phone buzzing every two moments, or having coworkers popping over to talk to you at random, or having to constantly check email, leads to distractions that very quickly drain whatever willpower you have. So as just a common sense takeaway, it seems reasonable to divide your time into deep and shallow work so that you can group the distractions together and prevent them from draining your willpower. You can let your default circuit run rampant around messaging and communication at a different time from your deep work - during those deep work times just create your own personal "work bunker" \cite{newport} in order to protect yourself from distractions.
\newline

In summary then, to use our willpower effectively we must organize our days and habits to create alignment between our default behaviors and what we want to get out of work while creating clear distinctions between deep and shallow work. In so doing we prevent our willpower from being drained unnecessarily. 

\subsection{How to Get More}
Willpower often gets compared to a muscle - use to much and it gets tired. But the analogy goes further because like a muscle will power can be trained. Two researchers, Megan Oaten and Ken Cheng, did a series of experiments in which they got people to exercise their willpower in one arena (like going to the gym, or doing better money management, etc) and then showed that it created spontaneous changes in other parts of their lives (like eating better, smoking less, or becoming more productive at work) \cite{duhigg}. Exercise your willpower muscles and you strengthen them. But the reverse is also true - give into distractions frequently and your willpower wanes. Cliff Nass's research has shown that the brains of those who are chronic multitaskers or who are chronically distracted just remain overstimulated all the time and have trouble focusing even when they want to \cite{newport}. In order to keep our willpower muscle healthy we have to keep exercising it. So it's important to set aside time to focus deeply and practice using willpower so that you can push your limits and stay strong.
\newline

Another useful note is that like a muscle, giving your willpower rest gives it time to bounce back a little. In one study two groups were asked to go on a walk before doing a task requiring a lot of concentration. For one group, they went for a walk in the woods, for the other they walked through a busy city center where they had to dodge all manner of people, cars, and the like. The group that went on a lovely walk in the woods did significantly better on the task than those who'd been busy trying to make their way through the city center. The lead author Marc Berdman attributes this to the fact that while walking through the woods we no longer direct our attention, whereas in the city center we have to be on guard and so are sapping our willpower as we go about. Taking time to be idle gives us an opportunity to replenish a little bit \cite{newport}.
\newline

Now one note we should make before coming to any conclusions about "unlimited power" is that like any other muscle there's not only diminishing returns to any of these techniques but there are also limits. For example, in his studies on expert performers Anders Ericsson rarely found people who's time spent truly focused on difficult \textit{new} tasks ever exceeded 4 hours \cite{newport}. So while there are certainly ways to extend the time you can spend draining willpower, be sensitive to your limits. Work beyond that time will, by definition, not be the best you can do.

\subsection{When the Tap Runs Out}
So what to do when your willpower tap runs out and you find yourself struggling to focus or realize you are operating suboptimally due to mental fatigue? Embracing idleness can be extremely powerful. People have run studies that show that sleep (or even naps) can allow your subconscious to solve problems that you'd been struggling with all day \cite{kaufman} or that going for a walk or using some other way to get your mind off things will suddenly give you that eureka moment that you've been looking for \cite{newport}. In general studies have found that your subconscious faculties are very powerful on their own and often times better suited to certain kinds of problem solving. So taking a break doesn't actually mean you've stopped thinking or being creative - it just means you've let a different part of your mind come into play. And from this vantage point it's just plain dumb to burn yourself out working suboptimally when a nice rest and good night's sleep would mean you'll have meaningful breakthroughs you can follow up on with your full faculties the next day.
\newline

When the tap runs out, and perhaps even before then, \textit{rest}.

\subsection{Use Willpower Wisely}
All in all then, we've learned a couple of things about using willpower wisely:
\begin{enumerate}
\item Divide your deep and shallow work intentionally, no point draining your willpower unnecessarily
\item Use willpower to create habits not to fight yourself day in and day out
\item Exercise your willpower to keep it strong (i.e. work in large blocks of focused concentration)
\item Take breaks when you feel yourself getting tired 
\item Embrace idleness and don't overwork yourself
\end{enumerate}

One final note is that willpower is not just for work. It's for anything you really deeply care about and want to give your all. So friends, family, hobbies - they will all need willpower too. Which points out something big - even if you had 10 hours of deep work in you everday, with habits and the benefits of practice involved, you only really have 5 or so hours to dedicate to work each day. And as anyone who's worked on tough problems knows - five hours quickly seems like very little. Unless that is, you have \textit{focus}. If you can dedicate that attention to one or two things, fine. Anything more than that and you'll just perpetually feel like you aren't doing enough. So, once again we come to the conclusion that being effective requires focus. Multitasking just spreads you thin.
\newline

But enough of that, onto the growth mindset!

\section{The Growth Mindset}
We often make decisions on the basis of what's happening in there here and now which... makes sense. The here and now is what we can feel and the future is not a given, so this, current moment is certainly of far greater importance than any single moment to come. The problem in this calculus is that we often forget that while each coming moment may have less valuable than the present, there are \textit{way} more future moments. So thinking about how to build upon all of that time-to-come is one of the best places to find ways of being ever more effective.
\newline

The first and most obvious of these "growth mindset" strategies is education. Everything you can do today is thanks to what you learned yesterday, and so what you learn today will make you better tomorrow. If, once out of school, you just stopped learning and worked as hard as you could you'll seem extremely productive in the short term, but in the long term you'll get outpaced by the version of yourself that decided to dedicate time to education as well. However the reverse is also true, a version of yourself that never ever starts working but just learns gets even less far. So, a balance is necessary. Continuing education is one of the best ways to ensure you just get more and more effective over the long run. 
\newline

Then there is what Keller refers to as the "domino effect" \cite{keller}. I find the best way to illustrate this (given what a long play it is) is to refer to one of my favorite kinds of game - simulated city/civilization building. In these games you normally start out with very, very few resources and people. For example, in one of my favorites as a child you start out with a few huts on the Nile river. At this point in the game your options are quite limited - instead of building pyramids, you're organizing farmland, building new huts, and getting cleaner water sources. But as your people get healthier and their work more efficient, a surplus appears and you can start to use that to create healers, roads, and wagons. This in turn provides for even more surplus which allows you to take even larger actions, which creates more surplus, which generates more action, which... you get the idea. Before you know it you've started construction on your first pyramid. This kind of exponential growth comes when each new increment of work we do both takes advantage of what we've already done, and sets us up to be more efficient in the future. We get buoyed by our own past doings! The reason why Keller refers to this as the domino effect is because a falling domino can knock down another domino that's a fixed multiple of its size. So for example if that multiple is 1.5x then if your first domino is an inch tall, the next domino can be an inch and a half, the one after that 2 and a quarter inches, then 3 and a third, then 5 inches and by the time we get to the tenth domino it is roughly 5 feet tall! Compound growth is an exceptionally powerful thing but it requires that you are in fact compounding something on itself. If instead you just redirect your energy in a different direction from one moment to the next, no compounding is going to happen.
\newline

Now when it comes to getting that compound interest to have a quicker return it's wise to remember the Pareto principle (also known as the 80-20 rule). The idea here is that for any effect, 80\% of it is likely the result of only 20\% of the underlying causes. The classic example of this (and the one that led to Pareto coming up with this principle) is wealth. A small fraction of people control the majority of the world's wealth. In Pareto's day it was roughly 80-20. But as it turns out this rule applies pretty much everywhere. I've come across it while teaching people, building software, or even while trying to get a recipe to work just right - a minimal amount of causes creates the most impact. Another way of looking at this is in terms of diminishing returns. Everyone knows that the more you try to optimize something, the harder and harder it is to make it better. Learning how to go from making a sauce so awful that no one wants to eat it to something palatable - relatively easy. Making a sauce that just comes out above the best in the world? Really, really hard. Minority of causes, majority of results. 
\newline

This principle is very useful because it also means that if we take the time to figure out what that hypothetical 20\% is, we can save ourselves a lot of time - we can get to knocking over that next domino that much faster. If instead we don't take the time, chances are we'll end up dipping quite deeply in the 80\% of causes that create only 20\% of the impact - which definitely doesn't help us work more efficiently. Here then is our third and final principle of this group - take the time to figure out the "keystone" causes of whatever it is you're doing so that you can spend less time working and get just as much result.
\newline

So what's in common between education, the domino effect, and using Pareto's principle to our advantage? It's the fact that if all you cared about was the present, you'd never invest time in any one of these! Education takes away time from high priority work that you might be stressing about. The domino effect requires that you stay focused on building on what you've already done even when your boss turns around and tells you there's something higher priority but relatively unrelated that needs to get dealt with urgently. And to find that golden 20\% requires once again taking time away from work to deeply understand what it is you're doing and what its keystone causes are. For those who apply those principles they will \textit{always} look less productive than their peers in the moment. 
\newline

However, over time they get to reap the rewards of compound interest! And this means that their base level of productivity grows exponentially. So while they may look unproductive in comparison to their present-focused peers in one moment, a few years down the line the two won't even be comparable. \textit{Investing} your time is the most important thing you can do to make yourself more productive - everything else we've talked about pales in comparison because this stuff \textit{raises the baseline}. 
\newline

However, as we mentioned before, compound interest only works if you are actually compounding on the same thing. All of these strategies require that you have focus, and focus that you can maintain over \textit{years}. So far we've noted the importance of depth and focus in work, but here we're talking about depth and focus that remains consistent over long periods of time. Yet, without finding that focus we'll never be able to take conscious, deliberate advantage of compound interest on our own capabilities.

\section{People, then Problems}
The first thing that should be clear from everything we've said thus far, is that multitasking, for the sake of multitasking, is not a virtue. It creates more overhead, it reduces our willpower, prevents us from educating ourselves as deeply or consistently, interrupts the compounding of our own work, and in general just takes the potential we have and smears it out so that we become a kind of "master of none" \cite{keller}. Focus is good, and for making real progress, focus is necessary. 
\newline

However when we look at the fact that the whole point of our "compound interest" is to take us beyond what we can perceive or imagine today, it also becomes clear that our focus is not just about the goals of today - instead it has to encompass our growth over many years to come. How on earth can we find focus that will extend past what we know today? Well, imagine once again that we're playing the civilization building game. At no point are our goals fixed, instead what we are doing each and every moment is monitoring what we have, what's been working well, and how we can take advantage of it to launch ourselves forward. In other words we're not so much focusing on a specific outcome but on understanding what is driving our success today and how we can continue to grow \textit{that}. 
\newline

Similarly when it comes to our considerations of willpower we found that the key is to use our discipline not to fight our default behaviors but to rather use that discipline to engineer and redirect our default behaviors through the formation of habits. But as we've seen, habits are rituals, processes, ways of life. And so once again we see a need to look for common, repeatable patterns in our lives that we can turn into habits. 
\newline

In other words as we go about our work, the key to our long term success is in finding this unifying framework around which we can build habits and upon which we can keep compounding our efforts. We are not just a bag of skills but a moving, developing centroid of processes, skills, resources, connections, capabilities, ideas, and perspectives that have been honed over time.
\newline

Why does this matter? In our wildly varied and oftentimes overwhelming world we typically take the problems or opportunities out there and try to figure out how to fit ourselves to them. You see this all over the place. Charitable operations ask "what can you do?". Your work hands you jobs and expects you to be resourceful enough to figure out how to get it done. We ourselves orient around far off dreams that more often than not have very little to do with who we are today. We see the world as a collection of aspirations, and success as our ability to conform to those things. 
\newline

However now that we know what creates success in the individual, it should be clear that this whole mentality should be flipped upon its head. Success comes not from launching ourselves into whatever problem is at hand but instead in looking for the things that we are best equipped to do, and especially those things that will allow us to build upon and develop ourselves even further. It's not about finding the most important problem you but finding the problem you can be most important to. And to be important to that problem you must prevent yourself from being diluted by the world in the process. 
\newline

So, to bring this back to the beginning - yes, the world is overwhelming and often times frightening, but it is not your responsibility to be involved with it all. The best you can do with the time and energy you've got is to follow the part of the adventure well suited to you and devote your heart and soul to it. 
\newline

The problem now is, if rather than being able to specify what needs doing and driving people towards those tasks, each person is just taking up the things best suited to them, how do we ensure the work that needs doing still gets done? This, is a challenge to organizational development theory. 

\chapter{Cheap Failure}
In the summer prior to my senior year at college I got an internship with a satellite communications company where I, and two fellow interns, attempted to build some machine learning models. I ended up getting hired full time and spent the next two and half years working with that group. They were the best engineers I'd ever worked with and I still attribute a lot of my personal success to what I learned and absorbed by being around all of them. However I eventually ended up leaving for greener pastures specifically because the project we were working on just never really seemed to go anywhere. Indeed shortly after my departure the group re-branded, switched tactics, and more or less left the work we were doing behind. 
\newline

During college (and even before that) I had this notion that if you had the skills and the time to use them - success was just around the corner. Yet here I was, having worked for several years with some of the best engineers I've known and we'd more or less gotten nowhere. For a young professional it was a moment of existential anxiety, but as I thought about it more I started realizing how naive I had been.
\newline

If skill was all that was needed, companies would never fail. Yahoo had incredible talent, yet it was taken down by Google. At the time of writing Google is failing to step up to the Artificial Intelligence plate in its tussle with OpenAI and Microsoft even though the underlying technology (transformers) was created at Google! Facebook still attracts some of the best talent in the world and yet that business seems to be going down the tube quite quickly. The pattern is clear, skill and masterful execution are necessary but not sufficient conditions for success. Which left me wondering what other conditions were needed.
\newline

In the course of my reading and subsequent experience I've come to the following conclusion - the key to success is making failure as cheap as possible. The reason for this is that failure is inevitable, curve-balls will come your way, things you thought you knew well will surprise you, and so failure, over time, just becomes a familiar face. However, failure itself doesn't mean the end of success, so long as you can pick yourself up, learn your lesson, and move on, you'll eventually make it. The problem comes when failure dashes you to the ground in such a way that you can't get back up - maybe you're out of resources, maybe you're not reading the writing on the wall, maybe you've eroded the trust of those you're working with - it's the magnitude of the consequences, not the failure itself that's the problem. So to find success, one simply needs to make failure cheap.
\newline

The easiest way to do this is to recognize that imagination is cheap \cite{macmillan}. If you discover a mistake in a design that's in your brain, you can just go right ahead and fix the problem - mental mistakes just cost time. So the more you can work things out with pen and paper, the less likely you're going to make the mistake in the real world and suddenly have money, resources, trust, and the like at risk. Proper (and thorough) design thinking is the easiest and most convenient way of making failure cheap. However it doesn't cover everything.
\newline

Your brain and the physical world are two distinct things, and while you can get them into pretty close alignment the fact of the matter is that there's no proof like the pudding. You can only answer questions in your brain that you know you need to ask - however there are a whole suite of questions that you simply won't realize exist. The answers to these questions are unknown unknowns and the only way you find out about them is by making mistakes in the real, physical world where consequences can be nasty. Therefore figuring out how to find these kinds of things early (and thereby minimize their overall impact) is also super important - we'll call this line of "failure cost reduction" iterative execution.
\newline

In my experience these two strategies - iterative execution and design thinking - \textit{are} how to make failure cheap. So, if we want to understand how to turn our skills into success we'll need into what makes these two things tick. Let's begin with systems thinking.

\section{Think in Systems}
It is one of the supreme pleasures of childhood to build castles in the sand. Why they are called castles, I don't really know, given they always morphed into sand-holes or sand-mountains but it was nonetheless a joy to be out on the beach with a bucket and shovel in hand reshaping the topography. 
\newline

However, it was always a bitter sweet ordeal because one knew that within a few hours all evidence of your toil would be wiped away. The topography of the beach was not controlled by you but by forces of waves, wind, and water. The grand processes of the natural world were the ones that ended up having the final say.
\newline

Life, in many ways, is no different. The politicians you vote for, the price of the food you buy, the nature of the education you receive are all the results of underlying systems and processes. And so, when we go about trying to make a difference in the world (and reducing the cost of any failure along the way) it's important to we make sure we don't end up building castles in the sand. 
\newline

We have a tendency, as a species, to gravitate toward dealing with symptoms. Angry masses depose their leaders without changing the underlying corrupt system. Students cram for tests again and again instead of altering their long term study habits. Doctors prescribe all manners of medicine without a lot of focus on the behaviors that led to the issues in the first place. But if we want to be able to create lasting, consistent change we must move beyond this sand castle building - we must look to the underlying systems that led to where we are and make changes to them. In other words, we must think in terms of systems.
\newline

A great place to start understanding how to do so is Meadows' book "Thinking in Systems" \cite{meadows}. While it would be impossible for me to recap everything in that book, there is one very important conclusion that I wish to highlight here. In a similar vein to our "symptoms first thinking" we also have a tendency to try and blame the ills of the world on poor thinking or just straight up evil. As a result it seems like an obvious conclusion that if we could just put better people in the right places all would be well. Yet as Meadows points out, once you start understanding the non-linearity that is inherent in systems it becomes clear that perfectly rational, well meaning, good people can, together in a system, generate some pretty awful and seemingly irrational behaviors. 
\newline 

As an example let's take a case out of Meadows' systems zoo - a car dealer \cite{meadows}. In this situation the car dealer has two interactions - one with the customers and one with the car manufacturers. However, while this situation seems only trivially complicated, it quickly illustrates what can go wrong with systems. What's important in this example is that the car dealer is having to balance out how quickly she buys cars from the manufacturer with how quickly she sells them.
\newline

Now, in this example, as with the real world, there are delays involved. It takes time to place the orders to the car manufacturer and it takes time to get a good sense of how customer behavior has changed. As Meadows demonstrates, these kinds of delays result in oscillations in the stock of cars at the dealership at anytime. Basically, what happens is that if sales increase slightly, given it takes time to notice the shift, a drop in inventory will result. This drop in inventory ends up pushing the car dealer to put in a large order to the manufacturer. However, because this order needs to compensate not only for the loss in inventory but also the increased rate of sales, at some point the inventory overshoots the sales alone. But it takes time to correct this with the manufacturer and by that point the dealer will be trying to correct downward in order to deal with the excess of stock. This then brings us back to a lack of stock and the cycle continues.
\newline

This however is where things get interesting. The obvious thing to do here is to remove the delays as they seem to be the source of the problem. However there is really only one delay the car dealer has any control over - her own response time to customer sales. But as Meadows demonstrates by actually simulating what happens with a shorted delay, the oscillations only get worse! More or less, the quicker response also results in more dramatic and therefore over-compensatory responses. As it turns out the way to dampen the oscillations is not to reduce the perception delay but to increase it so that it better matches the delay from the car manufacturer.
\newline

So even in a single system with one actor and two interactions, that person can take the action that seems like the obvious good solution and just make things far worse! Our dealer was having issues due to delays but it turned out that the solution was actually making one of the delays longer! Rational actors in complex systems do not necessarily produce rational results. 
\newline

What to make of all of this? Well remember that, you yourself are one of the rational actors in this system and so you too can make the very same mistake our car dealer did. Indeed the only way to avoid their mistake is to recognize and understand the non-linearity present in the system and come up with solutions cognizant of that. So we've yet another reason why systems thinking is exceptionally important to improving our chances of success.
\newline

But there's another lesson here, and one that becomes apparent when we think about how multi-agent systems \textit{select} for certain kinds of behavior or people - even when those systems have the best of intentions.
\newline

The most obvious example of this is the political system in the United States. Politicians here are reelected every few years. This makes good sense if you're trying to provide a quick feedback opportunity for your voters. However it also leads to politics that has a hard time dealing with long term issues as politicians are motivated by their immediate next upcoming election. Another example comes from the financial world. Given the risk involved in lending money, it only makes sense to receive some kind of return on investment. Yet this results in a self reinforcing feedback loop that inexorably drives money into fewer and fewer hands. The principle here is clear, systems with feedback loops (which as Meadows points out is pretty much all systems) have the effect of actually changing or selecting for the behavior of the actors!
\newline

So rather than playing the blame game and wishing for better people or better behavior, a more constructive approach is to figure out what needs to be added to the system to select for the kinds of behavior \textit{that would be helpful}. Like with our sandcastles analogy we are thereby changing what leads to behavior instead of trying to patch the current behavior (just to have it wiped out again in the future). 
\newline

All in all then it should be apparent how important it is to think about things as systems and specifically what thinking in systems really means:

\begin{enumerate}
\item Looking to change the sources not the symptoms
\item Recognizing the importance of non-linearities
\item Understanding that the behavior of actors is a property of the system itself
\end{enumerate}

To make real lasting change we have to look beyond the symptoms and into the wild, non-linear systems underlying the change we want. In other words, we should ensure that when we are make designs those designs are systemic.  

\section{Be Scientific}
At the basis of all the designing, executing, and, frankly, imagining that you'll be doing is an underlying model of the world sitting in the brains belonging to yourself and those you end up working with. When you make assertions about how things work, what value you can provide, and the systems you integrate with, all of this is founded upon what you know about the world - or rather what you \textit{think} you know. 
\newline

Two issues face us here - comprehension and truthfulness. Gaps in our knowledge act more or less as blind spots that we can either resolve before we get to deep into the work or have resolved for us later when we blunder into our blind spot. But even worse than this is basing decisions on knowledge we think we have but which is actually inaccurate. So one of the primary focuses of our "design process" has to be acquiring and testing relevant knowledge. This brings us straight into the realm of philosophy because in order to understand how to be accurate we must ask what "truthfullness" is in the first place.

\subsection{Truthiness}
Any introduction to logic and critical thinking will point out two ways of reasoning about the world - deduction and induction. Deduction, from a logical thinking standpoint, is definitely the nicer of the two. Deduction entails taking a series of premises (which you hold to be true) and then using formal logic to create a proof of some conclusion. What's nice about deduction is the fact that it leads to conclusions that are not likely or probable, but absolutely and fundamentally true - given, of course, your premises are correct. And that's the Achilles heel of deduction - it is absolutely dependent on the premises being true themselves. Likewise, deduction cannot create its own premises because that would just require a series of different premises whose truth would then be in question. This is why ancient Greek philosophers made such a stink about "self evident" premises - premises so obvious they don't require proof. 
\newline

At first this seems like a reasonable way out. For example, Euclid proposed a set of very basic premises (like "things which are equal to the same thing are also equal to one another") and came up with the geometry you use in high school and which for thousands of years was considered \textit{the} geometry of our world. And yet, thanks to Einstein and Riemann we now know that some of his self evident premises are not actually true at all. Deduction can show things follow from their premises, but it can never show that something is truthfully real.
\newline

Alright, so what about induction? Induction is what's behind the fact that you believe the sun will rise again tomorrow. The basis of induction is the assumption that given we see a pattern many times, it's likely you'll see that pattern continue into the future. Seems pretty reasonable, right? Unfortunately, as many philosophers have pointed out (with Hume being the most notable) induction can in no way \textit{prove} anything \cite{lipton}.
\newline

The first reason for this is that there's no proof for induction that isn't itself an inductive argument \cite{lipton}. And circular arguments are no good. For example, suppose I assumed the opposite of induction as an argument - i.e. that if I observe a pattern I expect it to not work out the same way the next time. Then the fact that so far the pattern of normal induction has held I'm allowed to believe that in the future my twisted (radical) version of induction will hold. Circular arguments let you prove pretty much anything, so the fact that there's no argument for induction that isn't itself circular doesn't bode well.
\newline

However, we could suggest that maybe we just take induction at face value and use it without trying to build any kind of proof. Couldn't it still be useful? The problem is that induction at face value is demonstrably very dangerous. A classic example uses the notion of a contrapositive. In formal logic if I have the statement - all ravens are black - the contrapositive statement - all things that aren't black aren't ravens - is taken as logically equivalent. In other words, if I can prove the latter I prove the former. However from the point of view of an inductive argument all I need to do is go outside and collect green leaves to gather evidence that non-black things aren't ravens \cite{lipton}. However this is clearly absurd, and so straight induction is not enough. 
\newline

Clearly need something else. So what's the right way of creating arguments that can allow us to prove things? Loads of philosophical and mathematical literature will quickly convince you that the answer is no one knows for sure \cite{lipton} \cite{tkuhn}. However we can't throw up our hands and give up because we need to get at the truth or something like it somehow! So what are we to do? Well we can take a leaf out of our own book. Remember with design thinking we aren't trying to avoid failure altogether, just reduce its impact. Similarly, here rather than pursuing truth we could look instead to reducing error. Put another way, what if instead of trying to prove our mental model right, we tried to look for how it's wrong. Then over time, as it became harder and harder to poke holes in our mental model we could, understandably, trust it more and more! So let's take a look at how we might do this. 

\subsection{Stress Testing}
The first problem we come across when thinking about stress testing our model is figuring out what our model actually is. For most of us we make assertions automatically using the mental models that have become embedded in our subconscious. This is useful because our subconscious is very powerful \cite{newport} but it also means it can be pretty unclear what's actually happening behind the scenes. Thankfully there's a pretty straight forward way to start teasing out the details of our mental models - and that's in making predictions. 
\newline

The idea is pretty straight forward, for some particular subject or event that you want some certainty around, start by thinking of all the measurables you could create from it. In doing this it is extremely helpful to look for the ways others have measured or represented data in the past in order to get ideas for the kinds of observations you could make. For example if we're talking about the physics of the sky, weather, temperature, color, consistency, wind speed, etc. could all be measureables. In other words, start by being very \textit{observant}. The more ways you look at the data the higher the odds you find some illuminating surprises. But surprises only occur when you yourself have \textit{expectations}. So in each of these cases come up with a prediction (as well as an explanation for that prediction) and then see whether you were right. One of three things will happen:

\begin{enumerate}
\item Your expectation and reality line up
\item You realize you don't have any expectations
\item Your expectation and reality don't line up
\end{enumerate} 

In the first case there's nothing to, your mental model worked pretty well. In the second it means you've found a spot where your mental model is lacking comprehension (it's up to you to determine if that lack of comprehension matters). And in the third case it means that you've found an error in your mental model that needs to be dealt with. In general through this observant, fact-foil \cite{lipton} approach you'll quickly find the boundaries and specifics of your current mental model as well as any obvious errors.
\newline 

It's worth noting here how empirical you have to become for all of this to be possible. You must have some basis to make predictions and to test them you need to run experiments and actually collect data. So if you find yourself in a position where you realize you have no data behind the assumptions you're running on something is clearly wrong.
\newline

Now, at some point you may find that your mental model and what you're observing line up pretty well. At this point (and honestly before it) you can deploy another mechanism for rooting out errors - learning from the mistakes of others. This too is a straight forward process, all you need to do is find folks who've been making predictions like your own and see where it is they've been found to go wrong. Then push your model into those areas as well and see how you fare. For example if you're running surveys it's good idea to know what the various kinds of survey bias are that others have run into in the past.  
\newline

And speaking of learning from others it's just as important to seek alternate perspectives to your own. We usually do this when we're learning about something for the first time and are trying to absorb a matured perspective, but even as we become experts, finding ways in which people are thinking differently from us opens the doors to understanding where our models might be wrong (or not comprehensive enough). In general, like with the fact-foil, a compare and contrast approach is usually extremely productive.
\newline

An important point when looking for the perspectives of others it to look for mental models that have clear internal mechanisms. The easiest way to illustrate what I mean by this comes from an example in cooking. If you open any odd cookbook you will likely find recipes telling you how to, say, pan fry chicken. However many recipes tell you how this works from a "steps" approach - do X get Y. This is fine until you do X and get something that isn't Y at which point you're totally lost. However if you find a recipe that explains the \textit{mechanisms} of what's going on (for example why brining chicken helps prevent drying out the meat or why oil helps create a crispy skin) then you have the knowledge to \textit{modify} the steps to meet your specific conditions. 
\newline

In general understanding \textit{why} something is the way it is is far more valuable than simply just knowing \textit{that} it is a certain way - it allows you to see where things could go wrong, fix issues that do arise, and in general just gives you a much deeper and more thorough understanding. So when you're building mental models and learning from the models of others, search for the mechanisms (or physics) of how it all works and why.
\newline

Finally it's important to avoid the most (in my mind) nefarious error in model building - over fitting. Over fitting occurs when you build a model that can explain everything you've already seen but has no predictive power. In general given the flexibility in model building you always have enough freedom to build some rube-goldberg machine that through its sheer complexity is able to capture every bit of variance you've found so far. However these are not useful models because they never let us predict anything new (and therefore rarely if ever line up with the truth). To protect yourself from this its important that every time you make modifications to your model you have plans on how to test the modifications on something new that you haven't seen before, otherwise you'll always have success in modifying your model but no success in using it in the future. 
\newline

Alright, there are obviously loads more tips and tricks you can use to seek out and destroy errors in your thinking but between these four - observant fact-foil, learning from the mistakes of others, seeking alternate perspectives, and avoiding over fitting - I suspect you'll naturally come across the others. Indeed the combination of these four, and the fact that they specifically drive exploration and expansion of your mental models means that they also drive comprehension. Removing error and gaining comprehension typically go hand in hand. 




\subsection{Summing it Up}
In summary, to get our knowledge into shape and make sure it's comprehensive enough we should actively search for and root out the errors in our ways of thinking. This however is a very unnatural thing to do - indeed one might say that constantly just looking for where you're wrong could get pretty depressing. As such it's extremely important to the success of this kind of whack-a-mole thinking that you develop a mindset where finding and resolving new errors is rewarding and fun, otherwise you'll eventually just lose patience and move onto something else. This is also the kind of exercise that can quickly get extremely overwhelming - so make a habit of documenting things clearly and just in general writing down and organizing what you find - otherwise you'll quickly feel like you're losing your mind. But, if you are able to setup consistent habits that work to drive out the errors in your thinking, then you'll find yourself day by day becoming more confident in what you know and the boundaries of that knowledge.

\section{Create Together}
Alright, so you've got a grand idea. Maybe you've got a notion for some social good you can provide, maybe you've got an idea for an awesome app, or maybe you just want to provide a really neat and novel experience. Regardless, you've got a value proposition and a set of people who you think stand to benefit \cite{valueprop}. 
\newline

Now if you're like me (and most people I've bumped into and read about \cite{patton}) your natural inclination will be to assume that this brilliant idea you have is rock solid and anyone with half a brain would recognize just how cool it is too. However, as you might expect at this point, life is never so simple. 
\newline

For example, the idea behind Zambia Feeds \cite{macmillan} was to provide low cost chicken feed in order to kick start a market of small scale poultry farmers thereby providing competition to the incumbent (and indifferent) poultry providers and also provide a means of living for many subsistence farmers. Seems simple - provide the feed at a low cost, create an industry. However, upon thinking about how this would work from the beneficiary's perspective it becomes clear that there are a \textit{lot} of other things that have to fall into place for the cheaper feed to even matter! People need education on how to raise chickens, need to know where to purchase chicks, have to be able to construct the appropriate fencing and hen houses, need vaccination services for the animals, and so on. Without all of these things the value add just doesn't matter because the value is couched in additional required context. 
\newline

But beyond just additional context there's also the matter of what \textit{specifically} creates value in the beneficiary's mind. For example when people first started trying to market toothpaste, they found that the health benefits alone weren't enough to do the trick \cite{duhigg}. People knew, in theory, that it was good for their dental health and yet they just wouldn't really latch on. Instead what really launched toothpaste was the tingling feeling you get from specific oils companies add to it. Turns out that people need a sign that their mouth is clean to build the habits that have made toothpaste such a central part of our culture today. 
\newline

And then of course, sometimes the value proposition can just be dead wrong. I myself have personal experience with this - building things that in the end no one ends up wanting even though I was dead convinced it was going to be the best thing since sliced bread.
\newline

The point of all of this is that developing a value proposition is not straight forward and one needs to work with and develop a deep shared understanding with your users or beneficiaries to know you're really on the right track and have all your ducks in a row.
\newline

But users are not the only people you'll be working with during this whole process. Once you know you want to build you'll almost certainly be working with other people to make it happen. I'm talking about vendors, sales folks, developers, and so on. And there's a lot of potential for miscommunication here too. 
\newline

A classic example is the failure of "requirements" \cite{patton}. In his book "User Story Mapping", Patton talks about the fact that sending requirements around through mediums like email or chat is a lot like playing telephone - each person listens, interprets it their own way, and then sends on what they thought they heard so that the end result is something bearing almost no resemblance to the original directive. But, as Patton also points out, even if the requirements got through unscathed by this grown up version of telephone, they rarely do the trick because any document that we create is incomplete. He likens it to a picture you took on a vacation. Look at it yourself and you see so much context, such great times, so many stories. When someone else looks at it, they just see a nice picture with some people smiling. To get everyone on the same page requires actively pursuing shared understanding, not just passing on requirements \cite{patton}.
\newline

Another problem that can arise is building the right thing and then realizing it's not what you wanted at all. For example, you might find that things end up being more costly or more complicated that you expected. Or perhaps you end up asking for something that's nigh impossible and therefore delicate and hard to maintain. A conversation with your developers would quickly elucidate these kinds of things, as they have a better understanding of the costs and feasibility of the details than you will. Yet again a place to build shared understanding.
\newline

Okay, so you've got a value prop and you've got a way to build it. But with what money? How will you keep it running and make it sustainable in the long run (assuming that's a goal of course)? This brings us to our third group - the folks who pay for it. Rarely are these people and your beneficiaries the same - even if you're selling something directly it's likely that you also care about folks like shareholders and other investors. If you're working towards some social good it's likely that you're being funded by groups like the local government, NGO's, charities, or some combination thereof. And many applications end up making most of their money through side channels that fall out of their main value proposition. Google, a search engine, makes their money through the advertising they can do while you're searching the web (and they're extremely good at it 'cause they know what you've been looking at). 
\newline

It should go without saying that blindly plotting a path forward without figuring out what's going to be required to keep the lights on is a bad idea. And so here is yet another group with whom you'll need to build shared understanding \cite{macmillan}. 
\newline

Given these examples it should be apparent now that part of building this shared understanding is taking advantage of the expertise of those you'll be working with. Your users are the best source of information on what is valuable to them and what fits into their lives, your developers best understand the details of how to get things done, and your investors and benefactors know best how to navigate politics and keep your value proposition up and running. So it should come as no surprise that our fourth group of people who you'll want to bring into the fold are those who've built things like what you're trying to build. What better place to get wisdom and to avoid silly mistakes than to team up with people who've already run the gamut \cite{macmillan}! It's always best to stand on the shoulders of giants. 
\newline

In sum then we've seen that we have four groups of people that we need to be building shared understanding with:
\begin{enumerate}
\item The beneficiaries - those who receive the value
\item The developers - those who build and maintain the value
\item The benefactors - those who resource the value
\item The advisers - those who's been in your shoes before
\end{enumerate}  
Each of these groups hold the expertise and knowledge to flesh out, refine, test, and create a different facet of your idea and without all of them you won't get a complete picture. So now the obvious question is - how does one create shared understanding?

\subsection{Shared Understanding}
There are a \textit{lot} of books on this subject. Patton's book \cite{patton}, Value Proposition Design \cite{valueprop}, and the Social Entrepreneur's Playbook \cite{macmillan} are just a few examples. Part of the reason there are so many different resources out there is that the specific tips and tricks depend on what you are building shared understanding around and with whom. So I'm certainly not going to try to get into all that depth here. However, I have found in my reading, practicing, and thinking that there are a few general themes that are certainly worth mentioning and run across the board:
\begin{enumerate}
\item Don't just talk, document
\item Be precise
\item Be relevant
\item Create together
\end{enumerate}

The first - don't just talk, document - gets to the idea that it's very easy to hold a conversation with someone and never actually end up talking about the same thing. It's only once you start drawing out the details, writing definitions down, seeing the whole for its parts, and just generally getting past the five or six things you can hold in short term memory at once \cite{ericsson} that you start to see the differences between your understanding of things and someone else's. Furthermore as you go between different groups and try to create shared understanding around the same thing in different contexts you don't just want to depend on your memory otherwise you'll just end up playing telephone. Documents (with lots of illustrations and definitions) are how we can capture, record, and refine our present understanding of things with others. However, make sure you don't fall into the requirements trap - documents are only complete to those who've developed a shared understanding around them through actively brainstorming and conversing - they are never a complete form of communication unto themselves \cite{patton}.
\newline

The second principle - be precise - digs into these ideas further. The devils are in the details and you'll never find them if you don't get into those details. So find ways of being as precise as you can (without becoming inaccurate) when having conversations with others so that you can sort out not just the big picture with them, but the details as well. It's the difference between asking someone if they see the same bird and whether they see the same robin - in the first case you might both agree you're looking at the same bird when one of you is looking at a crow. 
\newline

Being precise is great, but it can also lead to unnecessary work if you're not asking questions relevant to person you're talking to. Remember, each person brings a specific kind of perspective and expertise and that's what you want to tap into. You don't want to ask your users how NGO funding works (unless your user is the NGO) and you don't want to ask your investors how to setup a database. Think carefully about what perspective a specific conversation will bring and stay focused on that.
\newline

Finally, building shared understanding is a \textit{two way street}. You should not just be passing off requirements to developers. You should not just be taking requirements from benefactors. And you certainly shouldn't just be forcing a "value proposition" down users' throats. Every one of these interactions should be collaborative and take advantage of all the creativity, passion, energy, and expertise available in each context. In other words you should be \textit{creating with} everyone you come in contact with. You and your developers should come up with the designs and implementations \textit{together}. You and your beneficiaries should develop the value proposition \textit{together}. You and your benefactors should develop a business plan \textit{together}. And this should be happening right from the start, don't wait until you have a polished version of things to bring your ideas forward and realize you have no idea what you're talking about. Create together, period.
\newline

So in sum, document like crazy, be precise and relevant, and if you ever find yourself doing things alone, stop, go find the folks you should be working with, and create together. In so doing you'll ensure that what you build not only takes advantage of far more expertise than you yourself possess but will actually matter to folks beyond yourself (which is the goal isn't it?).

\section{Be an Artist}
At this point if you've gone and started creating questions around your mental models, started roughing out the systems that you'll be interacting with, hit the books and found questions to help drive precise, shared understanding, and just generally started listing out the questions you need to answer and the things you need to create and design you'll have a list that is entirely overwhelming. So, just as important as asking all of these questions is understanding how to prioritize and organize the work. And for that we'll pull a leaf out of the artist's book \cite{patton}.
\newline

If you've ever watched someone draw a portrait, you'll know it happens in steps. First, the artist will begin by drawing ovals and rectangles and squares just to outline and orient basic things like the position of the face and any general background features. Maybe there'll be some lines thrown in to indicate the middle of the face or where the eyes will be, but at this point whoever is drawing is just trying to orient things and make sure proportions are correct. If they find something they don't like, with a quick use of their eraser they can get rid of the offending line and redraw it in a more befitting manner. 
\newline

Once orientations are looking good they'll move onto using simple shapes once again to start placing eyes, ears, the nose, mouth, etc. At this point all of these things are still largely unrecognizable but for their positions - blocks and ovals and triangles are still all that's in play. But once again this allows for simple and easy modification as the artist works out the details of perspective, scale, and placement.
\newline

Next the artist may start to refine the contours, make the overall silhouette actually start to look like a human being. Sharpen the chin a bit, turn the oval eyes into recognizable outlines, put the appropriate curvature into the nose and ears. 
\newline

Then as they keep working they'll start adding in general senses of color and shading, put in specific features, and then end off with things like freckles and highlights in hair. Honestly if you've never watched a timelapse of someone drawing go find one - it's awesome.
\newline

There's a lot of good in this approach and to see it, consider what would happen if instead we drew portraits like a printer does - from top to bottom in all of its detail all at once. If we realized we didn't like the orientation of the face or that the perspective was off, we'd have to erase \textit{so much} detail just to make that correction. And we would have only realized this after many laborious hours putting detail into hair and foreheads! We'd be learning slowly, and correcting expensively - a perfect recipe for expensive failure. 
\newline

Instead, by starting with the foundations and working up to detail the artist can confidently and easily modify precisely whatever level they are working on at the moment without having to worry about getting rid of loads of extra stuff they never had reason to add in the first place. Rather than erasing a full face, they just remove and replace a simple contour to change the position of the face. Working one level of abstraction or detail at a time makes for quick learning and cheap correction.
\newline

But note, it's not good enough to just iterate on something, you \textit{have} to iterate on the whole. If I were to draw an ear one level of detail at a time and only then move onto the other ear, and then the nose, and so on, I'd end up back at the printer problem where I've put in loads of work before I realize the face is just not the right size or the proportions are off. We want to be learning not just about the components, but how the components interact with one another to form the whole. So, you have to iterate on the whole at each level of abstraction or detail.
\newline

We can now see that we have an answer to our prioritization/organization problem - you should only worry about whatever's required to deal with your current level of detail or abstraction. If there's something that's come on your radar that's going to be a level or more of detail up from where you are, log it away and don't worry about it for now - acting on it would be like filling in detail on someone's hair before you've got the face positioned. In other words we now have a clear, self directing strategy for how to choose the specific kinds of things we're going to focus on at any one time - and it's guided by whatever step we are in drawing our "portrait". 
\newline

In sum then, the way to prioritize work in such a way as to minimize the cost of failure is to be like an artist and iterate through levels of detail one step at a time. In doing so you'll learn much more quickly and be able to make modifications simply and cheaply in the knowledge that your foundations are solid. Iteration is the pattern to execute on.

\section{Be Frugal}
As I noted before, doing as much of this stuff with pen, paper, and your imagination is one of the best ways to reduce the costs of failure. It's one thing to notice a fundamental in flaw in something that's going to require you to blow the whole thing away when it's still sitting in your head. It's quite another to have to tell a whole bunch of developers and investors that you've gone down the wrong path. Imagination is free, development is not - so use your head as long as you can.
\newline

However, at some point you're going to need feedback from the real world. This is true for two reasons. First, you'll end up reaching the boundaries of what you know and what you have data for and you'll have to start building things and running experiments in order to fill those gaps in. The other reason is far more nefarious - unknown unknowns. So far we've be largely talking about known unknowns, things that either you or your advisory panel know to watch out for, gaps you know are there and need to fill in. But, no matter the project you will always come across unknown unknowns once you actually start building things. In other words, no design you ever make will actually be perfectly complete, and you'll do learning along the way as you build things out. 
\newline

So does this mean that all of the design we've been doing in our head is for naught? Not at all! It's still important to minimize the learning you're doing to unknown unknowns, it's far worse to fail because of something you \textit{knew} you needed to cover and just didn't. All this means is that the principles we've been talking about thus far in terms of building good models, being iterative in how we go about things, thinking in systems, and so on still applies even as we get into the execution and experimentation stage. However one major thing has changed - rather than the cost being some paper, graphite, and probably a little pride, resources and money are now on the line. As a result it's in our best interest to figure out how to learn and build with as little cost as possible and there are, broadly speaking \cite{macmillan} \cite{patton} three ways to reduce costs. 
\newline

The first and most obvious is just to be aggressively frugal. For example if you're trying to understand whether people will really engage with a specific UI design you could go ahead and build the UI into an app and then test it on people. But this would really expensive. Alternatively you could use something like Figma (a mock-up building tool) to build a mock up and use that to run your tests. Or, perhaps even better, you could just sit down with users and, with markers and whiteboard in hand, design a UI they like with them in the room. Just being conscientious about being aggressively frugal goes a long way to making things (including mistakes) cheaper.
\newline

The second is sort of the reverse of this, instead of minimizing the cost of what you're going to do you break up and minimize the goal. For example, if you're building an application that you envision having ten features you could ask yourself whether five would provide value, or two, or one, or even a half. That half may not be super viable as a product, but it might allow you to learn something essential quickly (remember learning is as much a value add). Now the point here is not that you're throwing out the rest of the features - you can certainly get to them after the first few are successful - it's just that you're breaking your problem into more bite sized, cheaper to implement pieces so that you reduce the impact of getting something wrong. 
\newline

Finally, the third principle is building with a mind toward reuse. If you can build APIs or databases or whatever else in a manner that allows you to build other things more easily (and using the work you've already done) this dramatically reduces costs as well. Build with a mind to reuse and you'll build less overall.
\newline

In summary, at some point you're going to have to move to the execution stage in order to start building things and receive feedback from the real world. When this happens pretty much everything stays the same - the same principles apply as before. Create with people, iterate, be scientific, remember to work towards causes not relief of symptoms. However, what does change is the cost of getting answers, because now you'll be dealing with real resources rather than just time, pen, paper, and imagination. So, to that end, you can use the principles of frugality, focus, and a mind to reuse to ensure that you get feedback quickly and as cheaply as possible. 

\section{Capture it All}
Alright, at this point, we've gone through \textit{a lot} of different ideas. But what does all of this look like in reality? How do we keep track of it all? And while we keep talking about the "devil in the details" we've really just been talking in broad strokes. To my mind this all comes together in the form of a kind of "collection". And that collection is a document in which all of this questioning, crafting, and collaboration can find a place.
\newline

The collection begins where it should - by capturing each of the groups of people involved. There's a section for the beneficiaries, who they are, the value we intend to bring them, and how one goes about working directly with them. Likewise there are sections for the benefactors, developers, and advisers as well, each with their own details and processes relevant to their perspective and impact on the whole. The reason for starting with here is that the rest of the collection needs to be a collective effort and collective efforts require people.
\newline

Second comes diagrams and documentation detailing the system(s) we intend to change and the specifics on how we intend to change it. This is \textit{the thing} that will be created and everything else, the iterations, mental models, experiments, and cost assessments will be a means to this end. But it's important to remember that this system is beholden to those we are working with and as we learn and create will constantly change - the design is a living document. As a living document there will always be gaps, questions, and other things to be refined. In order to not overwhelm ourselves with these it's important to start creating a queue of refinements to be made, ranked by priority (driven by our iteration cycles), indicating who needs to be involved in working it out, and eventually overwritten by the conversations, conclusions, and caveats that drive the eventual conclusions. This section is not just a guide to the system but to its development and evolution as well.
\newline

Next in line is where we pull in the science. This, in my mind, divides into three parts. The first is a summary of what we think we know and the theory and mechanisms inherent in it today. Bringing together data into a coherent picture is not only how we go about understanding something for ourselves but is also how we end up teaching others. But alongside any theories or ideas are questions. The unanswered questions form the second part of our "science" section. These should each have what we want to learn, why we care, how we intend to figure it out, and, most importantly, the priority of getting an answer - remember detail only matters when your level of iteration is up for it. Finally there are the answered questions and all of the attendant data. You did work to answer these questions and what a waste it would be to lose track of what you did and what you found.
\newline

As our system design matures we're going to start moving from the world of imagination, conversation, pen, and paper and into the world of resources and costs - the world of real execution. At this point two things become paramount - breaking things up into development iterations and then minimizing the costs of each of those iterations. Just as our system design was a cumulative effort among all of our collaborators, so too should this be a common effort. Experiments can only be run with users, you need developers to build things, and benefactors to fund things. So working together on these minimized iterations and time lines is essential.
\newline

Okay, at this point we've captured creating with others, being scientific, thinking in systems, iterating, and being frugal, so are we done? Not at all. What we've dealt with so far are the guideposts and ideas that more or less apply across the board regardless of what you're doing. But as you go along and work with people, think about things, build stuff out, and make mistakes you're going to come across innumerable considerations, questions, processes, and the like that are helpful to your \textit{specific} situation. One way to think of it is like this - for every mistake you or someone else has made in your field there is a question or a process that, in hindsight, could've saved you from that mistake. Capturing those questions and processes and incorporating them into an algorithm for designing and developing is how you learn from past mistakes and ensure you do better next time. So one last section must be added - a section to capture your specific algo. For each issue you come across, for each mistake you or someone else makes, capture what could've saved you and add it to the algorithm in this final section. This will ensure that in the future you are able to engage that new bit of wisdom at the right time. And as a result you'll end up building a self refining process of design and iterative execution.
\newline

With this final piece in place, we can now see the "collection". It starts with who you're working with, why, and how. It gets into what you all are building together from a systems thinking perspective along with the queue of questions and work required to bring about greater and greater accuracy and precision in that vision. It captures the mental models and knowledge underpinning those ideas along with all of the attendant questions, research, and hypothesis that need doing or resolving. As it comes time to get feedback from the real world it captures the evolving plan on how to iterate and be frugal. And finally, as you learn from doing, it captures your cumulative experience in an algorithm for design and execution in your specific line of expertise.
\newline

Alright, that was a lot, but let's step back for a moment and consider something astonishing. Suppose you start this whole thing with a horrendous idea. I don't mean an evil idea, just one couched in a lot of naivety. Through this process you'll start by engaging with people - users, developers, and benefactors - who will quickly help you see what you're missing. But in doing so will also provide the seeds for new ideas, ones better couched in reality. Or suppose that you feel like you have this amazing design but it turns out that there's something horribly wrong in one of your mental models. The focus on empiricism and testing your ideas will quickly lead you to uncovering your error which in turn will allow you to go back to the drawing board and think of something more feasible. Or perhaps you, your stakeholders, and all of your data suggests you're right, and yet unknown unknowns are waiting for you around the corner. By iterating and being frugal you'll quickly learn what you need and be able to work with your collaborators to update and pivot.
\newline

In other words, with these processes and mentalities it doesn't really matter a whole lot where you start. Even if you begin with bad ideas these guideposts and interactions will drive you closer and closer to something that does matter because you'll be getting constant real feedback from all of the contexts that underpin whatever it is you are building. Put another way, so long as you are committed and follow the feedback you will find your way to something successful in the end. 
\newline

What we have here then is part of the solution to the problem posed in the last chapter where we asked how someone with a singular focus can end up creating value. This way of thinking allows an individual to have singular focus, and through constant, thorough, and consistent feedback find their way to something meaningful. But as we just noted, this only works if they stay doggedly true to task. So there is still a key piece missing here - what creates the kind of motivation where the ideas and details can keep changing, your assumptions and designs can keep failing, and yet you still stay determinedly on the path? What allows you to stay motivated, involved, and, most importantly, happy when you don't really know where the adventure leads? It is to this question that we turn next.

\chapter{Follow the Adventure}
We \textit{plan} vacations. You figure out where you want to go, when you want to go there, for how long, what you'll be doing, how you'll get there, and so on. Why? Because you want to ensure you have a good time. Go to the wrong place at the wrong time and a relaxing vacation can become miserable. Fail to plan how you're going to get around and things may end up being way to expensive or simply not happening. And planning a good vacation means being thoughtful about what you like to do and how you like to do it (not to mention whether those things fit within your budget). 
\newline

We tend to treat life, and especially our careers, in much the same way. It starts real early when people ask you what you want to be when you grow up, becomes more serious in high school when you start having to take specialized exams to get into the colleges you like, and becomes a downright existential crises once you actually get to college. From there there's lots of job planning, specialized classes, and just in general trying to plan out your professional life so you don't end up screwing up your career path. People ask themselves a lot of questions like - what am I good at, what do I like to do, what is my purpose on this planet, do I like working with lots of people or mostly on my own, and so on. Like with vacations, we're given the sense that planning everything out is very important to both our success and our happiness. Create a goal, create a path, and follow it.
\newline

However, the last chapter showed us quite a different perspective. Rather than having a fixed goal and an easy path we instead found a maelstrom of pivoting, learning, iterating, and just in general having to change your path, mind, and goals over and over and over again as you learn what the world (and your ideas) is really like. Indeed the last chapter encourages not getting to attached to any specific goal because what you are building or doing will change quite a bit over time and you need to have the flexibility to follow the cues so that you can get somewhere successful rather than just burying your head in the sand. 
\newline

Now, while hopefully I was able to convince you how important it is to follow the cues, it's also a rather terrifying prospect! To make the analogy once again to going on vacation, rather than planning things out this is much more like just getting a general idea of where you want to go, booking a one way ticket to get there and then kind of letting the adventure just take you. Who knows where you might end up? Who knows exactly how much it'll cost? Who knows what you'll end up doing or exactly who you'll meet along the way? To a culture so focused on planning and having insurance that things are going to work out all of that sounds like a lot of anxiety. In our case rather than picking a career and career goals up front to match your skills, lifestyle, and "purpose" you're just starting with an idea or opportunity and seeing where the winds take you. How on earth can we be sure we'll enjoy the ride?
\newline

But it gets even sketchier because remember this isn't some work-free adventure vacation - this is work! So in addition to all the twists and turns there's also going to be a lot of hard labor, frustration, costs, stress, challenges, and worries. Not exactly selling this am I? We've just said this "cheap failure" approach is the key to successfully creating value and yet I'm also saying it requires seemingly giving up control while also having to deal with loads of challenges. All in all how can we stay true to the path when we don't know where it leads and its covered in boulders, brambles, and all manner of obstacles? How do we stay motivated? How do we enjoy it all? How can we know it's for us?
\newline

To answer this question we must turn to another part of life in which we face challenges, overcome obstacles, deal with frustration, and generally let the adventure take us. However in this part of life we do so with glee often bordering on straight up addiction. Many of us have had these experiences - trying to keep an empire from falling apart, trying to raise half a dozen kids while working a part time job, running up against the clock to take home the prize, or finding ourselves at our wits end on a battle field. Games. Games are a perfect example of how to find challenge and adventure absolutely awesome instead of absolutely anxiety ridden. 
\newline

Now I know what you're thinking - games just give immediate gratification, games are not filled with real life difficulties, games deal with exciting story lines - in short games are cool 'cause they remove the annoying stuff about life. However I think a quick look at the kinds of games out there people play is enough to show that that's really not true. Think games are just about instant gratification? Look at games like "Animal Crossing" or "Pokemon Go" that run \textit{in real time} and in some cases require you to go to real physical locations to get things. Think games are just easy peasy? See how long that hypothesis lasts while you play games like "Dark Souls" or "Elden Ring" or hardcore difficulty on just about any strategy or combat game. Think games are just about exciting story lines? Then how are the "Sims "games so incredibly popular? You literally deal with high school drama, change diapers, and work a desk job. Games are not fun and exciting \textit{in spite} of the challenge, instead they are somehow exciting \textit{because} of it. And so, in order to understand how to make our "cheap failure" approach equally exciting we simply must figure out how to turn it into a video game. 

\section{One Last Turn}
At the end of a game of Sid Meier's Civilization you're given the option to take "one last turn" which more or less allows you to play your game indefinitely past the point of actually winning, if you so choose. It's a bit of a not-so-inside joke because of course, anyone who grew up playing video games is well familiar with asking your parents or guardians for "one last turn". This is one of the great strengths of games in general - there's always something to do. In "Civilization" there's always a new wonder to start construction on (or one to finish), always a battle to be fought, or a world congress to attend, or a technology to choose, or... you get the idea. In the "Sims" there's always a new skill to learn, a piece of furniture you're just moments away from buying, or a relationship you're about to cinch. And you know about all of these things because the game designer's do a really good job of keeping you distinctly aware of everything that's going on in the game at all times. 
\newline

Continuous engagement and feedback is... well... the name of the game. And it all starts with the trigger response mechanism we met back in the chapter on focus when we spoke about habits. Remember  that habits form when we get a specific kind of trigger that, after some specific set of actions, results in a specific kind of reward \cite{duhigg}. Repeat this several times and your brain begins to anticipate the reward when it sees the trigger creating the \textit{exact} kind of motivation that causes you to ask for "one last turn". Creating habits then requires three things, a trigger, a relatively repetitive kind of action, and a reward. Games make this possible by structuring themselves into many different specific actions or minigames, each of which has a specific visual or auditory trigger and when completed gives you some kind of bonus or score. In "Civilization" there's typically a little notification bauble that indicates the next action that needs to be taken and depending on the kind of action the bauble changes shape, color, or design. Then that action always takes you to the same screen where you look at information in a similar kind of way each time, and choose from a subset of focused actions or ideas relevant to that task. Then depending on the kind of action taken you might win a battle, build a new wonder, start earning some more money, get some culture points, and so on. Trigger, action, reward. Before you know it the moment the bauble turns yellow and shows a hammer you're excited for selecting a new building before you've even gotten to it! The principle is simple - by taking advantage of the anticipation of reward that comes along with forming habits, games keep you motivated about the next move continuously.  
\newline

Note though that this requires that the "game" (whether that be a video game or a project you're trying to gamify) be broken up into small, repeatable parts that have this trigger, action, response format. When applied to our everyday work that means breaking things into small pieces, having a clear set of triggers that will keep popping up as you work, and making sure at the end of action there is a clear reward that you get. By doing this we begin to build up the same kind of engagement we get from video games by also tapping into our "habit brain". 
\newline

And there's another reason why breaking things up into these smaller pieces makes for a more enjoyable time and specifically a better attitude toward challenge and failure. It turns out the degree to which you view challenges as positive opportunities for growth or as a source of pain and frustration is related to not how \textit{big} your successes or failures are but to how many of them you end up having \cite{superbetter}. So when you go many weeks without feeling much success at work because you're just focused on meeting that deadline you end up feeling pretty useless and un-optimistic. But if instead you have loads of little successes day in and day out then even a failure here and there can't dampen the overall sense that you're a rockstar. And this keeps you optimistic and motivated and more likely to hit that big deadline. This also part of the "secret" as to why games are so powerful. While you certainly have loads of challenges in games and lots of failure to boot, by keeping things small, repeatable and manageable, games give you lots of successes to and therefore keep you largely optimistic \cite{superbetter}. If you want to learn more about this specific phenomena check out the "Power Ups" section of McGonigal's book \cite{superbetter}. 
\newline

Now while keeping things small and repeatable allows habits to build up, small and repeatable can also lead quite quickly to boredom. Eventually doing the \textit{exact} same thing over and over again will just lead to one getting sick and tired of doing the same thing over and over. So how do games solve this? By incorporating diversity and challenge. The first, diversity, is pretty obvious - "Civilization" doesn't just ask you to build buildings, there's also communicating with other rulers, leading armies, exploring the world, researching technologies, and so on. Likewise in first person shooter games there's different maps, weapons, objectives, and so on. The underlying mechanics stay repeatable but diversity keeps any one thing from getting too boring. But even diversity can stave off challenge for only so long. Thing is that your brain just gets used to whatever it is feeling and if all it is feeling is success then it soon normalizes to that and you just end up not caring anymore. Therefore games intentionally incorporate challenge to introduce frustration, fear, and other obstacles to success so that when you overcome them you can actually feel successful \cite{superbetter}. There's no victory without challenge. 
\newline

One way to illustrate the power of this is a common issue people have on vacations. People who go on vacations with "nothing to do" quickly get bored of just relaxing. They'll end up looking for things like golf, or hiking, or board games to play. And all of these possess what folks term unnecessary obstacles \cite{superbetter} - things that provide the kinds of challenge that allow us to enjoy feeling victory again. For example, if golf was just about getting a ball in a hole you'd just pick the ball up, walk over to the hole, and drop it in. Sound pretty boring right? All the rules in golf are introduced to make it difficult and thereby fun. Fun requires challenge and so when we're lacking it we \textit{invent} challenge for ourselves by creating unnecessary obstacles. Game designers recognize this need for challenge and incorporate it straight into the game along with another principle - leveling.
\newline

As you take on the game's challenges and get the better of them you end up getting better at the game. This is a problem because it means what was challenging before ends up becoming easy and throws us right back into the boredom rink. So how do games solve this? They get progressively harder as we get progressively better thereby keeping us challenged and engaged.
\newline

So how do we apply these lessons to our own lives? First is recognizing that the only really problem with challenge is when it overwhelms us. Therefore if we do come across challenges that are completely overwhelming we should think like a game designer and ask ourselves how we can level down the challenge into some smaller pieces that will allow us to build up to the level we just walked into unprepared. In other words, as before, we want to break things up into smaller more manageable pieces. But manageable shouldn't mean easy - we should make sure the work is still challenging otherwise we'll just grow bored. Finally, we need to make sure that there's some diversity in what we do so that we don't just normalize to a particular kind of work or challenge overall. This diversity doesn't necessarily mean taking on multiple jobs or projects but often times just means having a home life, having hobbies, hanging out with friends, etc. What's important is having other things in our lives so we can relish our work rather than feeling a slave to it. Balance is key here. 
\newline

One final point of advice here is that \textit{all} of this requires being conscientious and seeing the little challenges and wins. As we mentioned before, if all you're doing is keeping track of long term goals there's no way to relish the immediate challenges, build habits, level yourself, and so on. But often times the "mini goals" or "mini games" are not apparent and in some cases they may, at first, not exist! In other words you have to get inventive. Find ways to create scores for yourself, find mini games to play that build into your work, create levels you want to strive toward, and so on. All of this may seem a little silly at first but there's a lot of research and anecdote to prove your mind (and work) will love it \cite{superbetter}! The point is to be creating and think like a game designer to keep yourself engaged, challenged, and growing. 
\newline

Okay, so at this point we've broken down our work into things that have clear rewards, relatively quick turnaround, and keep us challenged. We've organized our work so the triggers for each of these habits stay in clear view in order to keep us continuously engaged. Our optimism is building as we crush challenge after challenge. And diversity, challenge, and leveling are keeping us from getting bored or complacent. So now, when we show up to work we start getting excited the moment we start seeing our triggers and anticipate the victories we're going to have today, we appreciate the challenges as they come because they make the wins so much better, but we keep ourselves from getting overwhelmed by keeping ourselves challenged without drowning. At this point you may still feel like this is still all a bit silly. Did the greats ever have to do this? Did they really work like this? Was their life built like a video game? Turns out, that's absolutely how they worked. Mihaly Csikszentmihalyi's whole career has been dedicated to figuring out what creates success in people and one big result of his work was discovering that great work (and what he terms peak experience) comes from a state he coined "flow" \cite{flow}. Flow is the state we find ourselves in when we have clear goals and feedback, when we lose a sense of time and self, become absorbed in our work, find ourselves challenged but overwhelmed, and just generally lose ourselves in what we are doing - the exact "gamified" state that we just described! And he discovered this phenomena while studying some of the best living artists, scientists, engineers, athletes, and writers. In other words the same principles that fuel games fuel the very flow that has powered the human greatness all around us!
\newline

What's truly fascinating then is that games seem to have demonstrated that flow can be achieved around pretty much anything. I mean think about games like the "Sims" where you \textit{literally} just live a normal life. Or "Animal Crossing" where you have to \textit{wait} for crops to grow in real time. Or the fact that we have been deriving flow as a species from chess for centuries - a game whose rule set can be explained in a couple of minutes and which never leaves the same 8 by 8 grid.
\newline

So is that it then? Have we solved it? So long as we can turn our "cheap failure" approach into a flow machine are we good? That thought should make you immediately uncomfortable because it seems to suggest humans can just do whatever and be fine - a statement that is contradicted by pretty much every living person's experience. So what gives? The answer comes in recognizing that even in games like chess, "Civilization", futball, or the "Sims" that have captured the attention of so many different people, everyone plays differently and with different goals in mind. But to follow this lead we must take a step away from games and toward a discipline known as logotherapy.

\section{What Makes it Worth It}
Victor Frankl was a psychologist who, during WWII, ended up in a concentration camp in Nazi Germany and found himself struggling with how people can find meaning, purpose, and the will to live in even the most dire situations. In such a terrible setting he found that many of the things we subsume as one when thinking about well-being are really entirely different dimensions of experience. For one thing, in his book "Will to Meaning", he points out that while many people think of themselves as "pursuing happiness", emotions are not really ends but just indicators we use to find the ends themselves \cite{frankl}. As he puts it, when you feel enjoyment from something this should not indicate that you should pursue more happiness but rather than you have found something that matters to you, and \textit{that} new end should be what you pay attention to. Likewise he points out that, unless people have serious neuroses, those who lose a loved one don't want their sadness taken away from them as that sadness is a natural consequence of loss. In his mind then, emotions are more like dials on a dashboard - indicators, not the point. If your gas gauge falls too low, you go and get gas not to move the gauge but to ensure that you can keep driving. 
\newline

In a similar way, Frankl points out that things like wealth, and money, and power are simply just means \textit{to} ends. For example, he cites one study of Harvard graduates who, while extremely successful in the standard sense of the word (doctors, lawyers, and the like) felt they had a crises of meaning and felt like life was empty and pointless for them \cite{frankl}. In stark contrast to this, from his own experience in the concentration camps Frankl knew that people could find hope and purpose in even the darkest of places where "success" in the normal sense of the word seemed infinitely far away. In short, success and fulfillment exist along orthogonal dimensions. 
\newline

So if not happiness, success, or power, what is it that we are after? We are after the ends that these things point to or help us achieve. What happiness is indicating, what we pursue success for, those \textit{ends} are what matter to us. What Frankl found was that we spend so much time worried about happiness and power that we often don't find ourselves conscientious of the value in our own lives and therefore end up feeling rather awful in the face of a lot of opportunity. For example in one case he had a patient who was feeling lost at his job. Every other therapist the patient had seen was focused on the emotional side of things. But Frankl just ended up having a frank conversation with the patient and realized he just wasn't interested in what his job \textit{did}. So Frankl advised him to simply switch careers to something he found meaningful and voila the patient's crisis disappeared \cite{ikigai}. 
\newline

These ends, this meaning, is what makes everything else matter and everything else tick. This is why people play different games and/or play the same games differently - because they perceive meaning differently. When playing the "Sims" some people go down the artist route because that speaks to them whereas others focus on building big families. Some people go for science victories in "Civilization" whereas others try to become masters of military strategy. Our perception of meaning is behind all of this. 
\newline

So, in order to find motivation and flow we need this foundation of meaning. How then do we go about building and finding it? Obviously I'm not about to expound the secrets of life in the next few pages, but I think there's a few pointers we can take from game design, positive psychology, and logotherapy that will help us a long way in understanding how we can embrace the adventure specifically.

\section{Built not Found}
The first pointer is that meaning is shared. We talk a lot about personal meaning and as we'll see in a bit there's definitely something to that but as Frankl points out it's not precisely what we usually think. In Frankl's view meaning is not subjective \cite{frankl}. He illustrates it by way of perspective. Each of us, as he points out, has a different perspective on the meaning of things - it's just as how if we all stand at different points around a building we have different perspectives on the building. None of our perspectives are wrong, but none of them are complete either - building a complete picture of the building in front of us requires putting our heads together, appreciating our various perspectives, and trying to understand how they are part of a whole. Trying to assert that all are complete or only one is right just results in us shooting ourselves in the foot because we then never end up walking away with the "real" picture. Instead, like the parable of the elephant and the blind men we all walk away with a wrong impression - something we certainly don't want to do with something so important as meaning! 
\newline

Meaning then should not be something you try to understand or build on your own. Instead it should be a process in which you invite and seek out the perspectives of others and in as many ways as you can. You are not alone in your meaning and finding it is a cooperative act among us all. 
\newline

The next point is that meaning is positive. As Frankl points out, meaning is something we strive toward, something we attempt to transcend to in a way \cite{frankl}. Therefore meaning is not about finding problems it's about finding solutions and goodness in the world! If you made a massive list of all the issues you still wouldn't have something concrete to strive after. So in developing meaning it's important that we conscientiously observe and keep track of the positives, the solutions, and build our meaning from those things. But the usefulness of this mode of thinking extends beyond just building meaning, this way of thinking abounds in both game design \cite{superbetter} and positive psychology \cite{seligman}. McGonigal points out that when we pay attention to what she terms "power ups" (the wins and positives we can have each and every day) we find ourselves becoming more optimistic, positive, and just generally resilient. Likewise most of the most potent interventions in positive pyschology are simply about being mindful about the wonderful things that have happened in your day, week, or year. By shifting our focus to what's going well and what we are thankful for, we naturally find ourselves building meaning, hope, and motivation for ourselves. 
\newline

Meaning is also something we discover as we make our way through life - it's not something we just find all at once lying somewhere on the street. Meaning comes from taking our experience and attempting to parse what matters to us out of it. This, in my mind, is an extremely important point because a lot of folks (including myself) go looking for "their purpose" as if until they find it life cannot start. But purpose is found by being a part of and experiencing the wonders and solutions life has to offer! What's important then is not that you have some kind of complete answer, but rather that you are resolving your own sense of meaning and being mindful about it. To bring back the example earlier of the patient who just needed a change in career path, it wasn't that they didn't know what their purpose was, it was just that they needed Frankl to help them make constructive sense of the emotional indicators they were getting at the time. Likewise positive psychology and even our discussion of game design earlier demonstrates that it's not about having no questions, or no challenges, but rather that you feel in overall scheme of things that there's a greater balance of positive rather than negative. Put another way so long as you feel you are moving in a good direction that captures what you know about yourself you'll be less concerned about what the "right" direction is. People get obsessive over "rightness" when they feel themselves stagnating.
\newline

If that last piece still feels a little wonky to you consider how when playing video games or watching a movie or reading a book, the full meaning is never revealed until usually the very end. Yet you're not worried because you \textit{trust} the author of the story to bring you there in due time. In a similar way so long as you can look back on your life and see yourself moving and growing in ways you feel proud of you'll learn to \textit{trust} yourself as the author of your own story and be less concerned with where it all leads because in the past you've always ended up finding your way. 
\newline

At this point all of this should sound remarkably familiar. Building things with others? Iterative development? Focusing on the solutions and good we can bring to the table? This is exactly the "cheap failure" strategy outlined in the last chapter! Building meaning is like any creative activity in life and so the lessons learned creating other things can be applied here as well. The key to building meaning specifically is recognizing what it is you are trying to build and who should be involved in helping build it which brings us to perhaps the most important point in this whole chapter.

\section{You are the Hero}
All of this meaning building is about finding what matters to you so that you can strive towards it. But here's the thing - there's only one thing you can directly change in this world - \textit{yourself}. As Frankl puts it, even in the most dire of situations where it seems like you are powerless you still have the ability to take a stand - one's own agency is never lost \cite{frankl}. So this meaning, this narrative, it has to be centered around \textit{you as the protagonist}. If you build all these visions for how the world can be but never consider your place in that story you're just going to become frustrated with the magnitude and seeming immovability of the world. But, if instead you focus on centering this whole thing around your own agency and what and who you can be, then you possess all the power. 
\newline

Beyond this we find that once again both game design \cite{superbetter} and positive psychology \cite{seligman} point to the fact that in shifting the framing of your life over to you being the hero of your own story makes you more imaginative, creative, optimistic, and resilient! A great example of this is \textit{self distancing}. You know how you'll often give more forgiving and hopeful advice to others than you'd ever give yourself? Well it's been shown that people will give themselves that same advice if they view and talk about themselves in the third person \cite{superbetter}! Making yourself the hero gives yourself the chance to be fair and kind to yourself in the same way you would be to a close friend.
\newline

So what is it you are building? You are building your own adventure! One in which you are the hero of the story. And all of this centering around yourself is not about being cocky or selfish, it's about recognizing that the one thing you have control over is your own agency and so your agency should drive your story. This is the meaning we build, this is the narrative that drives our motivation, and it is therefore this that should be at the center of gamifying our own life. We need to build an adventure worth striving after in which we are the hero. 
\newline

\section{Journey over Destination}
Meaning is what gets us out of bed in the morning. The new challenges to overcome, the new experiences to enjoy, the friendships to relish, the impact to have. But as we've just seen, this meaning is something that grows and develops and changes over time. In short life \textit{is} adventure. There is no plan that can capture it all or that won't change as time goes on and so enjoying life is not about whether or not it is an adventure - because it just is. Instead it's a question of how you \textit{approach} the adventure. Do you center your view around your own agency or do you end up getting overwhelmed by the sheer scale of the world? Do you focus on all the small wins you can get each and every day or does your life become a drag from one far off deadline to the next? Do you focus on your "power ups" and all the positives in life or get drowned in an ocean of unknowns and issues? There's no need to have some long term plan. Instead what matters is having a grasp of what ways of being and what processes, in the now, will allow your adventure to unfold in meaningful and exciting ways. And this isn't something you have to figure out on your own - game design, positive pyschology, and logotherapy abound with ideas, strategies, and recommendations for how to do just that. Follow these strategies and you'll find life turns into an adventure that is fun and motivating. Because motivation is not about what life becomes, it is about how we approach it.

\chapter{Never Stop Learning}
I'd warrant that the vast majority of challenges anyone comes across can be solved directly with learning. For example, if you find yourself in a position where something you're building is just not working right it's highly likely that someone else has run into the same problem before. Learn from them and you'll have the solution to your problem. 
\newline

Or perhaps you're having trouble communicating something to your management. Once again it's unlikely you're the first person to ever run into this problem. So, go find folks who've been successful at overcoming this issue and learn from them rather than trying to reinvent the proverbial wheel. 
\newline

By and large, I think we have the tendency to think of our situations as unique and therefore requiring novel treatment, but as a general rule I have found human experience to be so ubiquitous that there's seldom a situation I've seen that was \textit{truly} new. Usually a blog post, book, youtube video, or conversation had the answers I or someone else was looking for. In general, learning solves the vast majority of problems.
\newline

This pattern continues if we return to what we noted about creativity in the chapter on peak performance. As we saw, creativity is really just the process of taking something we've learned elsewhere and applying it to something new. So even when we \textit{are} dealing with novelty, one of the main ingredients is, once again, learning.  
\newline

In sum, learning is the solution to the vast majority of our problems and is also the means by which we fuel our creativity, however the value of learning doesn't stop there. For one thing, learning creates perspective. For example if you're learning to bake or cook all you see are ingredients and steps, but as you get better with practice you start to see flavors, techniques, chemistry, textures, smells, colors and so on. As you learn your perspective gets richer and this richer perspective helps you see new opportunity. So not only does learning help you solve problems, it allows you to see opportunities and problems in the first place.
\newline

Furthermore, learning and its counterpart, teaching, are also a means by which we support each other professionally and, oftentimes, personally. See a coworker struggling with something that you've got down pat? What a great opportunity to help them out. Likewise, when you're struggling with something a coworker or mentor can help you out by showing you what they know. By learning from one another we grow and develop so much faster than if we were to just try and figure things out on our own.
\newline

Learning then is at the center of everything we do and, more fundamentally, is the means by which we make ourselves more capable. So it follows that one of the most important questions we can ask is this - how can we go about learning effectively? 

\section{Learning through Doing}
The first question we should ask ourselves is of course this - what is learning? Learning can be many things, but for our purposes we're looking to increase our capabilities. And we specifically want to do it in such a way that it becomes second nature and allows for the layering that creates experts \cite{ericsson}. That means we want to take skills and knowledge and embed them \textit{usefully} in our intuitive systems. But as we now know from several sources \cite{duhigg} \cite{ericsson}, your mind only embeds things that deep if it believes it part of a new norm - a deviation from your standard homeostasis. Several practical observations come from this.
\newline

\subsection{Embed the Right Thing}

First of all, and most importantly, we have to very careful about what we end up embedding. For example, when I was growing up and learning mathematics, all of the problems were very much abstracted from reality. As a result I ended up being able to take integrals, calculate the volumes of shapes, do geometric proofs, and solve all sorts of equations but ended up with no real sense of how to apply any of this to reality. As a result, mathematics, for quite a while, felt pretty useless. This was because I was embedding abstractions without embedding the means or routes by which I'd use those abstractions. However, in college I ended up taking a class that did things the opposite way around. We started with real problems that showed up in research or in the exams we were taking in other classes, and then worked back to finding the appropriate mathematical principles and applying them. As a result I was not just learning \textit{about} math but was learning how to \textit{use} math. The difference was profound. Suddenly math was useful and I knew how to use it instinctively. The moral of the story here should be clear - knowledge and/or skills are only useful to you in context and so you need to ensure you're embedding them in context for what you learn (and therefore what you embed) to be useful to you. Practice what you want to be able to do, not some abstraction of it \cite{stick}.
\newline

In line with making information useful comes acknowledging how your mind makes use of information. It's not as if knowledge is always online and ready to go in our brains - ever present information would be incredibly overwhelming. So instead, your mind stores things away through relationships and triggers. But what this means is that for what you are learning to remain useful in the future you need the right kinds of relationships and triggers. Like with my math class what makes information useful is linking it to the problems and opportunities you want to trigger its remembrance. Without doing that it's like locking information away in a vault and then throwing out the key. The information is there somewhere but nothing you do will ever really bring it to the fore. So once again, as you're learning it's important to practice things in the context you want to use. 
\newline

However, another really useful tool for creating rich context and connections around something is to reflect on it and try to tease out rules, relationships, and meaning \cite{stick}. By really moving the information about in your head and looking for patterns and connections you create the kinds of mental links that will make the information more useful later. Indeed, even this book is just my way of doing exactly that! So as you're learning about things, take the time to reflect on them and very conscientiously look for connections, mental models, and other useful kinds of context. 
\newline

\subsection{Shift the Norm}
Okay, so far we've talked about making sure that what you embed is embedded usefully. But how does one actually do the embedding? Well remember from what we learned in creating peak performers and building habits - embedding requires altering your norm in such a way that your brain throws things into your "intuitive" brain to establish a new homeostasis \cite{duhigg} \cite{ericsson} - and that means continued, effort full practice. 
\newline

First of all, creating this "new normal" requires practicing over time. Cram for a test or try to learn a new skill at a bootcamp over a couple of days and you'll likely learn very little \cite{stick} largely because no normal is actually created - as far as your brain is concerned the experience was just a very exhausting aberration. Creating a new normal takes time.
\newline

Second, if what you're practicing is always easy your brain has no need to re-circuit anything - the point of re-circuiting is to make something difficult easy. So to really learn something in a meaningful way you need to make the practice difficult. One way to make things difficult that also improves (and is possibly necessary for) the usefulness of learning is varied practice \cite{stick} \cite{ericsson}. Your brain is tricky and very lazy and therefore looks for shortcuts wherever it can. If you just practice the same index cards in the same order every time, your brain will literally learn \textit{just that deck}. To actually learn the information or skill you have to keep things varied so your brain \textit{has} to learn the whole skill and all its nuance rather than some "spark notes" version of it.
\newline

Furthermore, effort only happens when you do something your brain isn't yet accustomed to doing. Things like recalling knowledge you just learned about will feel awful at first because you never remember anything, but that's the point - it'll cause your brain to file away the information precisely because you needed it, couldn't find it, and got frustrated. Likewise, practicing a skill is rarely all fun but it is the effort and frustration that gets your brain to pay attention and start to learn. 
\newline

Another great way to create effort and one that is recommended to teachers and coaches alike is captured in the "generative principle" \cite{stick} \cite{stanier} \cite{coactive}. The idea is simply this - if the student or coachee is the one to come up with a solution or idea it's far more likely to stick than if you simply tell them. This principle is also seen when you give people problems they don't have the tools to solve, let them struggle with it for a bit, and then point them to the information or skills they need to solve it. You yourself have likely experienced this yourself (perhaps accidentally) as it happens any time you get stuck on something and then go looking for the answer - having struggled makes the answer so much more meaningful and therefore helps it stick.
\newline

Finally there's the notion that making errors in learning is essential to the learning process \cite{stick}. Errorless learning never creates the kind of struggle needed for your brain to catch on that something has changed and therefore leads to far less retention and embedding.
\newline

All of these principles are what the authors of "Make it Stick" \cite{stick} describe as desirable difficulties. They are difficulties or challenges that shift your norm in ways that catalyze embedding. However, they also point out that it's important to recognize that not all difficulties are desirable - some just provide barriers without making learning or embedding more effective. For example making someone feel bad about their learning performance or giving people tasks or information far beyond their capabilities both represent challenges that will just grind people down rather than helping them learn. Effort is necessary to learning, but not all effort is helpful.

\subsection{Avoiding Illusions}
One really big obstacle to learning is the myriad ways in which we can end up with the illusion of learning without actually learning. A couple simple examples will make this clear.
\newline

The first example is that feeling of mastery you get once you've finished reading some non-fiction. Perhaps there were a few sections that at first went over your head, but you stuck through and struggled with them until you felt like you understood them and now, thanks to the book, you feel pretty fluent. However for most of us, if you then go ahead and test yourself on what you "learned" you'll find most of it went in through one ear and out the other \cite{stick}. This is especially weird and frustrating because if you then go back and start reading the book again you'll recognize everything. So what's going on? The point is familiarity or recognition isn't real learning. Another example of this would be that you probably (unless you have face blindess) recognize all of your friends' faces. However, unless you're quite the artist, if I asked you to draw their faces from memory you'd probably fail. Familiarity is not learning.
\newline

Another classic illusion comes from your brain's ability to embellish the truth with loads of lies. For example in "Make it Stick" they talk about an experiment where adults who were asked whether they'd broken a window in their lives were more likely to think they had later even if they'd never broken a window! This also shows up when students very confidently answer questions with knowledge that is just frankly not true. Our brains are very good at making us confident about things we have no reason to be confident about \cite{stick}. 
\newline

There are loads more of these kinds illusions (and becoming familiar with them is a pretty good way of finding tips and tricks for learning more effectively) but these two both demonstrate the point and also indicate that the only real way to know whether you've learned something or not is to test yourself. 
\newline

Which immediately brings up an issue - tests typically suck. However, that's because they are used not as a learning tool but as a gateway. In a typical classroom tests cause people to moan and complain because they know it's going to be something like a midterm or final where how they fare determines a lot about their academic future. However tests don't have to be like this, they can simply be used as a tool to assess where you stand, point out weaknesses, and then allow the learning to be directed more effectively. In other words, rather than testing once to figure out whether "someone is worthy", tests should be used over and over again to drive learning in the directions needed and ensure that eventually you ace the testing. Testing should be a tool not a gateway.

\subsection{Find a Guide}
Alright so we've talked about what to embed, how to embed it, and how to make sure the embedding really happened, but there's one final detail worth mentioning - the need for slow, guided layering. 
\newline

The whole point of this learning escapade is to layer things into your "intuitive" brain and from what we've seen, we now know this is a slow and laborious process (that's how we create a new norm and thereby trigger the brain to encode). So what this means is that trying to give a student all the information at once is going to do no one any good. They need to be able to work up from the simpler to more complicated things one step at a time - with patience \cite{ericsson}. 
\newline

However, this presents a problem, as knowing which layers to do first requires knowing the subject and the student, by definition, does not know the subject. Therefore, the student \textit{requires} a guide. That guide can be a person, an introductory text book, or a well organized youtube video but the point is every student \textit{must} have a guide - otherwise they'll just get overwhelmed and go no-where. And that guide must enforce a clear layer by layer strategy that the student must follow patiently if they ever wish to master what they intend to learn. 
\newline

Guidance is key because guidance is what ensures all your hard work doesn't go down the drain of confusion and frustration.

\subsection{Pulling it Together}
Okay so learning requires (broadly speaking) the following things:

\begin{enumerate}
\item To practice the same way you'll do in order to embed things properly
\item To really reflect upon and give context to what you're learning about
\item To incorporate desirable difficulties to trigger the embedding
\item To practice over time in order to actually shift your norm
\item To use testing to break through illusions of knowledge and objectively guide your learning
\item And to find a guide who can help you learn layer by layer
\end{enumerate}

However there's a bigger problem that's been looming in the background this whole time - what should we set out to learn in the first place? There are so many different things we could learn, so many different skills we could acquire, and indeed even within a field there's so many places to start and stop. So how do we make sense of it all? How do we sort out what matters to us? For that we turn to the subject of coaching.

\section{The Meta Skills}
People look to coaches for all kinds of reasons - trying to improve relationships, trying to figure out how to improve their work-life balance, looking to enrich their lives with new skills, etc. But in general people come to coaches because they are trying to improve some aspect of their well being \cite{coactive}. As a result coaching is extremely tricky - it is dealing with perhaps the most fundamental and at the same time most personal aspect of our lives - how to engineer a life that makes us feel fulfilled and happy. Thankfully, however, while it is definitely a tricky business, there is a vast literature out there on how to be an effective coach that anyone can pull from. But this vast literature also indicates something else - coaching is a skill like any other, and skills can be taught. This brings us to a rather profound point - coaching should aim at its own obsolescence. In other words coaches should try to give people the skills to find their own well-being rather than creating some kind of dependency between the coach and coachee. As such, we can rephrase what coaching is more succinctly - coaching is the process of teaching people the meta skills required to drive success and well being in their own lives. Coaching then, is just another form of teaching and everything we just talked about in the prior sections applies! However, there are a few points specific to coaching that I feel are worth mentioning.
\newline

The first is that coaching has to be about \textit{the whole person} \cite{coactive}. Why? As we learned from the "Will to Meaning" \cite{frankl}, motivation itself is about the whole person. Meaning and motivation comes from building a collective, cohesive narrative about ourselves. So trying to treat one aspect of our lives without connection to the others is a pointless affair. To make this more concrete think about all the times when struggles in one aspect of your life (say at work) has leaked into other aspects (family and friends for example). Ignoring one context while trying to improve another is like trying to juggle one ball at a time - you have to keep track of everything that's up in the air. So, as a coach, next time you're tempted to say something like "that's not work related", think twice.
\newline

Second it's important to recognize what coaching is about. Too often mentors, managers, and coaches will end up focusing on the problem at hand rather than the underlying processes that are making this a problem in the first place \cite{coactive} \cite{stanier}. For example, an employee might come to you with some technical questions and right out of the gate you're just dying to hand them one solution after the next. However, remember that the point of coaching is to get your coachee to a point where they \textit{don't} have to come and ask you these questions. So rather than providing a solution, you want to give them the tools required to come to their own conclusions. Once again, you want to focus on teaching them the meta-skills rather than providing them with a solution. 
\newline

In a similar way we often tend to end up patting ourselves on the back every time we help a coachee overcome a problem even though they might be just as helpless in a long term sense as before. The real measure of success as a coach comes from coachees not returning with similar problems in the future because they no longer need you and can just sort it out for themselves. Help guide them through the process of solving their own problems, don't try to go and solve them for them. (Something that is understandably very hard when you yourself may be under a lot of pressure)
\newline

Next, it's important to recognize that anyone's adventure in life is going to be filled with a lot of failures, disappointments, and struggle. Your coachee may think they're on the right path only to discover something isn't right for them after all. Or in the process of learning they may find themselves feeling pretty down and stupid given how difficult a specific subject may be for them. Or they may have come to you in the first place because they feel lost or confused or are finding it very difficult to be successful in some aspect of their lives. Regardless of the reason for their suffering the fact of the matter is that in the process of finding themselves and building a great life, suffering is going to happen (indeed as we've pointed out several times now, frustration, failure, and effort are not only unavoidable but often times necessary for growth). But this is where you as a coach can be their champion and help give them the skills to keep their head up and see the struggles and trials as a sign that they are growing and succeeding! But, once again, remember that rather than becoming their source of self worth (and thus creating a dependency) you need to coach them into becoming their own champion. 
\newline

However all of this is going to take time and so, as we mentioned in the last section, it's important to layer things in one at a time. Be conscientious about the skills you want them to learn, stay out of their way so they can learn those skills, but at the same time recognize that there will be a lot of hand holding going on for the layers they haven't gotten to yet. The point is to help them learn, not to overwhelm them. In the end, the goal is obviously to make yourself obsolete, but throughout the learning process they will need to know they can lean on you for the things they haven't gotten to learning yet.
\newline

Finally, and most importantly, you have to be very diligent about keeping yourself out of another person's life trajectory. When people bring us problems, or are clearly searching for something in their lives, it's extremely easy for us to project and assume that what we would want in the situation is what they would want - this must be avoided at all costs \cite{stanier}. Just as before when we noted that you rarely really know what your users want, your opinions on something will rarely line up perfectly with your coachee's opinions - and it's their life not yours that you're trying to help them build. Therefore it's paramount to become good at truly listening and working to understand what your coachee is actually after. Doing so is a difficult and nuanced skill in and of itself so I will defer to "The Coaching Habit: Say Less, Ask More, \& Change the Way You Lead" \cite{stanier} for lots of great and succinct advice on how to be a better listener within the context of coaching. But the bottom line is this - understand it will be your natural tendency to project yourself onto others' situations so do all you can to fight that tendency and thereby learn what's really going on.
\newline

All in all then, coaching is a tricky business. It requires getting yourself out of the way and focusing on learning what your coachee really is after and then, rather than giving them the solution, guiding them to their own solution all while trying to strike a balance between helping them learn while also not overwhelming them with new skills and techniques. It's about understanding the whole person in order to get a sense of their specific context and it requires being that person's champion no matter what. But in the end, if you're successful it means the person walks away confident in the skills necessary to build a successful and fulfilling life. 
\newline

However all of this should be sounding a little bit too rosy to you. Does any coach actually have \textit{all} of the skills needed to live a successful life? Is there really anyone out there with "The Meaning of Life and How to Achieve It"? Not at all. So what are we to do? 

\section{Self Healing}
Imagine an organization that's filled with learning and coaching. Everyone in the organization recognizes the importance of these meta-skills and the necessity of continuous, life long learning. Each time anyone in the organization learns something new and useful they go looking for folks who could take advantage of it and either teach or coach those skills. And every time someone in the organization runs across an issue, they go looking for something to learn rather than trying to reinvent the wheel. In this organization it doesn't really matter that no one person has all the answers because people are learning and growing all the time and the learning one person receives gets passed around to those who need it. In other words, while the organization is not perfect it is constantly healing any issues with itself. Put more directly, ubiquitous learning, teaching, and coaching are the means by which organizations \textit{self heal}. No one needs all the answers because people just need to keep looking for and distributing the answers they find, and, in the limit, the organization will become what it needs to be.
\newline

So how do we make this happen? How do we create a culture like this? Well think about it - all learning requires error, mistakes, and failure. Coaching requires loads of vulnerability, mutual respect, and honesty. Both require feedback from others that is often times critical and hard to hear. Both require acknowledging one's limitations and then looking to others for help. In other words coaching and teaching require feeling that you can be vulnerable around others and that they will be there for you. And human only vulnerable allow themselves to be vulnerable in either dire straits or around those they trust. Obviously the former is to be avoided and so trust is paramount. But trust is only built up through building honest and supportive relationships with others. In other words for effective learning and coaching positive relationships \textit{have} to be a central focus.
\newline

It's important to underscore just how powerful this principle is. Everything I've talked about in this book so far I've learned from others. One could say that you could remove all the other chapters and just ask the question - how do I learn and teach effectively. With just those two tools, over time you'd discover everything I've talked about here and more. Indeed this book is just the beginning for me! I'll keep learning long after I write the concluding sentences of this little jaunt through building organizations. Learning really is the key to \textit{everything}. And yet the only way effective learning happens between you and those around you is if you trust them and know they have your best interests at heart. So, if learning is the key to building great organizations, and positive relationships are the key to effective and open learning, then the single most important thing you can do to build great organizations is to build great relationships with those you work with.
\newline

Friendship is the cornerstone of greatness.

\chapter{Distributed and Co-Active}
We began with a simple question - how does one build effective organizations? Rather than starting at the top, we started at the bottom asking how to help each individual person soar because ultimately that is the point of an organization - to harness the energy and potential of each individual. 
\newline

What we found was that people, by their very nature, need focus. It is when we're able to pour mind, body, and soul into a few selective things that we are able to be our best selves. Moreover we've seen that trying to spread ourselves thin across a wide variety of concerns doesn't increase our impact - it just means we do a lot of different things poorly. Building skills takes a lot of time and effort, using those skills takes a lot of time and effort, building the relationships required to take advantage of our skills takes a lot of time and effort, and so focus is key.
\newline

But this left us with a bit of a problem - if everyone is to be so laser focused, how can a wide, sprawling organization run? Does someone just have to bite the bullet and spread themselves thin at the expense of their own skill and well being? The answer ended up being no, but to see that we had to take a few more steps. 
\newline

First we dove into what it really means to build success and found that it \textit{requires} being a little entrepreneur for whatever value add you, as an individual, are driving after. Put another way, any effective organization is really composed of a whole slew of people tending to and developing their own piece of the pie. However we also got our first insight into the connective tissue that holds the organization together - relationships. Relationships, as it turns out, are the key to understanding the value we have to provide as well as understanding and developing the means of cooperation. It is through gathering together and coordinating users, benefactors, developers, and advisers around a specific problem that success is found. In theory a network of such nodes all working on their respective pieces but coordinating with one another wherever they overlap (one person as entrepreneur, another as user for example) removes the need for a global coordinator but it left us with a new problem - in such changing seas how does one stay motivated and on track?
\newline

This then led us to gamification and the search for meaning. On the one hand gamification pointed out that it's not so much the specific goals that matter but the nature of how you play - incorporate the means for flow and find yourself having fun pursuing the adventure. Indeed, games show us just how much fun we can have even when the end is uncertain and how embracing the adventure can make us more resilient and optimistic. In similar fashion we found that finding our "north star" in life is also more about the journey than the destination - meaning is what we strive towards, it is about being reflective on what has had meaning to us and positively and constructively building a narrative around those things. As such we found that, overall, adventure doesn't put motivation at risk but rather motivation comes from a healthy dose of adventure! We are beings that crave striving toward something greater and finding flow in the process, and so, as long as we keep the toolkits of game design, positive pyschology, and things like logotherapy close at hand, following our adventure is actually what we want.
\newline

At this point it felt like we had the key - a distributed network of people, working hand in hand, following each of their individual adventures, and relishing the challenge presented to them. However all of this seemed to depend on a heady and unrealistic requirement - that each and every individual had a full toolkit for being successful in this distributed sense. Not only is that a lot to ask of the people - it's a lot to ask of the toolkit! This book has been nothing more but a general map of the space and actually applying any of this would require many more tomes of specific advice and application. Not only would that be a lot to learn, but I'd warrant most of the information doesn't exist in a tidy form just yet. 
\newline

However we still had a final arrow in our quiver - learning, teaching, and coaching. And, as we realized, if a culture of ubiquitous, continuous, honest, patient, and humble learning and coaching is developed then the organization will "self-heal". Put more specifically, if, when people encounter problems or see others struggling, they look for what they need to learn or teach to help resolve the issue, then, in the limit, the organization organically learns all the things it needs to be healthy, happy, and effective. But as we realized, such behavior requires a great deal of respect, trust, and vulnerability - and so we found that the basic building block of a healthy organization is healthy interpersonal relationships - i.e. friendship.
\newline

At this point we had our organization - a distributed web of entrepreneurial nodes, working together, continuously learning, and coaching and learning from one another. Everyone in this organization has focus, can specialize, feels supported by those around them, and is ready to step in and help where needed. Everyone feels motivated and is ready to take on the adventure. Everyone is reflecting on what really matters and through the interpersonal relationships they've developed are creating a collective sense of direction and purpose. Should the organization come into difficulty, the individuals within the organization who are effected reach out to one another and the greater body of human knowledge and skill to resolve the problem - and then whatever they learn makes its way through the web of learning-first relationships until the whole organization has absorbed the requisite skills. The organization has become a kind of super-organism - no one owns it, no one directly "runs" it, it arises as an organic result of the network of motivated, empowered, individuals of which it is composed. 
\newline

But I think what's most empowering to me is the first person view. Looking back now I can see clearly that the key to building great organizations is to not worry about them at all. Instead what matters is focusing on the local view - finding my own focus, finding my own meaning, and pursuing those things with all my gusto and strength while acknowledging all the growing I can and want to do. It's about pulling together the people who can help catalyze my own ideas and skills - the people who will benefit from what I do, the folks I'll build with, the benefactors who will help make it happen, and the advisers who can coach me (and eventually I can help coach). It's about building solid, trusting relationships with those people and looking out for one another - coaching one another whenever we collectively discover something new or interesting or when one of us is struggling. By pursuing my own adventure, finding my own focus, caring for those around me (and thereby helping them with their own adventures), and in all things seeing challenge as an opportunity for growth - I become a unit from which empowering, healthy organizations can grow and evolve. And in so doing I also sow the seeds for my own success and acknowledge and catalyze my own, individual, human psychology. 
\newline

This is the distributed, co-active model of organization development. Distributed in the sense that the organization grows from individual, distributed action, and co-active in the sense that everyone recognizes their own success comes from co-actively working with others and growing together. It is a model where everyone practices self mastery, everyone gets to enjoy their own adventure, and we pursue that adventure side by side with friends. It is, quite simply, a path to relish.
\newline

So follow your adventure, pull in and co-actively work with others, never stop learning, and above all things care for one another. Do this and the organization will build itself.
\newline

Shoot for the moon and \textit{together} you'll land among the stars.

\bibliographystyle{plain}
\bibliography{reference}
\end{document}